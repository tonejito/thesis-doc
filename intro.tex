\label{chap:intro}
\chapter*{Introducci\'{o}n}
\addcontentsline{toc}{chapter}{Introducci\'{o}n}

%  \addcontentsline{toc}{section}{Objetivo}
  \section*{Objetivo}

Implementar un servidor dedicado de almacenamiento que incluya las cuentas de usuario desde un directorio \textsc{LDAP} y que transmita la informaci\'{o}n por medio de \textsc{WebDAV} a trav\'{e}s de un canal cifrado.

%  \addcontentsline{toc}{section}{Estructura de la tesis}
  \section*{Estructura de la tesis}

    \subsection*{Cap\'{i}tulo~\ref{chap:cap1}}

En el primer cap\'{i}tulo se da a conocer el funcionamiento y clasificaci\'{o}n de los medios de almacenamiento que se utilizan en la actualidad, se enuncian los mecanismos de protecci\'{o}n para los medios seg\'{u}n su tipo y se describen las t\'{e}cnicas de respaldo que se utilizan para el almacenamiento en medios separados, respaldos en red y el uso de \textit{Cloud Storage}.

Por \'{u}ltimo se aborda una comparativa de los arreglos RAID simples y compuestos, se lista su funcionamiento y se presenta un diagrama que muestra el flujo de los datos en cada tipo de arreglo.

    \subsection*{Cap\'{i}tulo~\ref{chap:cap2}}

En el segundo cap\'{i}tulo se define el problema y se plantea la soluci\'{o}n utilizando un servicio de directorio para guardar a los usuarios, un servicio web que implementa el est\'{a}ndar \textsc{WebDAV} para acceder a los archivos y adem\'{a}s se mencionan las interfaces para administrar el directorio de usuarios y para que los usuarios puedan cambiar su contrase\~{n}a.

\newpage
    \subsection*{Cap\'{i}tulo~\ref{chap:cap3}}

En el tercer cap\'{i}tulo se realiza la configuraci\'{o}n de los servicios que forman parte de la soluci\'{o}n a implementar, se realiza la configuraci\'{o}n del directorio \textsc{LDAP} y se mencionan los mecanismos utilizados para cargar y borrar de forma masiva los usuarios en el directorio.

Se presenta tambi\'{e}n la configuraci\'{o}n del servicio de \textsc{HTTP} con el m\'{o}dulo de \textsc{WebDAV} y la conexi\'{o}n a \textsc{LDAP}, as\'{i} como la interfaz de administraci\'{o}n \textit{LDAP Account Manager} y la interfaz de cambio de contrase\~{n}a \textit{LDAP Toolbox}.

Por \'{u}ltimo se aplican las configuraciones de seguridad necesarias para restringir el acceso a la cuenta de administrador \texttt{root}, habilitar el envio de reportes de actividad inusual por correo electr\'{o}nico, habilitar un perfil de reglas de \textit{firewall} en el equipo y aplicar directivas de seguridad para el servicio \textsc{HTTP}.

    \subsection*{Cap\'{i}tulo~\ref{chap:cap4}}

En el cuarto cap\'{i}tulo se muestran las pruebas realizadas tomando como grupo de control los cursos intersemestrales que ofrece la \textsl{Unidad de C\'{o}mputo} donde se verific\'{o} la compatibilidad de la soluci\'{o}n con diversas plataformas de escritorio como GNU/Linux, Mac OS X y Windows y sistemas operativos de m\'{o}viles como Apple iOS y Android.

    \subsection*{Cap\'{i}tulo~\ref{chap:cap5}}

En el quinto cap\'{i}tulo se presentan las conclusiones y se listan los resultados obtenidos durante la implementaci\'{o}n y pruebas de la soluci\'{o}n propuesta, as\'{i} como las oportunidades de mejora que puede tener el proyecto en futuras versiones.

