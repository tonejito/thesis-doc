{
  \linespread{1}
  \cleardoublepage  
  \chapter{Implementaci\'{o}n de la soluci\'{o}n}
  \label{chap:cap3}
}

    \section {Configuraci\'{o}n del sistema operativo}

El sistema operativo que ser\'{a} utilizado para la implementaci\'{o}n es \textsc{Debian GNU/Linux} 7 \textit{\guillemotleft wheezy\guillemotright} que se instalar\'{a} de manera nativa en el servidor y se le aplicar\'{a}n configuraciones personalizadas para ejecutar el software que ser\'{a} implementado, ajustar el rendimiento y reforzar la seguridad de los servicios que ofrezca.

      \subsection {Arreglo de discos \textsc{RAID}}

Los discos duros del servidor ser\'{a}n configurados en un arreglo \textsc{RAID} debido a las ventajas que ofrece\footnote{Ver p\'{a}gina \pageref{Arreglos-RAID} secci\'{o}n \ref{Arreglos-RAID}}. La configuraci\'{o}n en el servidor de pruebas se realizar\'{a} con un arreglo \textsc{RAID} 1 por \textit{software} que se configur\'{o} de la siguiente manera: \footnote{https://wiki.debian.org/SoftwareRAID}

\begin{itemize}
  \item Instalar el software para habilitar la compatibilidad con arreglos \textsc{RAID} por software:
    \code{\# apt-get install mdadm}
  \item Crear las particiones de tipo \textit{Linux RAID Autodetect} en cada disco del arreglo
    \begin{itemize}
      \item Crear una partici\'{o}n primaria que ocupe todo el espacio del disco:

{
\scriptsize
\linespread{1}
\begin{verbatim}
    # /sbin/fdisk /dev/sdb

    Device contains neither a valid DOS partition table, nor Sun, SGI or OSF disklabel
    Building a new DOS disklabel with disk identifier 0x00bab10c.
    Changes will remain in memory only, until you decide to write them.
    After that, of course, the previous content won't be recoverable.

    Warning: invalid flag 0x0000 of partition table 4 will be corrected by w(rite)

    Command (m for help): n
    Partition type:
       p   primary (0 primary, 0 extended, 4 free)
       e   extended
    Select (default p): p
    Partition number (1-4, default 1): 1
    First sector (2048-16777215, default 2048):
    Using default value 2048
    Last sector, +sectors or +size{K,M,G} (2048-16777215, default 16777215):
    Using default value 16777215
\end{verbatim}
}

      \item Cambiar el tipo de la partici\'{o}n a \texttt{0xfd} - \textit{Linux RAID Autodetect} y escribir la tabla de particiones al disco:

{
\scriptsize
\linespread{1}
\begin{verbatim}
    Command (m for help): t
    Selected partition 1
    Hex code (type L to list codes): fd
    Changed system type of partition 1 to fd (Linux RAID Autodetect)

    Command (m for help): w
    The partition table has been altered!

    Calling ioctl() to re-read partition table.
    Syncing disks.
\end{verbatim}
}

    \end{itemize}
  \item Crear arreglo:
    \code{\# mdadm --zero-superblock /dev/sdb1 /dev/sdc1}
    \code{\# mdadm --create /dev/md0 --level=1 --raid-devices=2 /dev/sdb1 /dev/sdc1}
  \item Formatear el disco que representa el arreglo:
    \code{\# mkfs.ext4 /dev/md0}
  \item Asignar tipo y punto de montaje en el archivo \texttt{/etc/fstab}:
    \code{/dev/md0 /opt/xNAS/files ext4 noatime,rw 0 0}
  \item Guardar configuraci\'{o}n en el archivo \texttt{/etc/mdadm/mdadm.conf}:
    \code{ARRAY /dev/md0 devices=/dev/sdb1,/dev/sdc1 level=1 num-devices=2 auto=yes}
    \code{DEVICE /dev/sdb1 /dev/sdc1}
\end{itemize}

      \subsection {Punto de montaje de s\'{o}lo lectura}

Para evitar el acceso de escritura a los archivos pertenecientes a los profesores, se realiz\'{o} un montaje de tipo \textit{bind} con permisos de s\'{o}lo lectura. Debido a que se realizan dos operaciones de montaje, esta funcionalidad se implement\'{o} dentro del \textit{script} \texttt{/etc/rc.local}.

% http://linux.die.net/man/8/mount
% https://lwn.net/Articles/281157/
% http://karelzak.blogspot.mx/2011/04/bind-mounts-mtab-and-read-only.html

{
\scriptsize
\linespread{1}
\begin{verbatim}
PREFIX=/opt/xNAS/files
mount --bind $PREFIX/profesor $PREFIX/p
mount -o remount,ro,bind $PREFIX/profesor $PREFIX/p
\end{verbatim}
}

    \section {Configuraci\'{o}n de los servicios}

      \subsection {OpenLDAP}

          \subsubsection {Instalaci\'{o}n de \textsc{OpenLDAP}}

Se instala \textsc{OpenLDAP} utilizando el comando \texttt{aptitude}, es necesario contestar \textbf{NO} en el cuadro de di\'{a}logo para seguir el asistente de instalaci\'{o}n:

{
\scriptsize
\linespread{1}
\begin{verbatim}
    # aptitude install slapd
\end{verbatim}
}


{
\scriptsize
\linespread{1}
\begin{verbatim}
+---------------------------+ Configuring slapd +----------------------------+
| If you enable this option, no initial configuration or database will be    |
| created for you.                                                           |
|                                                                            |
| Omit OpenLDAP server configuration?                                        |
|                    <Yes>                       <No>                        |
+----------------------------------------------------------------------------+
\end{verbatim}
}

Se pide el nombre DNS del dominio que servir\'{a} para conformar la ra\'{i}z del \'{a}rbol de \textsc{LDAP}.

{
\scriptsize
\linespread{1}
\begin{verbatim}
+---------------------------+ Configuring slapd +----------------------------+
| The DNS domain name is used to construct the base DN of the LDAP           |
| directory. For example, 'foo.example.org' will create the directory with   |
| 'dc=foo, dc=example, dc=org' as base DN.                                   |
|                                                                            |
| DNS domain name:                                                           |
| xnas.local________________________________________________________________ |
|                                   <Ok>                                     |
+----------------------------------------------------------------------------+
\end{verbatim}
}

Adicionalmente el instalador pregunta por el nombre de la organizaci\'{o}n, aunque este dato no es necesario, se recomienda introducirlo:

{
\scriptsize
\linespread{1}
\begin{verbatim}
+---------------------------+ Configuring slapd +----------------------------+
| Please enter the name of the organization to use in the base DN of your    |
| LDAP directory.                                                            |
|                                                                            |
| Organization name:                                                         |
| xNAS_____________________________________________________________________  |
|                                  <Ok>                                      |
+----------------------------------------------------------------------------+
\end{verbatim}
}

A continuaci\'{o}n se pide que se introduzca la contrase\~{n}a que ser\'{a} utilizada por el administrador del directorio:

{
\scriptsize
\linespread{1}
\begin{verbatim}
+-------------------------+ Configuring slapd +------------------------------+
| Please enter the password for the admin entry in your LDAP directory.      |
|                                                                            |
| Administrator password:                                                    |
| **********____________________________________________________________     |
|                                 <Ok>                                       |
+----------------------------------------------------------------------------+

+---------------------------+ Configuring slapd +----------------------------+
| Please enter the admin password for your LDAP directory again to verify    |
| that you have typed it correctly.                                          |
|                                                                            |
| Confirm password:                                                          |
| **********_______________________________________________________________  |
|                                  <Ok>                                      |
+----------------------------------------------------------------------------+
\end{verbatim}
}

Se recomienda deshabilitar el soporte del protocolo \textsc{LDAPv2} puesto que es obsoleta, responder \textbf{NO} en el cuadro de di\'{a}logo.

{
\scriptsize
\linespread{1}
\begin{verbatim}
+----------------------------+ Configuring slapd +---------------------------+
|                                                                            |
| The obsolete LDAPv2 protocol is disabled by default in slapd. Programs and |
| users should upgrade to LDAPv3.  If you have old programs which can't use  |
| LDAPv3, you should select this option and 'allow bind_v2' will be added to |
| your slapd.conf file.                                                      |
|                                                                            |
| Allow LDAPv2 protocol?                                                     |
|                     <Yes>                        <No>                      |
+----------------------------------------------------------------------------+
\end{verbatim}
}

          \subsubsection {Configuraci\'{o}n de \textsc{OpenLDAP}}

Se configur\'{o} el sistema operativo para permitir que los usuarios de \textsc{LDAP} de tipo \textit{posixAccount} puedan iniciar sesi\'{o}n en el equipo.

{
\scriptsize
\linespread{1}
\begin{verbatim}
  # aptitude install libpam-ldapd
\end{verbatim}
}

Adicionalmente se instalaron los demonios \textsc{NSCD} y \textsc{NSLCD} que se utilizan para guardar en \textit{cache} los resultados de las consultas al directorio y para representar los objetos \textit{posixAccount} de \textsc{LDAP} como cuentas est\'{a}ndar de \textsc{UNIX}.

{
\scriptsize
\linespread{1}
\begin{verbatim}
+------------------------+ Configuring libnss-ldapd +-------------------------+
| For this package to work, you need to modify your /etc/nsswitch.conf to     |
| use the ldap datasource.                                                    |
|                                                                             |
| You can select the services that should have LDAP lookups enabled. The new  |
| LDAP lookups will be added as the last datasource. Be sure to review these  |
| changes.                                                                    |
|                                                                             |
| Name services to configure:                                                 |
|    [*] group                                                                |
|    [*] passwd                                                               |
|    [*] shadow                                                               |
|                                   <Ok>                                      |
+-----------------------------------------------------------------------------+
\end{verbatim}
}

Para la configuraci\'{o}n de \textsc{PAM}, se indica que se utilizar\'{a}n como fuentes de autenticaci\'{o}n la base de datos est\'{a}ndar de UNIX y adicionalmente el directorio LDAP.

{
\scriptsize
\linespread{1}
\begin{verbatim}
+----------------------------+ PAM configuration +----------------------------+
| Pluggable Authentication Modules (PAM) determine how authentication,        |
| authorization, and password changing are handled on the system, as well as  |
| allowing configuration of additional actions to take when starting user     |
| sessions.                                                                   |
|                                                                             |
| Some PAM module packages provide profiles that can be used to               |
| automatically adjust the behavior of all PAM-using applications on the      |
| system.  Please indicate which of these behaviors you wish to enable.       |
|                                                                             |
| PAM profiles to enable:                                                     |
|    [*] Unix authentication                                                  |
|    [*] LDAP Authentication                                                  |
|                    <Ok>                        <Cancel>                     |
+-----------------------------------------------------------------------------+
\end{verbatim}
}

          \subsubsection {Inicializaci\'{o}n del directorio \textsc{LDAP}}

Una vez instalado el servicio de directorio, es necesario inicializar la estructura b\'{a}sica que comprende los contenedores de usuarios, materias y grupos utilizando el archivo \texttt{xNAS-base.ldif} y el comando \texttt{ldapadd}. En caso de requerir cambiar la contrase\~{n}a del administrador del directorio, seguir las indicaciones mostradas en la URL de la nota al pie\footnote{http://techiezone.rottigni.net/2011/12/change-root-dn-password-on-openldap/}.

% http://www.linuxtopia.org/online_books//network_administration_guides/ldap_administration/slapdconf2_Configuration_Directives.html

{
\scriptsize
\linespread{1}
\begin{verbatim}

# aptitude install ldap-utils
 \end{verbatim}
}

Se verifica que no existan objetos adicionales en el directorio, la salida del comando \texttt{ldapsearch} debe ser similar a la que se muestra a continuaci\'{o}n:

{
\scriptsize
\linespread{1}
\begin{verbatim}

# ldapsearch -LLL -Y EXTERNAL -H ldapi:/// -b 'dc=xnas,dc=local'
SASL/EXTERNAL authentication started
SASL username: gidNumber=0+uidNumber=0,cn=peercred,cn=external,cn=auth
SASL SSF: 0
dn: dc=xnas,dc=local
objectClass: top
objectClass: dcObject
objectClass: organization
o: xNAS
dc: xnas

dn: cn=admin,dc=xnas,dc=local
objectClass: simpleSecurityObject
objectClass: organizationalRole
cn: admin
description: LDAP administrator
 \end{verbatim}
}

Se procede con la inicializaci\'{o}n de la estructura b\'{a}sica del directorio.

{
\scriptsize
\linespread{1}
\begin{verbatim}

# ldapadd -x -W -D "cn=admin,dc=xnas,dc=local" -H 'ldapi:///' -f ./xNAS-base.ldif
Enter LDAP Password: 
adding new entry "ou=services,dc=xnas,dc=local"
adding new entry "ou=materias,dc=xnas,dc=local"
adding new entry "ou=users,dc=xnas,dc=local"
adding new entry "ou=staff,ou=users,dc=xnas,dc=local"
adding new entry "ou=profesores,ou=users,dc=xnas,dc=local"
adding new entry "ou=grupos,ou=users,dc=xnas,dc=local"
adding new entry "ou=alumnos,ou=users,dc=xnas,dc=local"
adding new entry "ou=groups,dc=xnas,dc=local"
adding new entry "ou=unix,ou=groups,dc=xnas,dc=local"
adding new entry "ou=webdav,ou=groups,dc=xnas,dc=local"
\end{verbatim}
}

          \subsubsection {Carga de datos en el directorio \textsc{LDAP}}

Para realizar la carga de la base de datos de usuarios y grupos se desarroll\'{o} una biblioteca y un script de carga en el lenguaje de programaci\'{o}n \textsl{Ruby} que lee los datos desde un archivo origen en formato \texttt{CSV}, establece las relaciones entre los objetos y realiza el ingreso de los datos al directorio.

La biblioteca que realiza la carga de objetos funciona de acuerdo al siguiente algoritmo:

\begin{itemize}
  \item Realiza una conexi\'{o}n al directorio LDAP, a esta operaci\'{o}n se le denomina \textit{bind}.
  \item Convierte el archivo de entrada a la codificaci\'{o}n \textsc{UTF-8}.
  \item Lee cada rengl\'{o}n del archivo de entrada y verifica el contenido de cada campo contra una \textit{expresi\'{o}n regular} para identificar problemas.
  \item Si el rengl\'{o}n cumple con el formato, se contin\'{u}a con el pr\'{o}ximo paso, de lo contrario se guarda en una lista que debe ser revisada manualmente y salta al siguiente.
  \item Asigna los atributos de cada objeto LDAP de acuerdo al valor de cada campo.
  \item Establece las relaciones necesarias con otros objetos y realiza una transacci\'{o}n para insertar los datos en el directorio.
  \item Verifica que la inserci\'{o}n se haya realizado correctamente, en caso de existir alg\'{u}n error, se realiza un \textit{rollback} de las operaciones.
\end{itemize}

\diagramblock
{Diagrama de bloques de los scripts de carga}
{DiagramaScript}
{
 \psscalebox{1.0 1.0} % Change this value to rescale the drawing.
 {
    \begin{pspicture}(0,-2.275)(11.5,2.275)
  \rput[bl](0.18,0.40){\scriptsize \textit{Archivos de entrada}}
  \rput[bl](8.795,0.8467391){\scriptsize \textit{Archivos de salida}}
  \rput[bl](6.85,-1.57){\scriptsize \textit{Salida de error}}
  \rput[bl](4.92,0.20326087){\texttt{script.rb}}
  \rput[bl](4.6,-0.225){\scriptsize \texttt{require xNAS.rb}}
  \rput[bl](4.39,1.6902174){Directorio \textup{LDAP}}
  \rput[bl](8.88,0.4036232){Config Apache}
  \rput[bl](8.83,0.05507246){Bit\'{a}cora \texttt{mkdir}}
  \rput[bl](2.515,-1.57){\scriptsize \textsl{Revisi\'{o}n manual}}
  \rput[bl](0.875,-0.1576087){\textup{CSV} \scriptsize(4)}
  \rput[bl](9.375,-0.4576087){\textup{CSV} \scriptsize(4)}
  \rput[bl](7.225,-2.025){\textup{CSV} \scriptsize(4)}
  \rput[bl](3.21,-2.025){\textup{CSV}}
  \psframe[linecolor=black, linewidth=0.04, dimen=outer](7.25,0.775)(4.25,-0.425)
  \psframe[linecolor=black, linewidth=0.04, dimen=outer](7.25,2.275)(4.25,1.375)
  \psframe[linecolor=black, linewidth=0.04, dimen=outer](3.0,0.775)(0.0,-0.425)
  \psframe[linecolor=black, linewidth=0.04, dimen=outer](9.35,-1.075)(6.35,-2.275)
  \psframe[linecolor=black, linewidth=0.04, dimen=outer](11.5,1.175)(8.5,-0.525)
  \psframe[linecolor=black, linewidth=0.04, linestyle=dashed, dash=0.17638889cm 0.10583334cm, dimen=outer](5.1,-1.075)(2.1,-2.275)
  \psline[linecolor=black, linewidth=0.04, arrowsize=0.05291666666666667cm 2.0,arrowlength=1.4,arrowinset=0.0]{<->}(5.638506,1.3905172)(5.638506,0.7468391)
  \psline[linecolor=black, linewidth=0.04, arrowsize=0.05291666666666667cm 2.0,arrowlength=1.4,arrowinset=0.0]{->}(7.22971,0.540942)(8.505073,0.540942)
  \psline[linecolor=black, linewidth=0.04, arrowsize=0.05291666666666667cm 2.0,arrowlength=1.4,arrowinset=0.0]{->}(7.22971,0.16413043)(8.505073,0.16413043)
  \psline[linecolor=black, linewidth=0.04, arrowsize=0.05291666666666667cm 2.0,arrowlength=1.4,arrowinset=0.0]{->}(7.227778,-0.175)(8.516666,-0.175)
  \psline[linecolor=black, linewidth=0.04, arrowsize=0.05291666666666667cm 2.0,arrowlength=1.4,arrowinset=0.0]{->}(2.97,0.13166666)(4.25,0.13166666)
  \psline[linecolor=black, linewidth=0.04, linestyle=dashed, dash=0.17638889cm 0.10583334cm, arrowsize=0.05291666666666667cm 2.0,arrowlength=1.4,arrowinset=0.0]{->}(3.6366668,-1.0683334)(3.6366668,0.13166666)
  \psline[linecolor=black, linewidth=0.04, linestyle=dashed, dash=0.17638889cm 0.10583334cm, arrowsize=0.05291666666666667cm 2.0,arrowlength=1.4,arrowinset=0.0]{->}(6.3727274,-1.675)(5.077273,-1.675)
  \psline[linecolor=black, linewidth=0.04, dotsize=0.07055555555555555cm 2.0,arrowsize=0.05291666666666667cm 2.0,arrowlength=1.4,arrowinset=0.0]{*->}(7.85,-0.175)(7.85,-1.0861111)
  \end{pspicture}

 }
}

Cada archivo de entrada \texttt{CSV} tiene varias columnas que contienen informaci\'{o}n acerca del tipo de objeto que representa, a continuaci\'{o}n se muestran las columnas que se requieren en los archivos de entrada:

{
 \begin{table}[H]
 \caption{Formato de los archivos \textsl{CSV}}{}
 \label{tab:csv-format}
 \noindent\makebox[\textwidth]
 {%
  \begin{tabular}[c]{c|c c c c c}
  %\hline
  \textbf{Archivo} & \multicolumn{5}{c}{\textbf{Columnas}} \\
  \hline \hline
  \textit{staff.csv} & usuario & nombre & correo & curp & \\
  \textit{materias.csv} & id & grupo & materia & rfc & profesor \\
  \textit{profesores.csv} & rfc & nombre & correo & & \\
  \textit{alumnos.csv} & cuenta & nombre & correo & asignatura & grupo \\
  %\hline
  \end{tabular}
 } % ending of \makebox
 \end{table}
}

Para la correcta ejecuci\'{o}n de los scripts es necesario instalar los paquetes \texttt{ruby}, \texttt{rubygems} y la gema de ruby que realiza la conexi\'{o}n con el directorio \textsc{LDAP}.

{
\scriptsize
\linespread{1}
\begin{verbatim}
  # aptitude install ruby rubygems ruby-json
  # gem install net-ldap
\end{verbatim}
}

Utilizando el script \texttt{./load.rb} se cargan los datos y se crean los objetos en el directorio:

{
\scriptsize
\linespread{1}
\begin{verbatim}
# ./load.rb

xNAS - ./load.rb

Enter username, press <ENTER> for default:
    username    [cn=admin]

Enter password (will not echo):
    password
\end{verbatim}
}


{
 \begin{table}[H]
 \caption{Script de carga de objetos en el directorio}{}
 \label{tab:load-rb}
 \noindent\makebox[\textwidth]
 {%
  \begin{tabular}[c]{c|c c}
  %\hline
  \textbf{Tipo} & \multicolumn{2}{c}{\textbf{Objetos creados en el directorio}} \\
  \hline \hline
  \textit{staff} & \textit{posixGroup} & \textit{posixAccount} \\
  \textit{profesores} & \textit{posixGroup} & \textit{posixAccount} \\
  \textit{materias} & \textit{organizationalRole} & \textit{groupOfNames} \\
  \textit{alumnos} & \multicolumn{2}{c}{\textit{simpleSecurityObject}} \\
  %\hline
  \end{tabular}
 } % ending of \makebox
 \end{table}
}

          \subsubsection {Borrado de datos en el directorio \textsc{LDAP}}

Se desarroll\'{o} un script que ayuda a limpiar el directorio cuando se pretenda cargar una nueva base de datos en el mismo.

{
\scriptsize
\linespread{1}
\begin{verbatim}
  # make clean
  # ./clean.rb

xNAS - ./clean.rb

Enter username, press <ENTER> for default:
    username	[cn=admin]

Enter password (will not echo):
    password

\end{verbatim}
}

      \subsection {Apache httpd}

        \subsubsection {Esquema de configuraci\'{o}n}

La configuraci\'{o}n de Apache \textit{httpd} comprende tres \texttt{VirtualHost} que sirven como puntos de entrada para el acceso de s\'{o}lo lectura o lectura y escritura a los archivos almacenados en el servidor. Se realiz\'{o} el registro de los nomrbres de dominio con \textsc{NicUNAM}, gracias a esto se pueden acceder desde cualquier navegador web.

{
 \begin{table}[H]
 \caption{VirtualHost configurados en Apache HTTPD}{}
 \label{tab:virtualhost}
 \noindent\makebox[\textwidth]
 {%
  \begin{tabular}[c]{c|c}
  %\hline
  \textbf{VirtualHost} & \textbf{Funci\'{o}n} \\
  \hline \hline
  \texttt{xnas.local} & Acceso a los archivos por medio de \textsc{WebDAV} \\
  \texttt{reset.xnas.local} & Interfaz de cambio de contrase\~{n}a \\
  \texttt{admin.xnas.local} & Interfaz administrativa del directorio \textsc{LDAP} \\
  %\hline
  \end{tabular}
 } % ending of \makebox
 \end{table}
}

\diagramblock
{Diagrama Apache HTTPD VirtualHost}
{DiagramaVirtualHost}
{
 \psscalebox{1.0 1.0} % Change this value to rescale the drawing.
 {
    \begin{pspicture}(0,-1.8570131)(14.305195,1.8570131)
  \rput[bl](0.48181817,1.6170132){Dominio DNS}
  \rput[bl](0.99,0.41701317){\small \texttt{xnas.local}}
  \rput[bl](-0.2,-0.68298686){\small \texttt{reset.xnas.local}}
  \rput[bl](-0.2,-1.3829868){\small \texttt{admin.xnas.local}}
  \rput[bl](3.1900268,1.1167102){\tiny 443/tcp}
  \rput[bl](5.120027,-0.05571411){Apache}
  \rput[bl](5.050027,-0.564502){HTTPD}
  \rput[bl](7.9379673,0.46626446){\textsc{WebDAV}}
  \rput[bl](8.367968,-0.9837355){\textsc{PHP}}
  \rput[bl](10.681818,0.61701316){Archivos}
  \rput[bl](10.672728,0.1533768){Carpetas}
  \rput[bl](10.781818,-0.6466232){\scriptsize \textsc{LDAP} Toolbox}
  \rput[bl](10.772727,-1.4102596){\scriptsize \textsc{LDAP} Account Manager}
  \psframe[linecolor=black, linewidth=0.04, dimen=outer](6.773262,1.095088)(4.5967913,-1.5284414)
  \psframe[linecolor=black, linewidth=0.04, dimen=outer](9.708556,1.1186174)(7.7673798,0.053911556)
  \psframe[linecolor=black, linewidth=0.04, dimen=outer](9.708556,-0.18138257)(7.7673798,-1.5460885)
  \psframe[linecolor=black, linewidth=0.04, dimen=outer](14.305195,1.4287015)(4.3623376,-1.8570129)
  \psline[linecolor=black, linewidth=0.04, arrowsize=0.05291666666666667cm 2.0,arrowlength=1.4,arrowinset=0.0]{->}(3.090909,0.5134941)(4.3715544,0.5134941)
  \psline[linecolor=black, linewidth=0.04, linestyle=dotted, dotsep=0.10583334cm, arrowsize=0.05291666666666667cm 2.0,arrowlength=1.4,arrowinset=0.0]{->}(3.090909,-1.2736027)(4.3715544,-1.2736027)
  \psline[linecolor=black, linewidth=0.04, linestyle=dashed, dash=0.17638889cm 0.10583334cm, arrowsize=0.05291666666666667cm 2.0,arrowlength=1.4,arrowinset=0.0]{->}(3.090909,-0.5865059)(4.3715544,-0.5865059)
  \psline[linecolor=black, linewidth=0.04, arrowsize=0.05291666666666667cm 2.0,arrowlength=1.4,arrowinset=0.0]{->}(6.7909093,0.5134941)(7.7715545,0.5134941)
  \psline[linecolor=black, linewidth=0.04, linestyle=dotted, dotsep=0.10583334cm, arrowsize=0.05291666666666667cm 2.0,arrowlength=1.4,arrowinset=0.0]{->}(6.7909093,-1.2736027)(7.7715545,-1.2736027)
  \psline[linecolor=black, linewidth=0.04, linestyle=dashed, dash=0.17638889cm 0.10583334cm, arrowsize=0.05291666666666667cm 2.0,arrowlength=1.4,arrowinset=0.0]{->}(6.7909093,-0.5865059)(7.7715545,-0.5865059)
  \psline[linecolor=black, linewidth=0.04, arrowsize=0.05291666666666667cm 2.0,arrowlength=1.4,arrowinset=0.0]{->}(9.690909,0.5134941)(10.671555,0.5134941)
  \psline[linecolor=black, linewidth=0.04, linestyle=dotted, dotsep=0.10583334cm, arrowsize=0.05291666666666667cm 2.0,arrowlength=1.4,arrowinset=0.0]{->}(9.690909,-1.2736027)(10.671555,-1.2736027)
  \psline[linecolor=black, linewidth=0.04, linestyle=dashed, dash=0.17638889cm 0.10583334cm, arrowsize=0.05291666666666667cm 2.0,arrowlength=1.4,arrowinset=0.0]{->}(9.690909,-0.5865059)(10.671555,-0.5865059)
  \end{pspicture}

 }
}

\diagramblock
{Diagrama de configuraci\'{o}n de Apache HTTPD}
{DiagramaConf}
{
 \psscalebox{1.0 1.0} % Change this value to rescale the drawing.
 {
    \begin{pspicture}(0,-2.3535714)(10.61,2.3535714)
  \rput[bl](4.2,2.0035715){\texttt{/opt/xNAS}}
  \rput[bl](0.0,1.0035714){apache2}
  \rput[bl](0.4,0.46357143){extra}
  \rput[bl](0.4,0.03357143){sites-available}
  \rput[bl](3.4,1.0035714){app}
  \rput[bl](3.7,0.45357144){static}
  \rput[bl](3.7,0.03357143){lam}
  \rput[bl](4.0510206,-0.46142858){htdocs}
  \rput[bl](4.0510206,-0.95357144){conf}
  \rput[bl](3.7,-1.3964286){ltb}
  \rput[bl](4.0510206,-1.8678571){htdocs}
  \rput[bl](4.0510206,-2.3535714){conf}
  \rput[bl](5.8,1.0035714){ssl}
  \rput[bl](6.1,0.46357143){certs}
  \rput[bl](6.1,0.0035714286){private}
  \rput[bl](8.2,1.0035714){files}
  \rput[bl](8.5,0.41357142){profesor}
  \rput[bl](8.5,0.03357143){staff}
  \rput[bl](8.5,-0.49642858){p \ding{222} profesor}
  \psline[linecolor=black, linewidth=0.04, tbarsize=0.07055555555555555cm 5.0]{-|*}(3.488889,0.98134923)(3.488889,0.13690476)
  \psline[linecolor=black, linewidth=0.04, tbarsize=0.07055555555555555cm 5.0]{-|*}(3.488889,0.98134923)(3.488889,0.53690475)
  \psline[linecolor=black, linewidth=0.04, tbarsize=0.07055555555555555cm 5.0]{-|*}(3.488889,0.98134923)(3.488889,-1.2630953)
  \psline[linecolor=black, linewidth=0.04, tbarsize=0.07055555555555555cm 5.0]{-|*}(5.888889,0.98134923)(5.888889,0.13690476)
  \psline[linecolor=black, linewidth=0.04, tbarsize=0.07055555555555555cm 5.0]{-|*}(8.288889,0.98134923)(8.288889,0.13690476)
  \psline[linecolor=black, linewidth=0.04, tbarsize=0.07055555555555555cm 5.0]{-|*}(5.888889,0.98134923)(5.888889,0.53690475)
  \psline[linecolor=black, linewidth=0.04, tbarsize=0.07055555555555555cm 5.0]{-|*}(8.288889,0.98134923)(8.288889,0.53690475)
  \psline[linecolor=black, linewidth=0.04, tbarsize=0.07055555555555555cm 5.0]{-|*}(8.288889,0.98134923)(8.288889,-0.36309522)
  \psline[linecolor=black, linewidth=0.04, tbarsize=0.07055555555555555cm 5.0]{-|*}(0.18888889,0.98134923)(0.18888889,0.13690476)
  \psline[linecolor=black, linewidth=0.04, tbarsize=0.07055555555555555cm 5.0]{-|*}(0.18888889,0.98134923)(0.18888889,0.53690475)
  \psline[linecolor=black, linewidth=0.04, tbarsize=0.07055555555555555cm 5.0]{-|*}(3.788889,-0.018650794)(3.8,-0.8964286)
  \psline[linecolor=black, linewidth=0.04, tbarsize=0.07055555555555555cm 5.0]{-|*}(3.788889,-0.018650794)(3.788889,-0.35783207)
  \psline[linecolor=black, linewidth=0.04, tbarsize=0.07055555555555555cm 5.0]{-|*}(3.8888888,-1.4186507)(3.9,-2.2964287)
  \psline[linecolor=black, linewidth=0.04, tbarsize=0.07055555555555555cm 5.0]{-|*}(3.8888888,-1.4186507)(3.8888888,-1.757832)
  \psline[linecolor=black, linewidth=0.04, arrowsize=0.05291666666666667cm 2.0,arrowlength=1.4,arrowinset=0.0]{->}(5.0,1.979762)(5.0,1.5988095)(3.5952382,1.622619)(3.5952382,1.2892857)
  \psline[linecolor=black, linewidth=0.04, arrowsize=0.05291666666666667cm 2.0,arrowlength=1.4,arrowinset=0.0]{->}(5.0,1.9559524)(5.0,1.5988095)(5.8809524,1.5988095)(5.8809524,1.2178571)
  \psline[linecolor=black, linewidth=0.04, arrowsize=0.05291666666666667cm 2.0,arrowlength=1.4,arrowinset=0.0]{->}(5.0,1.9559524)(5.0,1.5988095)(8.380953,1.5988095)(8.380953,1.2416667)
  \psline[linecolor=black, linewidth=0.04, arrowsize=0.05291666666666667cm 2.0,arrowlength=1.4,arrowinset=0.0]{->}(5.0,1.9559524)(5.0,1.5988095)(0.2857143,1.6702381)(0.2857143,1.3130952)
  \end{pspicture}

 }
}

{
 \linespread{1}
 \begin{table}[H]
 \caption{Archivos de configuraci\'{o}n}{}
 \label{tab:system-config}
 \noindent\makebox[\textwidth]
 {%
  \begin{tabular}[c]{l|l}
  %\hline
  \multicolumn{1}{c}{\textbf{Liga simb\'{o}lica}} & \multicolumn{1}{c}{\textbf{Destino}} \\
  \hline \hline
  \scriptsize \texttt{/etc/apache2/extra}                            & \scriptsize \texttt{/opt/xNAS/apache2/extra} \\
  \scriptsize \texttt{/etc/apache2/sites-available/xnas.local}       & \scriptsize \texttt{/opt/xNAS/apache2/sites-available/default} \\
  \scriptsize \texttt{/etc/apache2/sites-available/reset.xnas.local} & \scriptsize \texttt{/opt/xNAS/apache2/sites-available/reset.xnas.local} \\
  \scriptsize \texttt{/etc/apache2/sites-available/admin.xnas.local} & \scriptsize \texttt{/opt/xNAS/apache2/sites-available/admin.xnas.local} \\
  \scriptsize \texttt{/etc/ssl/certs/ca.thesis.tonejito.info.crt}    & \scriptsize \texttt{/opt/xNAS/ssl/certs/ca.thesis.tonejito.info.crt} \\
  \scriptsize \texttt{/etc/ssl/certs/xnas.local.crt}                 & \scriptsize \texttt{/opt/xNAS/ssl/certs/xnas.local.crt} \\
  \scriptsize \texttt{/etc/ssl/certs/reset.xnas.local.crt}           & \scriptsize \texttt{/opt/xNAS/ssl/certs/reset.xnas.local.crt} \\
  \scriptsize \texttt{/etc/ssl/certs/admin.xnas.local.crt}           & \scriptsize \texttt{/opt/xNAS/ssl/certs/admin.xnas.local.crt} \\
  \scriptsize \texttt{/etc/ssl/certs/xnas.local.key}                 & \scriptsize \texttt{/opt/xNAS/ssl/certs/xnas.local.key} \\
  \scriptsize \texttt{/etc/ssl/certs/reset.xnas.local.key}           & \scriptsize \texttt{/opt/xNAS/ssl/certs/reset.xnas.local.key} \\
  \scriptsize \texttt{/etc/ssl/certs/admin.xnas.local.key}           & \scriptsize \texttt{/opt/xNAS/ssl/certs/admin.xnas.local.key} \\
  %\hline
  \end{tabular}
 } % ending of \makebox
 \end{table}
}

        \subsubsection {Configuraci\'{o}n del servicio}

Se instala el certificado \textsc{SSL} en \texttt{/etc/ssl/certs} y la llave privada en \texttt{/etc/ssl/private} mediante los siguientes comandos:

{
\scriptsize
\linespread{1}
\begin{verbatim}
    # cd /etc/ssl/private
    # ln -vsf ../../../opt/xNAS/ssl/xnas.local.key
    # cd /etc/ssl/certs
    # ln -vsf ../../../opt/xNAS/ssl/ca.thesis.tonejito.info.crt
    # ln -vsf ../../../opt/xNAS/ssl/xnas.local.crt
    # c_rehash
\end{verbatim}
}

Para configurar de manera adecuada la directiva \texttt{NameVirtualHost} para los sitios por \textsc{HTTPS} se copia el archivo \texttt{ports.conf} al directorio principal de Apache \textit{httpd}.

{
\scriptsize
\linespread{1}
\begin{verbatim}
    # cd /opt/xNAS/apache2
    # cp ports.conf /etc/apache2
    # /etc/init.d/apache2 restart
    # apache2ctl -S
\end{verbatim}
}

Se habilita el acceso a las configuraciones espec\'{i}ficas del \textit{appliance} al hacer una \textit{liga simb\'{o}lica} al directorio \texttt{extra}:

{
\scriptsize
\linespread{1}
\begin{verbatim}
    # cd /etc/apache2
    # ln -vsf ../../opt/xNAS/apache2/extra
\end{verbatim}
}

Se instalan y habilitan los m\'{odulos} necesarios para que las funcionalidades del sitio se realizen de manera adecuada, esto incluye la autenticaci\'{o}n y la funcionalidad de \textsc{WebDAV}. Por \'{u}ltimo se reinicia el servicio para aplicar los cambios.

{
\scriptsize
\linespread{1}
\begin{verbatim}
    # aptitude install libapache2-mod-rpaf
    # cd /etc/apache2/sites-available
    # ln -vsf ../../../opt/xNAS/apache2/xnas.local
    # a2dissite default
    # a2enmod headers ssl rewrite ldap authnz_ldap dav dav_fs dav_lock
    # a2enmod proxy proxy_http rpaf
    # a2ensite default
    # /etc/init.d/apache2 restart
\end{verbatim}
}

        \subsubsection {Compatibilidad con clientes \textsc{WebDAV}}

Aunque \textsc{WebDAV} es un est\'{a}ndar definido en el RFC4918 \footnote{http://tools.ietf.org/rfc/rfc4918.txt} existen varios clientes que ofrecen funcionalidades similares aunque son implementados de manera distinta, por ejemplo, los clientes de \textsc{Mac OS X} y \textsc{GNU/Linux} son los mas completos, mientras que el cliente nativo de Windows (\textit{Microsoft-WebDAV-MiniRedir}) tiene problemas de compatibilidad que lo hacen estar al borde de lo usable.

Se utilizaron las directivas \texttt{BrowserMatch} y \texttt{BrowserMatchNoCase} para distinguir a los clientes problem\'{a}ticos que se conectan al servidor y darles un tratamiento especial. \footnote{http://serverfault.com/questions/478528/webdav-with-apache-2-2-simply-wont-work} \footnote{http://blog.toxa.de/archives/387} A continuaci\'{o}n se muestran las directivas que se utilizaron para darle un tratamiento especial a los clientes \textsc{WebDAV} nativos de Windows.

{
\scriptsize
\linespread{1}
\begin{verbatim}
Header add MS-Author-Via "DAV" 
BrowserMatch "Microsoft-WebDAV-MiniRedir" redirect-carefully nokeepalive ssl-unclean-shutdown 
\end{verbatim}
}

Para el caso espec\'{i}fico de Windows, es necesario deshabilitar la detecci\'{o}n autom\'{a}tica de proxy en la configuraci\'{o}n de \textit{Internet Explorer}, esto debido a que de manera predeterminada intenta encontrar un servidor proxy \textit{en cada petici\'{o}n que realiza al servidor}.
\footnote{http://oddballupdate.com/2009/12/fix-slow-webdav-performance-in-windows-7/}
\footnote{https://support.microsoft.com/en-us/kb/2445570/}
\footnote{https://www.iis.net/learn/publish/using-webdav/using-the-webdav-redirector}

        \subsubsection {Conexi\'{o}n al servidor \textsc{LDAP}}

La conexi\'{o}n al servidor LDAP se configura con la directiva \texttt{AuthLDAPUrl} que especifica los par\'{a}metros de conexi\'{o}n, se compone de dos partes principales:

\begin{itemize}
  \item Cadena de conexi\'{o}n
  \begin{itemize}
    \item Protocolo
    \item Host
    \item Puerto
  \end{itemize}
  \item Filtro
  \begin{itemize}
    \item Contenedor
    \item Atributos de \'{i}ndice
    \item Profundidad de la b\'{u}squeda
    \item Condiciones de b\'{u}squeda
  \end{itemize}
\end{itemize}

        \subsubsection {B\'{u}squeda en el directorio}

Esta operaci\'{o}n se realiza estableciendo una conexi\'{o}n al servidor para enviar el comando y los par\'{a}metros de b\'{u}squeda. La respuesta puede ser una lista de los identificadores \'{u}nicos de cada elemento encontrado o un resultado vac\'{i}o.

{
\scriptsize
\linespread{1}
\begin{verbatim}
AuthLDAPUrl "ldapi:///ou=users,dc=xnas,dc=local?uid,cn?sub?
             (|(objectClass=posixAccount)(objectClass=simpleSecurityObject))"
\end{verbatim}
}


Para esta URL se realizar\'{a} una b\'{u}squeda en el directorio en el contenedor principal de usuarios y el objetivo es encontrar objetos de tipo \texttt{posixAccount} o \texttt{simpleSecurityObject} que est\'{e}n indexados por el atributo \textsl{uid} o \textsl{cn}. El alcance de la b\'{u}squeda se establece para todos los objetos que se encuentren bajo esta rama del \'{a}rbol, si se desea que \'{u}nicamente se exploren los objetos en este nivel del directorio, se debe establecer el par\'{a}metro \textsl{one}.

        \subsubsection {Grupos de \textsc{LDAP}}

Para el manejo de permisos de acceso se utilizaron grupos de \textsc{LDAP} que representan a los alumnos que se encuentran inscritos a una materia. Se incluye la directiva \texttt{Require} para establecer en la configuraci\'{o}n que es necesario pertenecer a determinado grupo para tener acceso al recurso solicitado.

{
\scriptsize
\linespread{1}
\begin{verbatim}
Require ldap-group cn=1667-AORJ771312,ou=webdav,ou=groups,dc=xnas,dc=local
\end{verbatim}
}

Al especificar la directiva \texttt{Require ldap-group} se realiza la b\'{u}squeda del usuario y adicionalmente del grupo con el objeto de encontrar todos sus miembros y poder verificar si el usuario en cuesti\'{o}n es uno de ellos.


    \section {Implementaci\'{o}n de las interfaces de usuario}

      \subsection {Acceso mediante el navegador web}

Para acceder mediante el navegador web, simplemente se accede a la URL, de manera predeterminada el servidor pide el usuario y el password del usuario para ofrecer el listado de  archivos y directorios disponible dependiendo de su nivel de acceso.

      \subsection {Acceso mediante cliente nativo de \textsc{WebDAV}}

Para el acceso mediante el cliente nativo se utilizan los siguientes par\'{a}metros:, en el ap\'{e}ndice A se puede consultar el manual de conexi\'{o}n utilizando el cliente nativo

{
 \begin{table}[H]
 \caption{Par\'{a}metros de conexi\'{o}n \texttt{WebDAV}}{}
 \label{tab:webdav-parameters}
 \noindent\makebox[\textwidth]
 {%
  \begin{tabular}[c]{c|c}
  %\hline
  \textbf{Par\'{a}metro} & \textbf{Valor} \\
  \hline \hline
  \textit{Endpoint} & \texttt{https://xnas.local/profesores} \\
  \textit{SSL} & \texttt{true} \\
  \textit{Authentication} & \textsc{basic} \\
  %\hline
  \end{tabular}
 } % ending of \makebox
 \end{table}
}

%        \subsubsection{P\'{a}gina de inicio del sitio web}
%
%Se desarroll\'{o} una p\'{a}gina para facilitar a los usuarios la b\'{u}squeda de los directorios asociados con sus materias

      \subsection {Interfaz de administraci\'{o}n \textit{LDAP Account Manager}}

Se descarga el \textit{tarball} del c\'{o}digo fuente y se descomprime en el directorio \texttt{lam/htdocs}:

{
\scriptsize
\linespread{1}
\begin{verbatim}
    # cd /opt/xNAS/app/lam
    # wget -c 'http://iweb.dl.sourceforge.net/project/lam/LAM/4.8/ldap-account-manager-4.8.tar.bz2'
    # tar -xvvjf ldap-account-manager-4.8.tar.bz2 -C /opt/xNAS/app/lam
    # mv ldap-account-manager-4.8 htdocs
\end{verbatim}
}

Se establecen permisos de escritura para el servidor web en los directorios \texttt{sess} y \texttt{temp}:

{
\scriptsize
\linespread{1}
\begin{verbatim}
    # chown -cR www-data:www-data /opt/xNAS/app/lam/htdocs/{sess,tmp}
\end{verbatim}
}

Se instalan los archivos de configuraci\'{o}n de la herramienta:

{
\scriptsize
\linespread{1}
\begin{verbatim}
    # cd /opt/xNAS/app/lam/htdocs/config
    # ln -vsf ../../lam.conf
    # ln -vsf ../../config.cfg
\end{verbatim}
}

Se instala el certificado, la llave privada y la configuraci\'{o}n del sitio, una vez realizado esto, se reinicia el servicio:

{
\scriptsize
\linespread{1}
\begin{verbatim}
    # cd /etc/ssl/private
    # ln -vsf ../../../opt/xNAS/ssl/admin.xnas.local.key
    # cd /etc/ssl/certs
    # ln -vsf ../../../opt/xNAS/ssl/admin.xnas.local.crt
    # c_rehash

    # cd /etc/apache2/sites-available
    # ln -vsf ../../../opt/xNAS/apache2/admin.xnas.local
    # a2ensite admin.xnas.local
    # /etc/init.d/apache2 reload
\end{verbatim}
}

      \subsection {Interfaz de cambio de contrase\~{n}a}

Se descarga el paquete de instalaci\'{o}n del sitio web oficial y se extrae en el directorio \texttt{ltb/htdocs}:

{
\scriptsize
\linespread{1}
\begin{verbatim}
    # cd /opt/xNAS/app/ltb
    # wget -c 'http://tools.ltb-project.org/attachments/download/497/\
      ltb-project-self-service-password-0.8.tar.gz'
    # tar -xvvzf ltb-project-self-service-password-0.8.tar.gz -C /opt/xNAS/app/ltb
    # mv ltb-project-self-service-password-0.8 htdocs
\end{verbatim}
}

Se instala el certificado, la llave privada y la configuraci\'{o}n del \texttt{VirtualHost}, hecho esto se reinicia el servicio para reflejar los cambios.

{
\scriptsize
\linespread{1}
\begin{verbatim}
    # cd /etc/ssl/private
    # ln -vsf ../../../opt/xNAS/ssl/reset.xnas.local.key
    # cd /etc/ssl/certs
    # ln -vsf ../../../opt/xNAS/ssl/reset.xnas.local.crt
    # c_rehash

    # cd /etc/apache2/sites-available
    # ln -vsf ../../../opt/xNAS/apache2/reset.xnas.local
    # a2ensite reset.xnas.local
    # /etc/init.d/apache2 reload
\end{verbatim}
}

    \section {Hardening}

      \subsection {Actualizaciones desatendidas}

Una parte importante de la configuraci\'{o}n de seguridad de un sistema radica en la instalaci\'{o}n peri\'{o}dica de actualizaciones, al ejecutar el siguiente comando y responder \textbf{YES} en el cuadro de di\'{a}logo, se configura el sistema para descargar e instalar \emph{autom\'{a}ticamente} las actualizaciones de seguridad que sean liberadas por el fabricante del sistema operativo.

{
\scriptsize
\linespread{1}
\begin{verbatim}
    # dpkg-reconfigure unattended-upgrades

+------------------------+ Configuring unattended-upgrades +------------------------+
| Applying updates on a frequent basis is an important part of keeping systems      |
| secure. By default, updates need to be applied manually using package management  |
| tools. Alternatively, you can choose to have this system automatically download   |
| and install security updates.                                                     |
|                                                                                   |
| Automatically download and install stable updates?                                |
|                       <Yes>                          <No>                         |
+-----------------------------------------------------------------------------------+
\end{verbatim}
}

      \subsection {Reducci\'{o}n de componentes instalados}

Como parte de la configuraci\'{o}n de seguridad, es recomendable minimizar los paquetes que se instalan en el sistema operativo para disminuir la ventana de posibilidades de tener una intrusi\'{o}n. Con la siguiente configuraci\'{o}n del gestor de paquetes se evita la instalaci\'{o}n de software adicional cuando se agrega un paquete, esto ayuda a reducir la complejidad del sistema y reduce el espacio en disco utilizado. \footnote{http://aptitude.alioth.debian.org/doc/en/ch02s05s05.html}

{
\scriptsize
\linespread{1}
\begin{verbatim}
# cat > /etc/apt/apt.conf.d/10Hardening << EOF
Apt::Install-Recommends "false";
Apt::Install-Suggests "false";
EOF

\end{verbatim}
}

      \subsection {Evitar el apagado o reinicio accidental del equipo}

Al administrar un equipo de manera remota, se corre el riesgo de ejecutar por error el comando de apagado o reinicio. Al suceder esto, se pierde la sesi\'{o}n de todos los usuarios, los servicios se interrumpen y si se dio la \'{o}rden de apagado, ser\'{a} necesario acudir a la ubicaci\'{o}n f\'{i}sica del equipo y prenderlo manualmente. Para mitigar el riesgo se instal\'{o} el programa \texttt{molly-guard}.

Al detectar el uso de comandos para apagar o reiniciar el sistema desde una sesi\'{o}n remota, \texttt{molly-guard} preguntar\'{a} el nombre del equipo para hacer una verificaci\'{o}n adicional y evitar que un equipo remoto sea apagado o reiniciado accidentalmente de manera remota.

{
\scriptsize
\linespread{1}
\begin{verbatim}
    # aptitude install molly-guard
\end{verbatim}
}

      \subsection {Reenvio del correo electr\'{o}nico de \texttt{root}}

Se instal\'{o} en el equipo el \textsl{MTA} \texttt{Postfix} configurado como \textit{Internet Site}. Para habilitar la redirecci\'{o}n de los mensajes de correo electr\'{o}nico dirigidos al administrador del sistema se realiz\'{o} lo siguiente:

\begin{itemize}
  \item Especificar la direccion de destino en el archivo \texttt{/etc/aliases}:

{
\scriptsize
\linespread{1}
\begin{verbatim}
    root:	xnas@tonejito.org
\end{verbatim}
}

  \item Regenerar la base de datos de \textit{aliases}:

{
\scriptsize
\linespread{1}
\begin{verbatim}
    # newaliases
\end{verbatim}
}

  \item Recargar la configuraicon del servicio de correo:

{
\scriptsize
\linespread{1}
\begin{verbatim}
    # /etc/init.d/postfix reload
\end{verbatim}
}

\end{itemize}

      \subsection {Restricci\'{o}n de acceso para las tareas programadas}

Para limitar la capacidad de los usuarios para programar la ejecuci\'{o}n de comandos en el sistema operativo se adopt\'{o} una politica restrictiva que consiste en tener una \textit{lista blanca} donde se inclyen los nombres de las cuentas que pueden hacer uso de trabajos de \textsl{cron} y \textsl{at}.

{
\scriptsize
\linespread{1}
\begin{verbatim}
    # cd /etc
    # getent passwd | sort -g -t ":" -k 3,3 | awk -F: '$3<1000 {print $1}' | tee cron.allow
    # ln -v cron.allow at.allow
\end{verbatim}
}
\footnote{https://www.debian.org/doc/manuals/system-administrator/ch-sysadmin-users.html\#s8.1.1}
\footnote{http://manpages.debian.org/cgi-bin/man.cgi?query=crontab\&manpath=Debian+7.0+wheezy}
\footnote{http://linux.die.net/man/1/at}

      \subsection {Configuraci\'{o}n de seguridad de \textsl{OpenSSH}}

Para la configuraci\'{o}n del servicio de \textsc{SSH} se aplicaron las siguientes opciones en el archivo de configuraci\'{o}n \texttt{/etc/ssh/sshd\_config}

\begin{itemize}
  \item Restringir el acceso administrativo directo \'{u}nicamente mediante autenticaci\'{o}n de llave p\'{u}blica.
    \code{PermitRootLogin without-password}
  \item Deshabilitar el acceso a cuentas que tengan una contrase\~{n}a vac\'{i}a.
    \code{PermitEmptyPasswords no}
  \item Permitir el acceso al grupo de usuarios de la \textit{Unidad de C\'{o}mputo} de la \textsc{DICyG}.
    \code{AllowGroups adm staff support}
  \item  Mostrar una pantalla antes de iniciar la sesi\'{o}n en la que se listen las pol\'{i}ticas de uso del sistema.
    \code{Banner /etc/issue.net}
  \item Evitar la desconexi\'{o}n manteniendo actividad en la sesi\'{o}n.
    \code{TCPKeepAlive yes}
    \code{ClientAliveInterval 30}
  \item Evitar la redirecci\'{o}n de puertos y sesiones gr\'{a}ficas.
    \code{AllowTCPForwarding no}
    \code{GatewayPorts no}
    \code{X11Forwarding no}
  \item Desconectar al usuario si no inicia sesi\'{o}n en 60 segundos.
    \code{LoginGraceTime 60}
\end{itemize}

      \subsection {Reglas de Firewall}
      \label{subsec:fw-rules}

Para filtrar el tr\'{a}fico de red se establecieron las siguientes pol\'{i}ticas de \textit{firewall} de host en el equipo:

\begin{itemize}
  \item Se permite todo el tr\'{a}fico proveniente de la interfaz \textit{loopback}.
  \item Las conexiones a trav\'{e}s de \textsc{SSH} se permiten \'{u}nicamente desde los siguientes segmentos confiables de red:
  \begin{enumerate}
    \item Segmentos de \textsc{RedUNAM}: \texttt{132.247.0.0/16} , \texttt{132.248.0.0/16}
    \item Red interna de la \textsc{DICyG}.
  \end{enumerate}
  \item Se permiten las conexiones por \textsc{HTTPS} desde cualquier host, ya sea de la red interna, segmentos de \textsc{RedUNAM} o Internet.
  \item Las conexiones al demonio de \textsc{LDAP} se permiten \'{u}nicamente desde \textit{localhost}.
\end{itemize}

\diagramblock
{Trusted Networks}
{TrustedNetworks}
{
 \psscalebox{1.0 1.0} % Change this value to rescale the drawing.
 {
    \begin{pspicture}(0,-1.84)(3.68,1.84)
  \rput[bl](1.225,1.48){Internet}
  \rput[bl](0.89,0.46){\textsc{RedUNAM}}
  \rput[bl](1.23,-0.76){\textsc{DICyG}}
  \pscircle[linecolor=black, linewidth=0.04, dimen=outer](1.84,-0.58){0.78}
  \pscircle[linecolor=black, linewidth=0.04, dimen=outer](1.84,-0.19){1.45}
  \psframe[linecolor=black, linewidth=0.04, dimen=outer](3.68,1.84)(0.0,-1.84)
  \end{pspicture}

 }
}

Para implementar las reglas de firewall se utiliz\'{o} \textit{iptables}, el software est\'{a}ndar para filtrado de paquetes en \textsc{GNU/Linux} y el paquete \textit{iptables-persistent} que sirve para aplicar las reglas guardadas en cada inicio del sistema operativo. El programa se instala con \textit{aptitude} y al configurarlo pregunta si se guardar\'{a}n las reglas de \textsc{IPv4} e \textsc{IPv6}.

{
\scriptsize
\linespread{1}
\begin{verbatim}
    # aptitude install iptables-persistent

+--------------------| Configuring iptables-persistent |--------------------+
|                                                                           |
| Current iptables rules can be saved to the configuration file             |
| /etc/iptables/rules.v4. These rules will then be loaded automatically     |
| during system startup.                                                    |
|                                                                           |
| Rules are only saved automatically during package installation. See the   |
| manual page of iptables-save(8) for instructions on keeping the rules     |
| file up-to-date.                                                          |
|                                                                           |
| Save current IPv4 rules?                                                  |
|                                                                           |
|                    <Yes>                       <No>                       |
|                                                                           |
+---------------------------------------------------------------------------+
\end{verbatim}
}

        \subsubsection {Protecci\'{o}n contra ataques de fuerza bruta con \texttt{fail2ban}}

Este software realiza un seguimiento de las conexiones que se realizan al equipo e identifica los ataques de fuerza bruta contra el servicio de \textsc{SSH}. Al detectar un ataque, realiza un bloqueo de la direcci\'{o}n \textsc{IP} de origen por un tiempo determinado.

Es necesario instalar el software y hacer una copia del archivo \texttt{jail.conf} con el nombre \texttt{jail.local} donde se realizar\'{a}n ajustes a la configuraci\'{o}n del servicio.

{
\scriptsize
\linespread{1}
\begin{verbatim}
# aptitude install fail2ban
# cp /etc/fail2ban/jail.conf /etc/fail2ban/jail.local
\end{verbatim}
}

Una vez copiado el archivo de configuraci\'{o}n, se agregan los bloques \textsc{CIDR} definidos en el \textsc{RFC1918} y los segmentos de \textsc{RedUNAM} que se definieron en la secci\'{o}n \ref{subsec:fw-rules} en la p\'{a}gina \pageref{subsec:fw-rules}. Adicionalmente se incrementa el tiempo de bloqueo de la direcci\'{o}n IP del atacante de 600 segundos a 3600 (una hora).

{
\scriptsize
\linespread{1}
\begin{verbatim}
[DEFAULT]
# "ignoreip" can be an IP address, a CIDR mask or a DNS host
ignoreip = 127.0.0.0/8 10.0.0.0/8 172.16.0.0/12 192.168.0.0/16 132.247.0.0/16 132.248.0.0/16
# seconds
bantime  = 3600
maxretry = 3
\end{verbatim}
}

Al finalizar el ajuste de los par\'{a}metros de configuraci\'{o}n, se reinicia el servicio.

{
\scriptsize
\linespread{1}
\begin{verbatim}
# service fail2ban restart
\end{verbatim}
}

      \subsection {Configuraci\'{o}n de seguridad de Apache HTTPD}

Se establecieron los valores recomendados en las siguientes directivas, con el objetivo de evitar la divulgaci\'{o}n de informaci\'{o}n acerca del software del servidor web y el sistema operativo. Estas directivas se encuentran presentes en el archivo \texttt{/etc/apache2/conf.d/security}.

{
 \begin{table}[H]
 \caption{Directivas de seguridad de Apache \textsc{HTTPD}}{}
 \label{tab:apache-hardening}
 \noindent\makebox[\textwidth]
 {%
  \begin{tabular}[c]{r|c}
  %\hlinoo
  \multicolumn{1}{c}{\textbf{Directiva}} & \multicolumn{1}{c}{\textbf{Valor}} \\
  \hline \hline
  \texttt{ServerSignature} & Off         \\
  \texttt{ServerTokens}    & ProductOnly \\
  \texttt{TraceEnable}     & Off         \\
  %\hline
  \end{tabular}
 } % ending of \makebox
 \end{table}
}

        \subsubsection{Configuraci\'{o}n para \textsc{PHP}}

Se configur\'{o} \textsc{PHP} para que no muestre la versi\'{o}n en las cabeceras del protocolo \textsc{HTTP}, evitar la impresi\'{o}n de errores y para que guarde los errores encontrados en una bit\'{a}cora. Para lograr esto, se establecieron los siguientes valores en el archivo \texttt{php.ini}.

{
 \begin{table}[H]
 \caption{Directivas de seguridad de \textsc{PHP}}{}
% \label{tab:php-hardening}
 \noindent\makebox[\textwidth]
 {%
  \begin{tabular}[c]{r|c}
  %\hline
  \multicolumn{1}{c}{\textbf{Directiva}} & \multicolumn{1}{c}{\textbf{Valor}} \\
  \hline \hline
  \texttt{expose\_php}              & Off \\
  \texttt{display\_errors}          & Off \\
  \texttt{display\_startup\_errors} & Off \\
  \texttt{log\_errors}              & On  \\
  \texttt{error\_log}               & /var/log/apache2/error.log  \\
  %\hline
  \end{tabular}
 } % ending of \makebox
 \end{table}
}

        \subsubsection{Deshabilitar el soporte de archivos \texttt{.htaccess}}

Se agreg\'{o} esta directiva de configuraci\'{o}n dentro de un bloque \texttt{<Directory>} para que el servidor ignore los archivos de configuraci\'{o}n \texttt{.htaccess} debido a que podr\'{i}an ser modificados por los usuarios.

{
\scriptsize
\linespread{1}
\begin{verbatim}
AllowOverride none
\end{verbatim}
}

        \subsubsection{Restricci\'{o}n de permisos de escritura}

Los m\'{e}todos \textsc{HTTP} se pueden clasificar de acuerdo al tipo de acceso que tienen, ya sea para obtener o enviar datos al servidor. En la siguiente tabla se muestra la clasificaci\'{o}n de acuerdo a este criterio:

{
 \linespread{1}
 \begin{table}[H]
 \caption{Clasificaci\'{o}n de m\'{e}todos \textsc{HTTP} de acuerdo al tipo de acceso}{}
 \label{tab:http-methods}
 \noindent\makebox[\textwidth]
 {%
  \begin{tabular}[c]{c|c c c}
  %\hline
  \textbf{Tipo} & & \textbf{M\'{e}todos} & \\
  \hline \hline
  \multirow{2}{*}{\textsl{S\'{o}lo lectura}} & \texttt{OPTIONS} & \texttt{PROPFIND} & \\ 
                                             & \texttt{HEAD}    & \texttt{GET}      & \\
  \hline
  \multirow{4}{*}{\textsl{Lectura y Escritura}} & \texttt{POST}      & \texttt{PUT}    & \texttt{MKCOL}  \\
                                                & \texttt{COPY}      & \texttt{MOVE}   & \texttt{DELETE} \\
                                                & \texttt{LOCK}      & \texttt{UNLOCK} & \texttt{PATCH}  \\
                                                & \texttt{PROPPATCH} &                 & \\
  %\hline
  \end{tabular}
 } % ending of \makebox
 \end{table}
}

Para establecer una restricci\'{o}n a los m\'{e}todos \textsc{HTTP} que pueden escribir datos en el servidor se utiliz\'{o} la directiva \texttt{LimitExcept} para definir los m\'{e}todos que se permiten en el servidor en determinado directorio \footnote{http://httpd.apache.org/docs/2.2/mod/core.html\#limit}. En el siguiente ejemplo se muestra la configuraci\'{o}n adoptada para permitir el acceso de solo lectura.

{
\scriptsize
\linespread{1}
\begin{verbatim}
<LimitExcept GET OPTIONS PROPFIND>
  Satisfy ALL
  Require ldap-group _
  Order Allow,Deny
  Deny From ALL
</LimitExcept>
\end{verbatim}
}

