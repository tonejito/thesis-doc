  \label{chap:cap3}

  \chapter {Implementaci\'{o}n de la soluci\'{o}n}

    \section {Configuraci\'{o}n del sistema operativo}

El sistema operativo que ser\'{a} utilizado para la implementaci\'{o}n es \textsc{Debian GNU/Linux} 7.0 \textit{\guillemotleft wheezy\guillemotright} que se instalar\'{a} de manera nativa en el servidor y se le aplicar\'{a}n configuraciones personalizadas para ejecutar el software que ser\'{a} implementado, ajustar el rendimiento y reforzar la seguridad de los servicios que ofrezca.

      \subsection {Arreglo de discos RAID}

Los discos duros del servidor ser\'{a}n configurados en un arreglo RAID debido a las ventajas que propone para el servicio\footnote{Ver p\'{a}gina \pageref{Arreglos-RAID} secci\'{o}n \ref{Arreglos-RAID}}. La configuraci\'{o}n en el servidor de pruebas se realizar\'{a} con un arreglo RAID 1 por \textit{software} que se configur\'{o} de la siguiente manera: \footnote{https://wiki.debian.org/SoftwareRAID}

\begin{itemize}
  \item Instalar el software para habilitar la compatibilidad con arreglos \textsc{RAID} por software:
    \code{\# apt-get install mdadm}
  \item Crear las particiones de tipo \textit{Linux RAID Autodetect} en cada disco del arreglo
    \begin{itemize}
      \item Crear una partici\'{o}n primaria que ocupe todo el espacio del disco:

{
\scriptsize
\linespread{1}
\begin{verbatim}
    # /sbin/fdisk /dev/sdb

    Device contains neither a valid DOS partition table, nor Sun, SGI or OSF disklabel
    Building a new DOS disklabel with disk identifier 0x00bab10c.
    Changes will remain in memory only, until you decide to write them.
    After that, of course, the previous content won't be recoverable.

    Warning: invalid flag 0x0000 of partition table 4 will be corrected by w(rite)

    Command (m for help): n
    Partition type:
       p   primary (0 primary, 0 extended, 4 free)
       e   extended
    Select (default p): p
    Partition number (1-4, default 1): 1
    First sector (2048-16777215, default 2048):
    Using default value 2048
    Last sector, +sectors or +size{K,M,G} (2048-16777215, default 16777215):
    Using default value 16777215
\end{verbatim}
}

      \item Cambiar el tipo de la partici\'{o}n a \texttt{0xfd} - \textit{Linux RAID Autodetect} y escribir la tabla de particiones al disco:

{
\scriptsize
\linespread{1}
\begin{verbatim}
    Command (m for help): t
    Selected partition 1
    Hex code (type L to list codes): fd
    Changed system type of partition 1 to fd (Linux RAID Autodetect)

    Command (m for help): w
    The partition table has been altered!

    Calling ioctl() to re-read partition table.
    Syncing disks.
\end{verbatim}
}

    \end{itemize}
  \item Crear arreglo:
    \code{\# mdadm --zero-superblock /dev/sdb1 /dev/sdc1}
    \code{\# mdadm --create /dev/md0 --level=1 --raid-devices=2 /dev/sdb1 /dev/sdc1}
  \item Formatear el disco que representa el arreglo:
    \code{\# mkfs.ext4 /dev/md0}
  \item Asignar tipo y punto de montaje en el archivo \texttt{/etc/fstab}:
    \code{/dev/md0 /opt/xNAS/files ext4 noatime,rw 0 0}
  \item Guardar configuraci\'{o}n en el archivo \texttt{/etc/mdadm/mdadm.conf}:
    \code{ARRAY /dev/md0 devices=/dev/sdb1,/dev/sdc1 level=1 num-devices=2 auto=yes}
    \code{DEVICE /dev/sdb1 /dev/sdc1}
\end{itemize}

    \section {Configuraci\'{o}n de los servicios}

      \subsection {OpenLDAP}

          \subsubsection {Instalaci\'{o}n de \textsc{OpenLDAP}}

Se instala \textsc{OpenLDAP} utilizando el comando \texttt{aptitude}, es necesario contestar \textbf{NO} en el cuadro de di\'{a}logo para seguir el asistente de instalaci\'{o}n:

{
\scriptsize
\linespread{1}
\begin{verbatim}
    # aptitude install slapd
\end{verbatim}
}


{
\scriptsize
\linespread{1}
\begin{verbatim}
+---------------------------+ Configuring slapd +----------------------------+
| If you enable this option, no initial configuration or database will be    |
| created for you.                                                           |
|                                                                            |
| Omit OpenLDAP server configuration?                                        |
|                    <Yes>                       <No>                        |
+----------------------------------------------------------------------------+
\end{verbatim}
}

Se pide el nombre DNS del dominio que servir\'{a} para conformar la ra\'{i}z del \'{a}rbol de \textsc{LDAP}.

{
\scriptsize
\linespread{1}
\begin{verbatim}
+---------------------------+ Configuring slapd +----------------------------+
| The DNS domain name is used to construct the base DN of the LDAP           |
| directory. For example, 'foo.example.org' will create the directory with   |
| 'dc=foo, dc=example, dc=org' as base DN.                                   |
|                                                                            |
| DNS domain name:                                                           |
| xnas.local________________________________________________________________ |
|                                   <Ok>                                     |
+----------------------------------------------------------------------------+
\end{verbatim}
}

Adicionalmente el instalador pregunta por el nombre de la organizaci\'{o}n, aunque este dato no es necesario, se recomienda introducirlo:

{
\scriptsize
\linespread{1}
\begin{verbatim}
+---------------------------+ Configuring slapd +----------------------------+
| Please enter the name of the organization to use in the base DN of your    |
| LDAP directory.                                                            |
|                                                                            |
| Organization name:                                                         |
| xNAS_____________________________________________________________________  |
|                                  <Ok>                                      |
+----------------------------------------------------------------------------+
\end{verbatim}
}

A continuaci\'{o}n se pide que se introduzca la contrase\~{n}a que ser\'{a} utilizada por el administrador del directorio:

{
\scriptsize
\linespread{1}
\begin{verbatim}
+-------------------------+ Configuring slapd +------------------------------+
| Please enter the password for the admin entry in your LDAP directory.      |
|                                                                            |
| Administrator password:                                                    |
| **********____________________________________________________________     |
|                                 <Ok>                                       |
+----------------------------------------------------------------------------+

+---------------------------+ Configuring slapd +----------------------------+
| Please enter the admin password for your LDAP directory again to verify    |
| that you have typed it correctly.                                          |
|                                                                            |
| Confirm password:                                                          |
| **********_______________________________________________________________  |
|                                  <Ok>                                      |
+----------------------------------------------------------------------------+
\end{verbatim}
}

Se recomienda deshabilitar el soporte del protocolo \textsc{LDAPv2} puesto que es obsoleta, responder \textbf{NO} en el cuadro de di\'{a}logo.

{
\scriptsize
\linespread{1}
\begin{verbatim}
+----------------------------+ Configuring slapd +---------------------------+
|                                                                            |
| The obsolete LDAPv2 protocol is disabled by default in slapd. Programs and |
| users should upgrade to LDAPv3.  If you have old programs which can't use  |
| LDAPv3, you should select this option and 'allow bind_v2' will be added to |
| your slapd.conf file.                                                      |
|                                                                            |
| Allow LDAPv2 protocol?                                                     |
|                     <Yes>                        <No>                      |
+----------------------------------------------------------------------------+
\end{verbatim}
}

          \subsubsection {Configuraci\'{o}n de \textsc{OpenLDAP}}

Se configur\'{o} el sistema operativo para permitir que los usuarios de \textsc{LDAP} de tipo \textit{posixAccount} puedan iniciar sesi\'{o}n en el equipo.

{
\scriptsize
\linespread{1}
\begin{verbatim}
  # aptitude install libpam-ldapd
\end{verbatim}
}

Adicionalmente se instalaron los demonios \textsc{NSCD} y \textsc{NSLCD} que se utilizan para guardar en \textit{cache} los resultados de las consultas al directorio y para representar los objetos \textit{posixAccount} de \textsc{LDAP} como cuentas est\'{a}ndar de \textsc{UNIX}.

%\ifdraft{Configuraci\'{o}n de \textsc{OpenLDAP}.}{\verbatimincludebox{./etc/ldap/ldap.conf}}

{
\scriptsize
\linespread{1}
\begin{verbatim}
+------------------------+ Configuring libnss-ldapd +-------------------------+
| For this package to work, you need to modify your /etc/nsswitch.conf to     |
| use the ldap datasource.                                                    |
|                                                                             |
| You can select the services that should have LDAP lookups enabled. The new  |
| LDAP lookups will be added as the last datasource. Be sure to review these  |
| changes.                                                                    |
|                                                                             |
| Name services to configure:                                                 |
|    [*] group                                                                |
|    [*] passwd                                                               |
|    [*] shadow                                                               |
|                                   <Ok>                                      |
+-----------------------------------------------------------------------------+
\end{verbatim}
}

Para la configuraci\'{o}n de \textsc{PAM}, se indica que se utilizar\'{a}n como fuentes de autenticaci\'{o}n la base de datos est\'{a}ndar de UNIX y adicionalmente el directorio LDAP.

{
\scriptsize
\linespread{1}
\begin{verbatim}
+----------------------------+ PAM configuration +----------------------------+
| Pluggable Authentication Modules (PAM) determine how authentication,        |
| authorization, and password changing are handled on the system, as well as  |
| allowing configuration of additional actions to take when starting user     |
| sessions.                                                                   |
|                                                                             |
| Some PAM module packages provide profiles that can be used to               |
| automatically adjust the behavior of all PAM-using applications on the      |
| system.  Please indicate which of these behaviors you wish to enable.       |
|                                                                             |
| PAM profiles to enable:                                                     |
|    [*] Unix authentication                                                  |
|    [*] LDAP Authentication                                                  |
|                    <Ok>                        <Cancel>                     |
+-----------------------------------------------------------------------------+
\end{verbatim}
}

          \subsubsection {Inicializaci\'{o}n del directorio \textsc{LDAP}}

Una vez instalado el servicio de directorio, es necesario inicializar la estructura b\'{a}sica que comprende los contenedores de usuarios, materias y grupos utilizandoel archivo \texttt{xNAS-base.ldif} y el comando \texttt{ldapadd}.

% http://techiezone.rottigni.net/2011/12/change-root-dn-password-on-openldap/
% http://www.linuxtopia.org/online_books//network_administration_guides/ldap_administration/slapdconf2_Configuration_Directives.html

{
\scriptsize
\linespread{1}
\begin{verbatim}
Se instala el software necesario
  # aptitude install ldap-utils
Se verifica que no existan entradas, la salida del comando \texttt{ldapsearch} debe ser similar a la que se muestra
  # ldapsearch -LLL -Y EXTERNAL -H ldapi:/// -b 'dc=xnas,dc=local'
SASL/EXTERNAL authentication started
SASL username: gidNumber=0+uidNumber=0,cn=peercred,cn=external,cn=auth
SASL SSF: 0
dn: dc=xnas,dc=local
objectClass: top
objectClass: dcObject
objectClass: organization
o: xNAS
dc: xnas

dn: cn=admin,dc=xnas,dc=local
objectClass: simpleSecurityObject
objectClass: organizationalRole
cn: admin
description: LDAP administrator

Se procede con la carga inicial
  # ldapadd -x -W -D "cn=admin,dc=xnas,dc=local" -H 'ldapi:///' -f ./xNAS-base.ldif
Enter LDAP Password: 
adding new entry "ou=services,dc=xnas,dc=local"
adding new entry "ou=materias,dc=xnas,dc=local"
adding new entry "ou=users,dc=xnas,dc=local"
adding new entry "ou=staff,ou=users,dc=xnas,dc=local"
adding new entry "ou=profesores,ou=users,dc=xnas,dc=local"
adding new entry "ou=grupos,ou=users,dc=xnas,dc=local"
adding new entry "ou=alumnos,ou=users,dc=xnas,dc=local"
adding new entry "ou=groups,dc=xnas,dc=local"
adding new entry "ou=unix,ou=groups,dc=xnas,dc=local"
adding new entry "ou=webdav,ou=groups,dc=xnas,dc=local"

\end{verbatim}
}

          \subsubsection {Carga de datos en el directorio \textsc{LDAP}}

Para realizar la carga de la base de datos de usuarios y grupos se desarroll\'{o} una biblioteca y un script de carga en el lenguaje de programaci\'{o}n \textsl{Ruby} que lee los datos desde un archivo origen en formato \texttt{CSV}, establece las relaciones entre los objetos y realiza el ingreso de los datos al directorio.

La biblioteca que realiza la carga de objetos al directorio funciona de acuerdo al siguiente algoritmo:

\begin{itemize}
  \item Realiza una conexi\'{o}n al directorio LDAP, a esta operaci\'{o}n se le denomina \textit{bind}.
  \item Convierte el archivo de entrada a la codificaci\'{o}n \textsc{UTF-8}.
  \item Lee cada rengl\'{o}n del archivo de entrada y verifica el contenido de cada campo contra una \textit{expresi\'{o}n regular} para identificar problemas.
  \item Si el rengl\'{o}n cumple con el formato, se contin\'{u}a con el pr\'{o}ximo paso, de lo contrario se guarda en una lista que debe ser revisada manualmente y salta al siguiente.
  \item Asigna los atributos de cada objeto LDAP de acuerdo al valor de cada campo.
  \item Establece las relaciones necesarias con otros objetos y realiza una transacci\'{o}n para insertar los datos en el directorio.
  \item Verifica que la inserci\'{o}n se haya realizado correctamente, en caso de existir alg\'{u}n error, se realiza un \textit{rollback} de las operaciones.
\end{itemize}

\diagramblock
{Diagrama de bloques de los scripts de carga}
{DiagramaScript}
{
  \psscalebox{1.0 1.0} % Change this value to rescale the drawing.
  {
    \begin{pspicture}(0,-1.9053571)(11.5,1.9053571)
    \rput[bl](0.33,1.1353571){\textbf{CSV}}
    \rput[bl](4.07,1.0803571){./script.rb}
    \rput[bl](3.295,-1.1396428){\textsl{Objetos Conflictivos}}
    \rput[bl](3.345,-1.9053571){Verificaci\'{o}n manual}
    \rput[bl](8.675,1.1353571){Directorio LDAP}
    \psframe[linecolor=black, linewidth=0.04, dimen=outer](6.1,1.8953571)(3.6,0.6153571)
    \psframe[linecolor=black, linewidth=0.04, dimen=outer](1.48,1.8953571)(0.0,0.6153571)
    \psframe[linecolor=black, linewidth=0.04, dimen=outer](6.6,-0.51964283)(3.1,-1.4996428)
    \psframe[linecolor=black, linewidth=0.04, dimen=outer](11.5,1.9053571)(8.48,0.6053571)
    \psline[linecolor=black, linewidth=0.04, arrowsize=0.05291666666666667cm 2.0,arrowlength=1.4,arrowinset=0.0]{->}(1.5,1.2644575)(3.592381,1.2454098)
    \psline[linecolor=black, linewidth=0.04, arrowsize=0.05291666666666667cm 2.0,arrowlength=1.4,arrowinset=0.0]{->}(6.087619,1.2553571)(8.506667,1.2553571)
    \psline[linecolor=black, linewidth=0.04, arrowsize=0.05291666666666667cm 2.0,arrowlength=1.4,arrowinset=0.0]{->}(6.0971427,0.84988093)(7.0590477,0.84988093)(7.04,-1.0834523)(6.582857,-1.0929762)
    \psline[linecolor=black, linewidth=0.04, arrowsize=0.05291666666666667cm 2.0,arrowlength=1.4,arrowinset=0.0]{->}(3.1161904,-1.1691667)(2.44,-1.1691667)(2.44,0.8213095)(3.6209524,0.8117857)
    \end{pspicture}
  }
}

Cada archivo de entrada \texttt{CSV} tiene varias columnas que contienen informaci\'{o}n acerca del tipo de objeto que representan, a continuaci\'{o}n se muestran las columnas que se requieren en los archivos de entrada:

{
\begin{table}[H]
\caption{Formato de los archivos \textsl{CSV}}{}
\label{tab:csv-format}
\noindent\makebox[\textwidth]{%
\begin{tabular}[c]{c|c c c c c}
%\hline
\textbf{Archivo} & \multicolumn{5}{c}{\textbf{Columnas}} \\
\hline \hline
\textit{staff.csv} & usuario & nombre & correo & curp & \\
\textit{materias.csv} & id & grupo & materia & rfc & profesor \\
\textit{profesores.csv} & rfc & nombre & mail & & \\
\textit{alumnos.csv} & cuenta & nombre & correo & asignatura & grupo \\
%\hline
\end{tabular}
} % ending of \makebox
\end{table}
}

Para la correcta ejecuci\'{o}n de los scripts es necesario instalar los paquetes \texttt{ruby} y \texttt{rubygems} y la gema de ruby que realiza la conexi\'{o}n con el directorio \textsc{LDAP}.

{
\scriptsize
\linespread{1}
\begin{verbatim}
  # aptitude install ruby rubygems ruby-json
  # gem install net-ldap
\end{verbatim}
}

Utilizando el script \texttt{./load.rb} se inicializa cada contenedor del directorio y se crean diversos objetos de acuerdo al:

{
\begin{table}[H]
\caption{Script de carga de objetos en el directorio}{}
\label{tab:load-rb}
\noindent\makebox[\textwidth]{%
\begin{tabular}[c]{c|c c}
%\hline
\textbf{Tipo} & \multicolumn{2}{c}{\textbf{Objetos creados en el directorio}} \\
\hline \hline
\textit{staff} & \textit{posixGroup} & \textit{posixAccount} \\
\textit{profesores} & \textit{posixGroup} & \textit{posixAccount} \\
\textit{materias} & \textit{organizationalRole} & \textit{groupOfNames} \\
\textit{alumnos} & \multicolumn{2}{c}{\textit{simpleSecurityObject}} \\
%\hline
\end{tabular}
} % ending of \makebox
\end{table}
}

          \subsubsection {Borrado de datos en el directorio \textsc{LDAP}}

Se desarroll\'{o} un script que ayuda a limpiar el directorio cuando se pretenda carga una nueva base de datos en el mismo.

{
\scriptsize
\linespread{1}
\begin{verbatim}
  # make clean
  # ./clean.rb
\end{verbatim}
}

      \subsection {Apache httpd}

        \subsubsection {Esquema de configuraci\'{o}n}

La configuraci\'{o}n de Apache \textit{httpd} comprende tres \texttt{VirtualHost} que sirven como puntos de entrada para el acceso de s\'{o}lo lectura o lectura y escritura a los archivos almacenados en el servidor.

{
\begin{table}[H]
\caption{VirtualHost configurados en Apache HTTPD}{}
\label{tab:virtualhost}
\noindent\makebox[\textwidth]{%
\begin{tabular}[c]{c|c}
%\hline
\textbf{VirtualHost} & \textbf{Funci\'{o}n} \\
\hline \hline
\textit{xnas.local} & Acceso a los archivos por medio de \textsc{WebDAV} \\
\textit{reset.xnas.local} & Interfaz de cambio de contrase\~{n}a \\
\textit{admin.xnas.local} & Interfaz administrativa del directorio \textsc{LDAP} \\
%\hline
\end{tabular}
} % ending of \makebox
\end{table}
}

\diagramblock
{Diagrama Apache HTTPD VirtualHost}
{DiagramaVirtualHost}
{
 \psscalebox{1.0 1.0} % Change this value to rescale the drawing.
 {
  \begin{pspicture}(0,-2.2)(15.1,2.2)
  \rput[bl](1.2,1.0){xnas.local}
  \rput[bl](0.22,-0.2){reset.xnas.local}
  \rput[bl](0.0,-1.4){admin.xnas.local}
  \rput[bl](4.3,-0.2){443/tcp}
  \rput[bl](6.67,0.2){Apache}
  \rput[bl](6.72,-0.6){HTTPD}
  \rput[bl](9.5,1.0){WebDAV}
  \rput[bl](12.7,1.0){Archivos}
  \rput[bl](12.7,0.6){Carpetas}
  \rput[bl](9.79,-1.0){PHP}
  \rput[bl](12.7,-0.6){ltb (reset)}
  \rput[bl](12.7,-1.4){lam (admin)}
  \psframe[linecolor=black, linewidth=0.04, dimen=outer](15.1,2.2)(5.9,-2.2)
  \psframe[linecolor=black, linewidth=0.04, dimen=outer](8.3,1.8)(6.3,-1.8)
  \psframe[linecolor=black, linewidth=0.04, dimen=outer](11.5,0.2)(9.1,-1.8)
  \psframe[linecolor=black, linewidth=0.04, dimen=outer](11.5,1.8)(9.1,0.6)
  \psline[linecolor=black, linewidth=0.04, arrowsize=0.05291666666666667cm 2.0,arrowlength=1.4,arrowinset=0.0]{->}(8.3,1.0)(9.1,1.0)
  \psline[linecolor=black, linewidth=0.04, linestyle=dashed, dash=0.17638889cm 0.10583334cm, arrowsize=0.05291666666666667cm 2.0,arrowlength=1.4,arrowinset=0.0]{->}(8.3,-0.6)(9.1,-0.6)
  \psline[linecolor=black, linewidth=0.04, linestyle=dotted, dotsep=0.10583334cm, arrowsize=0.05291666666666667cm 2.0,arrowlength=1.4,arrowinset=0.0]{->}(8.3,-1.4)(9.1,-1.4)
  \psline[linecolor=black, linewidth=0.04, linestyle=dashed, dash=0.17638889cm 0.10583334cm, arrowsize=0.05291666666666667cm 2.0,arrowlength=1.4,arrowinset=0.0]{->}(11.5,-0.6)(12.3,-0.6)
  \psline[linecolor=black, linewidth=0.04, linestyle=dotted, dotsep=0.10583334cm, arrowsize=0.05291666666666667cm 2.0,arrowlength=1.4,arrowinset=0.0]{->}(11.5,-1.4)(12.3,-1.4)
  \psline[linecolor=black, linewidth=0.04, arrowsize=0.05291666666666667cm 2.0,arrowlength=1.4,arrowinset=0.0]{->}(3.1820512,0.95897436)(3.7739928,0.95897436)(4.217949,0.13846155)
  \psline[linecolor=black, linewidth=0.04, linestyle=dashed, dash=0.17638889cm 0.10583334cm, arrowsize=0.05291666666666667cm 2.0,arrowlength=1.4,arrowinset=0.0]{->}(3.1820512,-0.2717949)(4.217949,-0.2717949)
  \psline[linecolor=black, linewidth=0.04, linestyle=dotted, dotsep=0.10583334cm, arrowsize=0.05291666666666667cm 2.0,arrowlength=1.4,arrowinset=0.0]{->}(3.1820512,-1.2974359)(3.7739928,-1.2974359)(4.217949,-0.6820513)
  \psline[linecolor=black, linewidth=0.04, arrowsize=0.05291666666666667cm 2.0,arrowlength=1.4,arrowinset=0.0]{->}(11.5,1.0)(12.3,1.0)
  \end{pspicture}
 }
}

{
\scriptsize
\linespread{1}
\begin{verbatim}
/opt/xNAS/
+ apache2
| + ports.conf
| + extra
| \ sites-available
+ app
| + lam
| | \ htdocs
| + ltb
| | \ htdocs
| \ static
+ ssl
| + certs
| \ private
\ files
  + staff
  + profesor
  \ p -> profesor

/etc/apache2
+ ports.conf
+ extra -> /opt/xNAS/apache2/extra
\ sites-available
  + default -> /opt/xNAS/apache2/sites-available/xnas.local
  + admin.xnas.local -> /opt/xNAS/apache2/sites-available/admin.xnas.local
  \ reset.xnas.local -> /opt/xNAS/apache2/sites-available/reset.xnas.local

/etc/ssl
+ certs
| + ca.thesis.tonejito.info.crt -> /opt/xNAS/ssl/certs/ca.thesis.tonejito.info.crt
| + xnas.local.crt -> /opt/xNAS/ssl/certs/xnas.local.crt
| + admin.xnas.local.crt -> /opt/xNAS/ssl/certs/admin.xnas.local.crt
| \ reset.xnas.local.crt -> /opt/xNAS/ssl/certs/reset.xnas.local.crt
\ private
  + xnas.local.key -> /opt/xNAS/ssl/private/xnas.local.key
  + admin.xnas.local.key -> /opt/xNAS/ssl/private/admin.xnas.local.key
  \ reset.xnas.local.key -> /opt/xNAS/ssl/private/reset.xnas.local.key
\end{verbatim}
}

        \subsubsection {Configuraci\'{o}n del servicio}

Se instalan los certificados \textsc{SSL} del servidor web mediante los siguientes comandos:

{
\scriptsize
\linespread{1}
\begin{verbatim}
    # cd /etc/ssl/private
    # ln -vsf ../../../opt/xNAS/ssl/xnas.local.key
    # cd /etc/ssl/certs
    # ln -vsf ../../../opt/xNAS/ssl/ca.thesis.tonejito.info.crt
    # ln -vsf ../../../opt/xNAS/ssl/xnas.local.crt
    # c_rehash
\end{verbatim}
}

Para configurar de manera adecuada la directiva \texttt{NameVirtualHost} para los sitios por \textsc{HTTPS} se copia el archivo \texttt{ports.conf} al directorio principal de Apache \textit{httpd}.

{
\scriptsize
\linespread{1}
\begin{verbatim}
    # cd /opt/xNAS/apache2
    # cp ports.conf /etc/apache2
    # /etc/init.d/apache2 restart
    # apache2ctl -S
\end{verbatim}
}

Se habilita el acceso a las configuraciones espec\'{i}ficas del \textit{appliance} al hacer una \textit{liga simb\'{o}lica} al directorio \texttt{extra}:

{
\scriptsize
\linespread{1}
\begin{verbatim}
    # cd /etc/apache2
    # ln -vsf ../../opt/xNAS/apache2/extra
\end{verbatim}
}

Se instalan y habilitan los m\'{odulos} necesarios para que las funcionalidades del sitio se realizen de manera adecuada, esto incluye la autenticaci\'{o}n y la funcionalidad de \textsc{WebDAV}. Por \'{u}ltimo se reinicia el servicio para aplicar los cambios.

{
\scriptsize
\linespread{1}
\begin{verbatim}
    # aptitude install libapache2-mod-rpaf
    # cd /etc/apache2/sites-available
    # ln -vsf ../../../opt/xNAS/apache2/xnas.local
    # a2dissite default
    # a2enmod headers ssl rewrite ldap authnz_ldap dav dav_fs dav_lock
    # a2enmod proxy proxy_http rpaf
    # a2ensite default
    # /etc/init.d/apache2 restart
\end{verbatim}
}

%\ifdraft{Configuraci\'{o}n de Apache \textsc{httpd}.}{\verbatimincludebox{./etc/apache2/apache2.conf}}

    \section {Implementaci\'{o}n de las interfaces de usuario}

      \subsection {Acceso mediante cliente nativo de \textsc{WebDAV}}

Para

        \subsubsection{P\'{a}gina de inicio del sitio web}

Se desarroll\'{o} una p\'{a}gina para facilitar a los usuarios la b\'{u}squeda de los directorios asociados con sus materias

      \subsection {Interfaz de administraci\'{o}n \textit{LDAP Account Manager}}

Se descarga el \textit{tarball} del c\'{o}digo fuente y se descomprime en el directorio \texttt{lam/htdocs}:

{
\scriptsize
\linespread{1}
\begin{verbatim}
    # cd /opt/xNAS/app/lam
    # wget -c 'http://iweb.dl.sourceforge.net/project/lam/LAM/4.8/ldap-account-manager-4.8.tar.bz2'
    # tar -xvvjf ldap-account-manager-4.8.tar.bz2 -C /opt/xNAS/app/lam
    # mv ldap-account-manager-4.8 htdocs
\end{verbatim}
}

Se establecen permisos de escritura para el servidor web en los directorios \texttt{sess} y \texttt{temp}:

{
\scriptsize
\linespread{1}
\begin{verbatim}
    # chown -cR www-data:www-data /opt/xNAS/app/lam/htdocs/{sess,tmp}
\end{verbatim}
}

Se instalan los archivos de configuraci\'{o}n de la herramienta:

{
\scriptsize
\linespread{1}
\begin{verbatim}
    # cd /opt/xNAS/app/lam/htdocs/config
    # ln -vsf ../../lam.conf
    # ln -vsf ../../config.cfg
\end{verbatim}
}

Se instala el certificado, la llave privada y la configuraci\'{o}n del sitio, una vez realizado esto, se reinicia el servicio:

{
\scriptsize
\linespread{1}
\begin{verbatim}
    # cd /etc/ssl/private
    # ln -vsf ../../../opt/xNAS/ssl/admin.xnas.local.key
    # cd /etc/ssl/certs
    # ln -vsf ../../../opt/xNAS/ssl/admin.xnas.local.crt
    # c_rehash

    # cd /etc/apache2/sites-available
    # ln -vsf ../../../opt/xNAS/apache2/admin.xnas.local
    # a2ensite admin.xnas.local
    # /etc/init.d/apache2 reload
\end{verbatim}
}

      \subsection {Interfaz de cambio de contrase\~{n}a}

Se descarga el paquete de instalaci\'{o}n del sitio web oficial y se extrae en el directorio \texttt{ltb/htdocs}:

{
\scriptsize
\linespread{1}
\begin{verbatim}
    # cd /opt/xNAS/app/ltb
    # wget -c 'http://tools.ltb-project.org/attachments/download/497/\
      ltb-project-self-service-password-0.8.tar.gz'
    # tar -xvvzf ltb-project-self-service-password-0.8.tar.gz -C /opt/xNAS/app/ltb
    # mv ltb-project-self-service-password-0.8 htdocs
\end{verbatim}
}

Se instala el certificado, la llave privada y la configuraci\'{o}n del \texttt{VirtualHost}, hecho esto se reinicia el servicio para reflejar los cambios.

{
\scriptsize
\linespread{1}
\begin{verbatim}
    # cd /etc/ssl/private
    # ln -vsf ../../../opt/xNAS/ssl/reset.xnas.local.key
    # cd /etc/ssl/certs
    # ln -vsf ../../../opt/xNAS/ssl/reset.xnas.local.crt
    # c_rehash

    # cd /etc/apache2/sites-available
    # ln -vsf ../../../opt/xNAS/apache2/reset.xnas.local
    # a2ensite reset.xnas.local
    # /etc/init.d/apache2 reload
\end{verbatim}
}

    \section {Hardening}

      \subsection {Sistema operativo}

        \subsubsection {Actualizaciones desatendidas}

Una parte importante de la configuraci\'{o}n de seguridad de un sistema radica en la instalaci\'{o}n peri\'{o}dica de actualizaciones, al ejecutar el siguiente comando y responder \textbf{YES} en el cuadro de di\'{a}logo, se configura el sistema para descargar e instalar \emph{autom\'{a}ticamente} las actualizaciones de seguridad que sean liberadas por el fabricante del sistema operativo.

{
\scriptsize
\linespread{1}
\begin{verbatim}
    # dpkg-reconfigure unattended-upgrades

+------------------------+ Configuring unattended-upgrades +------------------------+
| Applying updates on a frequent basis is an important part of keeping systems      |
| secure. By default, updates need to be applied manually using package management  |
| tools. Alternatively, you can choose to have this system automatically download   |
| and install security updates.                                                     |
|                                                                                   |
| Automatically download and install stable updates?                                |
|                       <Yes>                          <No>                         |
+-----------------------------------------------------------------------------------+
\end{verbatim}
}

      \subsection {Reducci\'{o}n de componentes instalados}

Como parte de la configuraci\'{o}n de seguridad, es recomendable minimizar los paquetes que se instalan en el sistema operativo para disminuir la ventana de posibilidades de tener una intrusi\'{o}n. Con la siguiente configuraci\'{o}n se guardan en disco \'{u}nicamente las paginas de manual en ingles y se evita la instalacion de paquetes no esenciales del sistema operativo \footnote{http://aptitude.alioth.debian.org/doc/en/ch02s05s05.html}

{
\scriptsize
\linespread{1}
\begin{verbatim}
    # dpkg-reconfigure -p low locales
+----------------------------+ Configuring locales +--------------------------------+
| Locales are a framework to switch between multiple languages and allow users      |
| to use their language, country, characters, collation order, etc.                 |
|                                                                                   |
| Please choose which locales to generate. UTF-8 locales should be chosen by        |
| default, particularly for new installations. Other character sets may be useful   |
| for backwards compatibility with older systems and software.                      |
|                                                                                   |
| Locales to be generated:                                                          |
|                                                                                   |
|    [*] en_US.UTF-8 UTF-8                                                          |
|                                                                                   |
|                       <OK>                       <Cancel>                         |
+-----------------------------------------------------------------------------------+
\end{verbatim}
}

{
\scriptsize
\linespread{1}
\begin{verbatim}
+------------------------------+ Configuring locales +------------------------------+
| Many packages in Debian use locales to display text in the correct language for   |
| the user. You can choose a default locale for the system from the generated       |
| locales.                                                                          |
|                                                                                   |
| This will select the default language for the entire system. If this system is a  |
| multi-user system where not all users are able to speak the default language,     |
| they will experience difficulties.                                                |
|                                                                                   |
| Default locale for the system environment:                                        |
|                                                                                   |
|                                 en_US.UTF-8                                       |
|                                                                                   |
|                    <Ok>                        <Cancel>                           |
+-----------------------------------------------------------------------------------+

Generating locales (this might take a while)...
  en_US.UTF-8... done
Generation complete.

\end{verbatim}
}

{
\scriptsize
\linespread{1}
\begin{verbatim}
  # aptitude install localepurge
+----------------------------+ Configuring localepurge +----------------------------+
| localepurge will remove all locale files from your system but the ones for the    |
| language codes you select now. Usually two character locales like "de" or "pt"    |
| rather than "de_DE" or "pt_BR" contain the major portion of localizations. So     |
| please select both for best support of your national language settings.           |
| The entries from /etc/locale.gen will be preselected if no prior configuration    |
| has been successfully completed.                                                  |
|                                                                                   |
| Selecting locale files                                                            |
|                                                                                   |
|    [*] en                                                                         |
|    [*] en_US.UTF-8                                                                |
|                                   <Ok>                                            |
+-----------------------------------------------------------------------------------+
\end{verbatim}
}

{
\scriptsize
\linespread{1}
\begin{verbatim}
Yes
+----------------------------+ Configuring localepurge +----------------------------+
| Based on the same locale information you chose above, localepurge can also delete |
| superfluous localized man pages.                                                  |
|                                                                                   |
| Also delete localized man pages?                                                  |
|                                                                                   |
|                     <Yes>                        <No>                             |
+-----------------------------------------------------------------------------------+
\end{verbatim}
}

{
\scriptsize
\linespread{1}
\begin{verbatim}
No
+----------------------------+ Configuring localepurge +----------------------------+
| If you are content with the selection of locales you chose to keep and don't want |
| to care about whether to delete or keep newly found locales, just deselect this   |
| option to automatically remove new locales you probably wouldn't care about       |
| anyway. If you select this option, you will be given the opportunity to decide    |
| whether to keep or delete newly introduced locales.                               |
|                                                                                   |
| Inform about new locales?                                                         |
|                                                                                   |
|                     <Yes>                        <No>                             |
+-----------------------------------------------------------------------------------+
\end{verbatim}
}

Para reducir los componentes instalados en el sistema se aplicaron políticas en el gestor de paquetes para evitar la instalación de software adicional cuando se agrega un paquete al sistema, esto ayuda tanto a reducir la complejidad del sistema, como el espacio en disco utilizado por el sistema operativo.

{
\scriptsize
\linespread{1}
\begin{verbatim}
# cat > /etc/apt/apt.conf.d/10Hardening << EOF
Apt::Install-Recommends "false"
Apt::Install-Suggests   "false"
EOF

\end{verbatim}
}

      \subsection {OpenSSH}

Para la configuraci\'{o}n del servicio de \textsc{SSH} se aplicaron las siguientes opciones:

\begin{itemize}
  \item Restringir el acceso administrativo directo \'{u}nicamente mediante autenticaci\'{o}n de llave p\'{u}blica.
    \code{PermitRootLogin without-password}
  \item Deshabilitar el acceso a cuentas que tengan una contrase\~{n}a vac\'{i}a.
    \code{PermitEmptyPasswords no}
  \item Permitir el acceso al grupo de usuarios de la \textit{Unidad de C\'{o}mputo} de la \textsc{DICyG}.
    \code{AllowGroups adm staff support}
  \item  Mostrar una pantalla antes de iniciar la sesi\'{o}n en la que se listen las pol\'{i}ticas de uso del sistema.
    \code{Banner /etc/issue.net}
  \item Evitar la desconexi\'{o}n manteniendo actividad en la sesi\'{o}n.
    \code{TCPKeepAlive yes}
    \code{ClientAliveInterval 30}
  \item Evitar la redirecci\'{o}n de puertos y sesiones gr\'{a}ficas.
    \code{AllowTCPForwarding no}
    \code{GatewayPorts no}
    \code{X11Forwarding no}
  \item Desconectar al usuario si no inicia sesi\'{o}n en 60 segundos.
    \code{LoginGraceTime 60}
\end{itemize}

        \subsubsection {Firewall}

Para filtrar el tr\'{a}fico de red se establecieron las siguientes pol\'{i}ticas de \textit{firewall} de host en el equipo:

\begin{itemize}
  \item Se permite todo el tr\'{a}fico proveniente de la interfaz \textit{loopback}.
  \item Las conexiones a trav\'{e}s de \textsc{SSH} se permiten \'{u}nicamente desde los siguientes segmentos confiables de red:
  \begin{enumerate}
    \item Segmentos de \textsc{RedUNAM}: \texttt{132.247.0.0/16} , \texttt{132.248.0.0/16}
    \item Red interna de la \textsc{DICyG}.
  \end{enumerate}
  \item Se permiten las conexiones por \textsc{HTTPS} desde cualquier host, ya sea de la red interna, segmentos de \textsc{RedUNAM} o Internet.
  \item Las conexiones al demonio de \textsc{LDAP} se permiten \'{u}nicamente desde \textit{localhost}.
\end{itemize}

\diagramblock
{Trusted Networks}
{TrustedNetworks}
{
 \psscalebox{1.0 1.0} % Change this value to rescale the drawing.
 {
 \begin{pspicture}(0,-1.84)(3.68,1.84)
 \rput[bl](1.225,1.48){Internet}
 \rput[bl](0.89,0.46){\textsc{RedUNAM}}
 \rput[bl](1.23,-0.76){\textsc{DICyG}}
 \pscircle[linecolor=black, linewidth=0.04, dimen=outer](1.84,-0.58){0.78}
 \pscircle[linecolor=black, linewidth=0.04, dimen=outer](1.84,-0.19){1.45}
 \psframe[linecolor=black, linewidth=0.04, dimen=outer](3.68,1.84)(0.0,-1.84)
 \end{pspicture}
 }
}

Para implementar las reglas de firewall se utiliz\'{o} \textit{iptables}, el software est\'{a}ndar para filtrado de paquetes en \textsc{GNU/Linux} y el paquete \textit{iptables-persistent} que sirve para aplicar las reglas guardadas en cada inicio del sistema operativo. El programa se instala con \textit{aptitude} y al configurarlo pregunta si se guardar\'{a}n las reglas de \textsc{IPv4} e \textsc{IPv6}.

{
\scriptsize
\linespread{1}
\begin{verbatim}
    # aptitude install iptables-persistent

+--------------------| Configuring iptables-persistent |--------------------+
|                                                                           |
| Current iptables rules can be saved to the configuration file             |
| /etc/iptables/rules.v4. These rules will then be loaded automatically     |
| during system startup.                                                    |
|                                                                           |
| Rules are only saved automatically during package installation. See the   |
| manual page of iptables-save(8) for instructions on keeping the rules     |
| file up-to-date.                                                          |
|                                                                           |
| Save current IPv4 rules?                                                  |
|                                                                           |
|                    <Yes>                       <No>                       |
|                                                                           |
+---------------------------------------------------------------------------+
\end{verbatim}
}

      \subsection {Servicios de red}

Restricci\'{o}n de acceso

        \subsubsection {ssh}

\begin{itemize}
  \item Se permite el acceso \'{u}nicamente al grupo de usuarios pertenecientes al staff de la DICyG
\end{itemize}

        \subsubsection {http}

\begin{itemize}
  \item Se restringe el acceso a cada \texttt{VirtualHost} desde determinado rango de direcciones IP
\end{itemize}

        \subsubsection {ldap}

\begin{itemize}
  \item Se permiten conexiones locales por \texttt{socket} \texttt{ldapi:///} y conexiones al puerto 389 para diagn\'{o}stico
\end{itemize}

