%\chapter{\'{I}ndice}
%\label{chap:index}

\addcontentsline{toc}{chapter}{\'{I}ndice}
\textbf{\Large \'{I}ndice Propuesto}

%\section{}
{
% Label items with arabic numbers
\renewcommand{\labelenumi}{\arabic{enumi}.}
\renewcommand{\labelenumii}{\arabic{enumi}.\arabic{enumii}}
\renewcommand{\labelenumiii}{\arabic{enumi}.\arabic{enumii}.\arabic{enumiii}}
\renewcommand{\labelenumiv}{\arabic{enumi}.\arabic{enumii}.\arabic{enumiii}.\arabic{enumiv}}

% On the final index only 3 levels will be shown (x.y.z) instead of the current 4 (x.y.z.w)

\begin{itemize}
  % •
  \item Introducci\'{o}n
  \begin{itemize}
    \item Objetivo
    \item Estructura de la tesis
  \end{itemize}
\end{itemize}
\begin{enumerate}
  % 1
  \item Marco Te\'{o}rico
  \begin{enumerate}
    \item Sistemas de almacenamiento
    \begin{enumerate}
      \item Ventajas y desventajas
      \item Escenarios de falla
      \item M\'{e}todos de protecci\'{o}n
      \begin{enumerate}
        \item T\'{e}cnicas de respaldo
        \item Arreglos RAID
      \end{enumerate}
    \end{enumerate}
    \item Appliances
    \item Seguridad Inform\'{a}tica
    \begin{enumerate}
      \item Principios de seguridad inform\'{a}tica
      \item Vulnerabilidades
      \item Hardening
    \end{enumerate}
    \item GNU/Linux
    \begin{enumerate}
      \item Historia
      \item Uso de GNU/Linux en la industria
      \item Distribuciones de GNU/Linux
      \item Debian GNU/Linux
    \end{enumerate}
    \item Protocolo HTTP
    \begin{enumerate}
      \item HTTPS - HTTP over SSL
      \item WebDAV
    \end{enumerate}
    \item Protocolo LDAP
    \begin{enumerate}
      \item Directorio de usuarios
    \end{enumerate}
    \item Protocolo SSH
    \begin{enumerate}
      \item SCP - Secure Copy
      \item SFTP - Secure FTP
      \item SSHFS - Secure Shell Filesystem
    \end{enumerate}
  \end{enumerate}
  % 2
  \item Definici\'{o}n del problema y soluci\'{o}n propuesta
  \begin{enumerate}
    \item Problem\'{a}tica actual
    \item Soluci\'{o}n propuesta
    \item Tecnolog\'{i}as a utilizar
    \item Arquitectura del prototipo
    \begin{enumerate}
      \item Diagrama funcional
      \item Autenticaci\'{o}n centralizada
      \begin{enumerate}
        \item Autenticaci\'{o}n por medio de directorio
        \item Estructura del directorio
      \end{enumerate}
      \item Mecanismos de acceso a los archivos
      \begin{enumerate}
        \item Acceso por HTTP mediante WebDAV
        \item Acceso por SSH utilizando SCP
        \item Acceso por SSH utilizando SFTP
        \item Acceso por SSH utilizando SSHFS
      \end{enumerate}
      \item Interfaces de usuario
      \begin{enumerate}
        \item Rails
        \item Dise\~{n}o de la interfaz de administraci\'{o}n
        \item Dise\~{n}o de la interfaz de cambio de contrase\~{n}a
      \end{enumerate}
    \end{enumerate}
    \item Especificaci\'{o}n del appliance
    \begin{enumerate}
      \item Hardware
      \item Software
      \item L\'{i}mites
    \end{enumerate}
  \end{enumerate}
  % 3
  \item Implementaci\'{o}n de la soluci\'{o}n
  \begin{enumerate}
    \item Configuraci\'{o}n del sistema operativo
    \begin{enumerate}
      \item Arreglo de discos RAID
      \item Reducci\'{o}n de componentes instalados
    \end{enumerate}
    \item Configuraci\'{o}n de las herramientas
    \begin{enumerate}
      \item OpenLDAP
      \item OpenSSH
      \item Apache httpd
    \end{enumerate}
    \item Desarrollo de las interfaces de usuario
    \begin{enumerate}
      \item Interfaz de administraci\'{o}n
      \item Interfaz de cambio de contrase\~{n}a
    \end{enumerate}
    \item Hardening
    \begin{enumerate}
      \item Sistema operativo
      \begin{enumerate}
        \item Firewall
      \end{enumerate}
      \item Servicios de red
      \begin{enumerate}
        \item ssh
        \item http
        \item ldap
      \end{enumerate}
    \end{enumerate}
  \end{enumerate}
  % 4
  \item Pruebas
  \begin{enumerate}
    \item Compatibilidad multiplataforma
    \begin{enumerate}
      \item GNU/Linux
      \item Mac OS X
      \item Apple iOS
      \item Android
      \item Windows
    \end{enumerate}
    \item Pruebas de carga al sistema
    \begin{enumerate}
      \item N\'{u}mero m\'{a}ximo de usuarios
      \item Rendimiento del arreglo RAID
    \end{enumerate}
    \item Pruebas de seguridad
    \begin{enumerate}
      \item SQLi - SQL Inyection
      \item XSS - Cross-site scripting
      \item Directory traversal
    \end{enumerate}
  \end{enumerate}
  % 5
  \item Conclusiones
  \begin{enumerate}
    \item Resultados obtenidos
    \item Oportunidades de Mejora
  \end{enumerate}
\end{enumerate}
} %end \section
