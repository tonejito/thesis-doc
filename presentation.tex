%%%%%%%%%%%%%%%%%%%%%%%%%%%%%%%%%%%%%%%%%
% BA04 presentation 
% LaTeX Template
% Version 1.0 (14/12/14)
%
% This template has been downloaded from:
% http://www.LaTeXTemplates.com
%
% License:
% CC BY-NC-SA 3.0 (http://creativecommons.org/licenses/by-nc-sa/3.0/)
%
%%%%%%%%%%%%%%%%%%%%%%%%%%%%%%%%%%%%%%%%%

%----------------------------------------------------------------------------------------
%	PACKAGES AND THEMES
%----------------------------------------------------------------------------------------

\documentclass{beamer}

\mode<presentation> {

% The Beamer class comes with a number of default slide themes
% which change the colors and layouts of slides. Below this is a list
% of all the themes, uncomment each in turn to see what they look like.

%\usetheme{default}
%\usetheme{AnnArbor}
%\usetheme{Antibes}
%\usetheme{Bergen}
%\usetheme{Berkeley}
\usetheme{Berlin}
%\usetheme{Boadilla}
%\usetheme{CambridgeUS}
%\usetheme{Copenhagen}
%\usetheme{Darmstadt}
%\usetheme{Dresden}
%\usetheme{Frankfurt}
%\usetheme{Goettingen}
%\usetheme{Hannover}
%\usetheme{Ilmenau}
%\usetheme{JuanLesPins}
%\usetheme{Luebeck}
%\usetheme{Madrid}
%\usetheme{Malmoe}
%\usetheme{Marburg}
%\usetheme{Montpellier}
%\usetheme{PaloAlto}
%\usetheme{Pittsburgh}
%\usetheme{Rochester}
%\usetheme{Singapore}
%\usetheme{Szeged}
%\usetheme{Warsaw}

% As well as themes, the Beamer class has a number of color themes
% for any slide theme. Uncomment each of these in turn to see how it
% changes the colors of your current slide theme.

%\usecolortheme{albatross}
%\usecolortheme{beaver}
%\usecolortheme{beetle}
%\usecolortheme{crane}
%\usecolortheme{dolphin}
%\usecolortheme{dove}
%\usecolortheme{fly}
%\usecolortheme{lily}
%\usecolortheme{orchid}
%\usecolortheme{rose}
%\usecolortheme{seagull}
%\usecolortheme{seahorse}
%\usecolortheme{whale}
\usecolortheme{wolverine}

%\setbeamertemplate{footline} % To remove the footer line in all slides uncomment this line
%\setbeamertemplate{footline}[page number] % To replace the footer line in all slides with a simple slide count uncomment this line

\setbeamertemplate{navigation symbols}{} % To remove the navigation symbols from the bottom of all slides uncomment this line
}

\usepackage{graphicx} % Allows including images
\usepackage{booktabs} % Allows the use of \toprule, \midrule and \bottomrule in tables
\usepackage[english]{babel}
\usepackage[utf8x]{inputenc}
%----------------------------------------------------------------------------------------
%	TITLE PAGE
%----------------------------------------------------------------------------------------

\title[Les Véhicules Electriques Solaires]{Les Véhicules Electriques Solaires} % The short title appears at the bottom of every slide, the full title is only on the title page

\author{
Battini Aline\\
Cordeiro Juliana\\
Gonçalves Samuel\\
Planchais Elric} % Your name

\institute[UTC] % Your institution as it will appear on the bottom of every slide, may be shorthand to save space
{
Université de Technologie de Compiègne \\ % Your institution for the title page
\medskip
\textit{BA04} % Your email address
}
\date{22 decembre 2014.} % Date, can be changed to a custom date

\begin{document}

\begin{frame}
\titlepage % Print the title page as the first slide
\end{frame}

\begin{frame}
\frametitle{Sommaire} % Table of contents slide, comment this block out to remove it
\tableofcontents % Throughout your presentation, if you choose to use \section{} and \subsection{} commands, these will automatically be printed on this slide as an overview of your presentation
\end{frame}

%----------------------------------------------------------------------------------------
%	PRESENTATION SLIDES
%----------------------------------------------------------------------------------------

%------------------------------------------------
\section{Introdution} % Sections can be created in order to organize your presentation into discrete blocks, all sections and subsections are automatically printed in the table of contents as an overview of the talk
%------------------------------------------------

\subsection{Bref Historique des Technologies} % A subsection can be created just before a set of slides with a common theme to further break down your presentation into chunks

\subsection{Véhicules Solaires}
\section{Solar Impulse}
\subsection{Historique Solar Impulse}
\subsubsection{HB-SIA x HB-SIB}
\subsection{Principe du vol}
\subsubsection{Moteur}
\subsubsection{Batterie}
\subsubsection{Cellules Photovoltaïques}
\subsection{Proposition d'amélioration}

\section{Conclusion}
\section{Sources}

\begin{frame}
\frametitle{Introdution}

L'énergie électromagnétique transmise par le soleil est la base de toute vie terrestre. L'énergie solaire est exploitée en capturant la chaleur (par chauffage solaire) ou de la lumière du soleil (grâce à des cellules 
photovoltaïques).\\~\\

L’énergie solaire présente de nombreux avantages. C’est énergie renouvelables : elle ne dégage pas de gaz à effet de serre, elle ne produit pas de déchet toxique, et elle est gratuite.\\~\\



\end{frame}
%------------------------------------------------
\begin{frame}

\frametitle{Introdution}
\framesubtitle{Bref Historique des Technologies}
Il existe plusieurs types de cellules photovoltaïques.\\~\\
Les cellules photovoltaïques à base de silicium (multicristallin et monocristallin) sont actuellement les plus utilisées (0.85 du parc installé) et ont un rendement de l’ordre de 0.15.)

\begin{figure}
%\includegraphics{images/Les-diff_rents-types-de-cellules-photovolta_ques.jpg}
\end{figure}

\end{frame}
%------------------------------------------------

\begin{frame}
\frametitle{Introdution}
\framesubtitle{Bref Historique des Techonologies}
Globalement, les coûts de production et d’installation des panneaux solaires sont en baisse alors même que la production mondiale augmente.
\begin{figure}
%\includegraphics[width=0.55\textwidth]{images/historique_des_couts.png}
\end{figure}
\begin{figure}
%\includegraphics[width=0.55\textwidth]{images/evolution_de_la_production.png}
\end{figure}

\end{frame}
%------------------------------------------------

\begin{frame}
\frametitle{Introdution}
\framesubtitle{Bref Historique des Techonologies}
Les technologies photovoltaïques de demain auront pour objectifs d’augmenter les rendements, de diminuer les coûts de production mais également de limiter l’impact environnemental de la construction et du démantèlement des panneaux. Les cellules organiques promettent un rendement de l’ordre de 5 à 10 pourcent  pour un processus de fabrication simple et bon marché. La technologie est actuellement au stade de recherche.

\end{frame}
%------------------------------------------------

\begin{frame}
\frametitle{Introdution}
\framesubtitle{Véhicules Solaires}

\end{frame}

%------------------------------------------------

\begin{frame}
\frametitle{Solar Impulse}
\framesubtitle{Historique}
"Si tous les avions devenaient solaires, ce serait formidable." - Yann-Arthus Bertrand

\begin{figure}
%\includegraphics[width=0.7\textwidth]{images/solar-impulse.png}
\end{figure}

\end{frame}


%------------------------------------------------

\begin{frame}
\frametitle{Solar Impulse}
\framesubtitle{Historique}
\begin{block}{Juillet 2010}
L’aventure Solar Impulse commence avec un 1er avion prototype immatriculé HB-SIA. 
\end{block}

\begin{block}{Juillet 2011}
Mise en chantier du 2ème avion immatriculé HB-SIB. Solar Impulse 2 devra pouvoir voler plus de cent-vingt heures d'affilée, cinq jours et cinq nuits, le temps dont il a besoin pour traverser le Pacifique ou l'Atlantique.
\end{block}

\begin{block}{Juin 2014}
1er essaie de vol avec succès pour le HB-SIB.
\end{block}


\begin{block}{Avril-Juillet 2015}
Tour du monde prévu en 2015 au départ d’Ahbu Dahbi pour une période de 5 mois avec des vols prévu jusqu’à 5 jours sans escale. 
\end{block}

\end{frame}


%-------------------------------------------------------------

\begin{frame}
\frametitle{Solar Impulse}
\framesubtitle{Comparaison HB-SIA/HB-SIB}

\begin{columns}[c] % The "c" option specifies centered vertical alignment while the "t" option is used for top vertical alignment

\column{.45\textwidth} % Left column and width
\textbf{SIA}
\begin{enumerate}
\item Envergure: 63.4m
\item Taille du Cockpit: 1.6m3
\item Poids: 1.3 tonnes
\end{enumerate}

\column{.5\textwidth} % Right column and width
\textbf{SIB}
\begin{enumerate}
\item Envergure: 72m
\item Taille du Cockpit: 3.8m3
\item Poids: 2.3 tonnes
\end{enumerate}

\end{columns}

L’avion HB-SIB:
\begin{itemize}
\item Charge utile sera augmentée
\item Circuits électriques rendus étanches pour pouvoir voler sous la pluie
\item Systèmes redondants améliorent la fiabilité
\item Bayer Material Sciences fait profiter le projet de ses nanotechnologies 
\item Décision utilise de fibres de carbones d’une légèreté jamais vues jusqu’à aujourd’hui.
\end{itemize}
\end{frame}


%------------------------------------------------


\begin{frame}
\frametitle{Solar Impulse}
\framesubtitle{Les caractéristiques et le  « fonctionnement » du Solar Impulse SIB}

Le projet Solar Impulse totalise aujourd’hui plus de 12 années d’études de faisabilité, de conception, design et construction. Le Solar Impulse :\\

\begin{itemize}
\item est composé de 80 ingénieurs et techniciens de Solar Impulse
\item 80 partenaires technologies
\item plus de 100 conseillers et fournisseurs
\item investissement d’environ 100 millions d’euros


\end{itemize}

\end{frame}

%------------------------------------------------
\begin{frame}
\frametitle{Solar Impulse}
\framesubtitle{Principe du Vol}




\end{frame}

%------------------------------------------------

\begin{frame}
\frametitle{Solar Impulse}
\framesubtitle{Propositions d'améllioration}




\end{frame}
%------------------------------------------------

\begin{frame}
\frametitle{Conclusion}





\end{frame}
%------------------------------------------------

\begin{frame}
\frametitle{Sources}





\end{frame}
%------------------------------------------------
\begin{frame}
\frametitle{Table}
\begin{table}
\begin{tabular}{l l l}
\toprule
\textbf{Treatments} & \textbf{Response 1} & \textbf{Response 2}\\
\midrule
Treatment 1 & 0.0003262 & 0.562 \\
Treatment 2 & 0.0015681 & 0.910 \\
Treatment 3 & 0.0009271 & 0.296 \\
\bottomrule
\end{tabular}
\caption{Table caption}
\end{table}
\end{frame}

%------------------------------------------------

\begin{frame}
\frametitle{Theorem}
\begin{theorem}[Mass--energy equivalence]
$E = mc^2$
\end{theorem}
\end{frame}

%------------------------------------------------

\begin{frame}[fragile] % Need to use the fragile option when verbatim is used in the slide
\frametitle{Verbatim}
\begin{example}[Theorem Slide Code]
\begin{verbatim}
\begin{frame}
\frametitle{Theorem}
\begin{theorem}[Mass--energy equivalence]
$E = mc^2$
\end{theorem}
\end{frame}\end{verbatim}
\end{example}
\end{frame}

%------------------------------------------------

\begin{frame}
\frametitle{Figure}
Uncomment the code on this slide to include your own image from the same directory as the template .TeX file.
%\begin{figure}
%\includegraphics[width=0.8\linewidth]{test}
%\end{figure}
\end{frame}

%------------------------------------------------

\begin{frame}[fragile] % Need to use the fragile option when verbatim is used in the slide
\frametitle{Citation}
An example of the \verb|\cite| command to cite within the presentation:\\~

This statement requires citation \cite{p1}.
\end{frame}

%------------------------------------------------

\begin{frame}
\frametitle{References}
\footnotesize{
\begin{thebibliography}{99} % Beamer does not support BibTeX so references must be inserted manually as below
\bibitem[Smith, 2012]{p1} John Smith (2012)
\newblock Title of the publication
\newblock \emph{Journal Name} 12(3), 45 -- 678.
\end{thebibliography}
}
\end{frame}

%------------------------------------------------

\begin{frame}
\Huge{\centerline{The End}}
\end{frame}

%----------------------------------------------------------------------------------------

\end{document}
