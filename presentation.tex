%%%%%%%%%%%%%%%%%%%%%%%%%%%%%%%%%%%%%%%%%
% BA04 presentation 
% LaTeX Template
% Version 1.0 (14/12/14)
%
% This template has been downloaded from:
% http://www.LaTeXTemplates.com
%
% License:
% CC BY-NC-SA 3.0 (http://creativecommons.org/licenses/by-nc-sa/3.0/)
%
%%%%%%%%%%%%%%%%%%%%%%%%%%%%%%%%%%%%%%%%%

%----------------------------------------------------------------------------------------
%	PACKAGES AND THEMES
%----------------------------------------------------------------------------------------

\documentclass{beamer}
\usepackage{etex}

\mode<presentation> {

% The Beamer class comes with a number of default slide themes
% which change the colors and layouts of slides. Below this is a list
% of all the themes, uncomment each in turn to see what they look like.

\usetheme{Berlin}

% As well as themes, the Beamer class has a number of color themes
% for any slide theme. Uncomment each of these in turn to see how it
% changes the colors of your current slide theme.

\usecolortheme{wolverine}

%\setbeamertemplate{footline} % To remove the footer line in all slides uncomment this line
%\setbeamertemplate{footline}[page number] % To replace the footer line in all slides with a simple slide count uncomment this line

\setbeamertemplate{navigation symbols}{} % To remove the navigation symbols from the bottom of all slides uncomment this line
% http://tex.stackexchange.com/a/113817
\setbeamercolor{title}{fg=black,bg=white}
\setbeamercolor{frametitle}{fg=black,bg=white}
}

\usepackage{graphicx} % Allows including images
\usepackage{booktabs} % Allows the use of \toprule, \midrule and \bottomrule in tables
\usepackage[english]{babel}
\usepackage[utf8x]{inputenc}

% <LatexDraw>
\usepackage{pstricks}
\usepackage{tikz}
% </LatexDraw>

% http://latex-beamer-class.10966.n7.nabble.com/justified-text-in-beamer-td1491.html
\usepackage{ragged2e}

% Placeholders
\newcommand{\UniversityName}{Universidad Nacional Aut\'{o}noma de M\'{e}xico}
\newcommand{\UniversityShortName}{UNAM}
\newcommand{\FacultyName}{Facultad de Ingenier\'{i}a}
\newcommand{\AuthorName}{Andr\'{e}s Leonardo Hern\'{a}ndez Berm\'{u}dez}
\newcommand{\CollegeMajor}{Ingenier\'{i}a en Computaci\'{o}n}
\newcommand{\PresentationTitle}{xNAS: Appliance de almacenamiento en red por medio de WebDAV}
\newcommand{\PresentationDate}{Enero 2016}


%----------------------------------------------------------------------------------------
%	TITLE PAGE
%----------------------------------------------------------------------------------------

% The short title appears at the bottom of every slide, the full title is only on the title page
\title
  [\PresentationTitle \hspace{25em} \thepage] % Bottom of each slide
  {\PresentationTitle}                        % Title page

\author{\AuthorName} % Your name

\institute[\UniversityShortName] % Your institution as it will appear on the bottom of every slide, may be shorthand to save space
{
%\large
\UniversityName \\ % Your institution for the title page
\medskip
\FacultyName \\
\medskip
\CollegeMajor
}
\date{\PresentationDate} % Date, can be changed to a custom date

\begin{document}

\begin{frame}
% Print the title page as the first slide
\titlepage
\end{frame}

\begin{frame}
% Table of contents slide, comment this block out to remove it
\frametitle{\'{I}ndice}
% Throughout your presentation, if you choose to use \section{} and \subsection{} commands, these will automatically be printed on this slide as an overview of your presentation
\tableofcontents 
\end{frame}

%----------------------------------------------------------------------------------------
%	PRESENTATION SLIDES
%----------------------------------------------------------------------------------------

%------------------------------------------------
% Sections can be created in order to organize your presentation into discrete blocks, all sections and subsections are automatically printed in the table of contents as an overview of the talk
\section{Problem\'{a}tica}
%------------------------------------------------

% A subsection can be created just before a set of slides with a common theme to further break down your presentation into chunks
  \subsection{subsection} 

\section{Objetivos}
  \subsection{subsection}
\section{Desarrollo}
  \subsection{subsection}
    \subsubsection{subsubsection}
  \subsection{subsection}
    \subsubsection{subsubsection}
    \subsubsection{subsubsection}
    \subsubsection{subsubsection}
  \subsection{subsection}

\section{Pruebas}
\section{Resultados}

%------------------------------------------------
\begin{frame}
\frametitle{Problem\'{a}tica}

\begin{itemize}
\justifying
  \item Los profesores de la Divisi\'{o}n de Ingenier\'{i}as Civil y Geom\'{a}tica utilizan medios de almacenamiento externos como memorias USB y discos duros externos para distribuir la informaci\'{o}n que utilizan para sus cursos.
\\~\\
  \item El personal de la Unidad de C\'{o}mputo utiliza discos \'{o}pticos para instalar el software en los equipos a los que da servicio.
\end{itemize}

%
\psscalebox{0.5 0.5} % Change this value to rescale the drawing.
{
  \begin{pspicture}(0,-1.7033483)(20.986668,1.7033483)
    \pscustom[linecolor=black, linewidth=0.04]
    {
      \newpath
      \moveto(0.32,1.2233483)
    }
    
% % Floppy
    \psline[linecolor=black, linewidth=0.02](1.5333333,1.2200148)(1.5333333,0.3533482)(2.6,0.3533482)(2.6,1.2200148)(2.6,1.2200148)
    \psline[linecolor=black, linewidth=0.02](1.0,-1.3799851)(1.0,0.06445931)(2.8666666,0.06445931)(2.8666666,-1.3799851)
    \psline[linecolor=black, linewidth=0.02](1.0,1.2200148)(1.0,0.3533482)(2.6,0.3533482)(2.8666666,0.3533482)(2.8666666,1.2200148)(2.8666666,1.2200148)
    \psline[linecolor=black, linewidth=0.04](0.6,1.2200148)(0.6,-1.3799851)(3.2666667,-1.3799851)(3.2666667,0.931126)(3.0,1.2200148)(0.6,1.2200148)(0.6,1.2200148)
% % CD / DVD / HD-DVD / BD
    \pscircle[linecolor=black, linewidth=0.04, dimen=outer](5.6,-0.0799852){1.6}
    \pscircle[linecolor=black, linewidth=0.04, dimen=outer](5.6,-0.0799852){0.26666668}
    \pscircle[linecolor=black, linewidth=0.02, dimen=outer](5.6,-0.0799852){0.53333336}
% USB Drive
    \psframe[linecolor=black, linewidth=0.04, dimen=outer](12.8,0.45334813)(11.733334,-0.6133185)
    \psframe[linecolor=black, linewidth=0.04, dimen=outer](11.773334,0.7200148)(8.44,-0.8799852)
    \psframe[linecolor=black, linewidth=0.02, dimen=outer](12.355556,0.27557036)(12.177777,0.09779258)
    \psframe[linecolor=black, linewidth=0.02, dimen=outer](12.355556,-0.24442965)(12.177777,-0.42220742)
    \psframe[linecolor=black, linewidth=0.02, dimen=outer](12.004444,0.00890369)(11.915556,-0.16887408)
% Network share
    \pspolygon[linecolor=black, linewidth=0.04](13.626521,0.07833838)(15.973188,0.07833838)(15.466521,1.145005)(14.119855,1.145005)
    \rput{-308.4926}(6.159207,-11.840579){\pstriangle[linecolor=black, linewidth=0.02, fillstyle=solid, dimen=outer](15.351614,0.17833838)(0.19,0.57)}
    \psellipse[linecolor=black, linewidth=0.02, dimen=outer](14.799999,0.8083384)(0.15,0.07)
    \psellipse[linecolor=black, linewidth=0.02, dimen=outer](14.799999,0.8083384)(0.725,0.27)
    \psframe[linecolor=black, linewidth=0.04, dimen=outer](16.0,0.10500504)(13.6,-0.45499495)
    \psline[linecolor=black, linewidth=0.14](14.8,-0.45499495)(14.8,-1.254995)
    \psline[linecolor=black, linewidth=0.14, dotsize=0.07055555555555555cm 2.0,dotsize=0.07055555555555555cm 2.0]{cc-cc}(13.466666,-1.254995)(16.133333,-1.254995)
  \end{pspicture}
}
%


\end{frame}
%------------------------------------------------
\begin{frame}
\frametitle{Problem\'{a}tica}
\framesubtitle{Desventajas}
\justifying

\begin{itemize}
\justifying
  \item No existen controles de acceso a los archivos
\\~\\
  \item Es propenso a borrado accidental de archivos
\\~\\
  \item La velocidad de transferencia es baja
\\~\\
  \item Puede propagar infecciones de \textit{malware}
\\~\\
  \item No soporta la copia simultanea de archivos
\end{itemize}

\end{frame}
%------------------------------------------------
\begin{frame}

\frametitle{Objetivo}
\justifying

Implementar un servidor dedicado de almacenamiento que incluya las cuentas de usuario desde un directorio \textsc{LDAP} y que transmita la informaci\'{o}n por medio de \textsc{WebDAV} a trav\'{e}s de un canal cifrado.

% Rack servers
%\psscalebox{0.5 0.5} % Change this value to rescale the drawing.
%{
  \begin{pspicture}(14,-1.7033483)(24,1.7033483)
    \psframe[linecolor=black, linewidth=0.04, dimen=outer](20.773333,-0.8233483)(17.48,-1.7033483)
    \psframe[linecolor=black, linewidth=0.02, dimen=outer](17.466667,-0.8966816)(17.266666,-1.6300149)
    \psframe[linecolor=black, linewidth=0.02, dimen=outer](20.986666,-0.8966816)(20.786667,-1.6300149)
    \psframe[linecolor=black, linewidth=0.04, dimen=outer](20.773333,-0.22334827)(17.48,-0.78334826)
    \psframe[linecolor=black, linewidth=0.02, dimen=outer](17.466667,-0.29668158)(17.266666,-0.71001494)
    \psframe[linecolor=black, linewidth=0.02, dimen=outer](20.986666,-0.29668158)(20.786667,-0.71001494)
    \psframe[linecolor=black, linewidth=0.02, dimen=outer](17.71,-0.32334822)(17.56,-0.48334822)
    \psframe[linecolor=black, linewidth=0.02, dimen=outer](17.91,-0.32334822)(17.76,-0.48334822)
    \psframe[linecolor=black, linewidth=0.02, dimen=outer](18.11,-0.32334822)(17.96,-0.48334822)
    \psframe[linecolor=black, linewidth=0.02, dimen=outer](18.31,-0.32334822)(18.16,-0.48334822)
    \psframe[linecolor=black, linewidth=0.02, dimen=outer](18.51,-0.32334822)(18.36,-0.48334822)
    \psframe[linecolor=black, linewidth=0.02, dimen=outer](18.71,-0.32334822)(18.56,-0.48334822)
    \psframe[linecolor=black, linewidth=0.02, dimen=outer](18.91,-0.32334822)(18.76,-0.48334822)
    \psframe[linecolor=black, linewidth=0.02, dimen=outer](19.11,-0.32334822)(18.96,-0.48334822)
    \psframe[linecolor=black, linewidth=0.02, dimen=outer](17.71,-0.5233482)(17.56,-0.68334824)
    \psframe[linecolor=black, linewidth=0.02, dimen=outer](17.91,-0.5233482)(17.76,-0.68334824)
    \psframe[linecolor=black, linewidth=0.02, dimen=outer](18.11,-0.5233482)(17.96,-0.68334824)
    \psframe[linecolor=black, linewidth=0.02, dimen=outer](18.31,-0.5233482)(18.16,-0.68334824)
    \psframe[linecolor=black, linewidth=0.02, dimen=outer](18.51,-0.5233482)(18.36,-0.68334824)
    \psframe[linecolor=black, linewidth=0.02, dimen=outer](18.71,-0.5233482)(18.56,-0.68334824)
    \psframe[linecolor=black, linewidth=0.02, dimen=outer](18.91,-0.5233482)(18.76,-0.68334824)
    \psframe[linecolor=black, linewidth=0.02, dimen=outer](19.11,-0.5233482)(18.96,-0.68334824)
    \psframe[linecolor=black, linewidth=0.02, dimen=outer](19.31,-0.32334822)(19.16,-0.48334822)
    \psframe[linecolor=black, linewidth=0.02, dimen=outer](19.51,-0.32334822)(19.36,-0.48334822)
    \psframe[linecolor=black, linewidth=0.02, dimen=outer](19.71,-0.32334822)(19.56,-0.48334822)
    \psframe[linecolor=black, linewidth=0.02, dimen=outer](19.91,-0.32334822)(19.76,-0.48334822)
    \psframe[linecolor=black, linewidth=0.02, dimen=outer](20.11,-0.32334822)(19.96,-0.48334822)
    \psframe[linecolor=black, linewidth=0.02, dimen=outer](20.31,-0.32334822)(20.16,-0.48334822)
    \psframe[linecolor=black, linewidth=0.02, dimen=outer](20.51,-0.32334822)(20.36,-0.48334822)
    \psframe[linecolor=black, linewidth=0.02, dimen=outer](20.71,-0.32334822)(20.56,-0.48334822)
    \psframe[linecolor=black, linewidth=0.02, dimen=outer](19.31,-0.5233482)(19.16,-0.68334824)
    \psframe[linecolor=black, linewidth=0.02, dimen=outer](19.51,-0.5233482)(19.36,-0.68334824)
    \psframe[linecolor=black, linewidth=0.02, dimen=outer](19.71,-0.5233482)(19.56,-0.68334824)
    \psframe[linecolor=black, linewidth=0.02, dimen=outer](19.91,-0.5233482)(19.76,-0.68334824)
    \psframe[linecolor=black, linewidth=0.02, dimen=outer](20.11,-0.5233482)(19.96,-0.68334824)
    \psframe[linecolor=black, linewidth=0.02, dimen=outer](20.31,-0.5233482)(20.16,-0.68334824)
    \psframe[linecolor=black, linewidth=0.02, dimen=outer](20.51,-0.5233482)(20.36,-0.68334824)
    \psframe[linecolor=black, linewidth=0.02, dimen=outer](20.71,-0.5233482)(20.56,-0.68334824)
    \psframe[linecolor=black, linewidth=0.02, dimen=outer](19.11,-0.90334827)(18.96,-1.6233482)
    \psframe[linecolor=black, linewidth=0.02, dimen=outer](18.91,-0.90334827)(18.76,-1.6233482)
    \psframe[linecolor=black, linewidth=0.02, dimen=outer](18.71,-0.90334827)(18.56,-1.6233482)
    \psframe[linecolor=black, linewidth=0.02, dimen=outer](18.51,-0.90334827)(18.36,-1.6233482)
    \psframe[linecolor=black, linewidth=0.02, dimen=outer](18.31,-0.90334827)(18.16,-1.6233482)
    \psframe[linecolor=black, linewidth=0.02, dimen=outer](18.11,-0.90334827)(17.96,-1.6233482)
    \psframe[linecolor=black, linewidth=0.02, dimen=outer](17.91,-0.90334827)(17.76,-1.6233482)
    \psframe[linecolor=black, linewidth=0.02, dimen=outer](17.71,-0.90334827)(17.56,-1.6233482)
    \psframe[linecolor=black, linewidth=0.02, dimen=outer](20.71,-0.90334827)(20.56,-1.6233482)
    \psframe[linecolor=black, linewidth=0.02, dimen=outer](20.51,-0.90334827)(20.36,-1.6233482)
    \psframe[linecolor=black, linewidth=0.02, dimen=outer](20.31,-0.90334827)(20.16,-1.6233482)
    \psframe[linecolor=black, linewidth=0.02, dimen=outer](20.11,-0.90334827)(19.96,-1.6233482)
    \psframe[linecolor=black, linewidth=0.02, dimen=outer](19.91,-0.90334827)(19.76,-1.6233482)
    \psframe[linecolor=black, linewidth=0.02, dimen=outer](19.71,-0.90334827)(19.56,-1.6233482)
    \psframe[linecolor=black, linewidth=0.02, dimen=outer](19.51,-0.90334827)(19.36,-1.6233482)
    \psframe[linecolor=black, linewidth=0.02, dimen=outer](19.31,-0.90334827)(19.16,-1.6233482)
    \pspolygon[linecolor=black, linewidth=0.04](17.5,0.6733185)(20.75,0.6733185)(20.365814,1.5233185)(17.873802,1.5133185)
    \psframe[linecolor=black, linewidth=0.04, dimen=outer](20.773333,0.69665176)(17.48,-0.18334827)
    \psframe[linecolor=black, linewidth=0.02, dimen=outer](17.466667,0.62331843)(17.266666,-0.11001493)
    \psframe[linecolor=black, linewidth=0.02, dimen=outer](20.986666,0.62331843)(20.786667,-0.11001493)
    \psframe[linecolor=black, linewidth=0.02, dimen=outer](19.91,0.57165176)(19.16,0.42165175)
    \psframe[linecolor=black, linewidth=0.02, dimen=outer](19.91,0.41165173)(19.16,0.26165175)
    \psframe[linecolor=black, linewidth=0.02, dimen=outer](20.71,0.57165176)(19.96,0.42165175)
    \psframe[linecolor=black, linewidth=0.02, dimen=outer](20.71,0.41165173)(19.96,0.26165175)
    \psframe[linecolor=black, linewidth=0.02, dimen=outer](19.91,0.25165173)(19.16,0.10165174)
    \psframe[linecolor=black, linewidth=0.02, dimen=outer](20.71,0.25165173)(19.96,0.10165174)
    \psframe[linecolor=black, linewidth=0.02, dimen=outer](19.91,0.091651745)(19.16,-0.058348253)
    \psframe[linecolor=black, linewidth=0.02, dimen=outer](20.71,0.091651745)(19.96,-0.058348253)
    \psframe[linecolor=black, linewidth=0.02, dimen=outer](18.31,0.57165176)(17.56,0.42165175)
    \psframe[linecolor=black, linewidth=0.02, dimen=outer](18.31,0.41165173)(17.56,0.26165175)
    \psframe[linecolor=black, linewidth=0.02, dimen=outer](19.11,0.57165176)(18.36,0.42165175)
    \psframe[linecolor=black, linewidth=0.02, dimen=outer](19.11,0.41165173)(18.36,0.26165175)
    \psframe[linecolor=black, linewidth=0.02, dimen=outer](18.31,0.25165173)(17.56,0.10165174)
    \psframe[linecolor=black, linewidth=0.02, dimen=outer](19.11,0.25165173)(18.36,0.10165174)
    \psframe[linecolor=black, linewidth=0.02, dimen=outer](18.31,0.091651745)(17.56,-0.058348253)
    \psframe[linecolor=black, linewidth=0.02, dimen=outer](19.11,0.091651745)(18.36,-0.058348253)
  \end{pspicture}
%}
%


\end{frame}
%------------------------------------------------

\begin{frame}
\frametitle{Objetivo}
\framesubtitle{Ventajas}
\justifying

\begin{itemize}
\justifying
  \item Servicio de almacenamiento \textit{on-premises}
\\~\\
  \item Utiliza un protocolo est\'{a}ndar
\\~\\
  \item Compatibilidad multiplataforma
\\~\\
  \item Control de acceso para lectura y escritura
\\~\\
  \item Soporta la descarga simultanea de archivos
\end{itemize}

\end{frame}
%------------------------------------------------

\begin{frame}
\frametitle{Desarrollo}
\framesubtitle{Componentes}
\justifying

\begin{itemize}
\justifying
%  \item Debian GNU/Linux como sistema operativo del \textit{appliance}.
  \item Sistema operativo Debian GNU/Linux 7 \guillemotleft wheezy\guillemotright
\\~\\
%  \item OpenSSH para el acceso remoto por linea de comandos.
  \item Servicio SSH (OpenSSH)
\\~\\
%  \item Apache HTTPD para proveer el servicio de HTTPS.
  \item Servicio HTTP (Apache HTTPD)
\\~\\
%  \item OpenLDAP para implementar el directorio de usuarios.
  \item Servicio LDAP (OpenLDAP)
\end{itemize}

\end{frame}
%------------------------------------------------

\begin{frame}
\frametitle{Desarrollo}
\framesubtitle{Diagrama de bloques}
\centering
 \psscalebox{0.9 0.9} % Change this value to rescale the drawing.
 {
  % \psscalebox{1.0 1.0} % Change this value to rescale the drawing.
% {
  \begin{pspicture}(0,-2.2364695)(12.5903225,2.2364695)
  \psframe[linecolor=black, linewidth=0.04, dimen=outer](7.7580647,0.10707894)(4.819355,-0.99614686)
  \psframe[linecolor=black, linewidth=0.04, dimen=outer](9.469356,1.4151434)(8.308065,0.7699821)
  \psframe[linecolor=black, linewidth=0.02, dimen=outer](12.5903225,1.7796595)(4.616129,-2.2364695)
  \psframe[linecolor=black, linewidth=0.04, dimen=outer](12.358065,1.5441756)(10.219355,-0.9590501)
  \psframe[linecolor=black, linewidth=0.04, dimen=outer](7.7580647,1.5441756)(4.819355,0.34094986)
  \psframe[linecolor=black, linewidth=0.04, dimen=outer](10.058064,-1.0558243)(7.919355,-1.9590502)
  \psline[linecolor=black, linewidth=0.04, arrowsize=0.05291666666666667cm 2.0,arrowlength=1.4,arrowinset=0.0]{->}(7.732975,1.0420251)(8.321864,1.0420251)
  \psline[linecolor=black, linewidth=0.04, arrowsize=0.05291666666666667cm 2.0,arrowlength=1.4,arrowinset=0.0]{->}(9.455197,1.0420251)(10.232975,1.0420251)
  \psline[linecolor=black, linewidth=0.04, arrowsize=0.05291666666666667cm 2.0,arrowlength=1.4,arrowinset=0.0]{->}(7.732975,-0.49130818)(10.232975,-0.49130818)
  \psline[linecolor=black, linewidth=0.04, arrowsize=0.05291666666666667cm 2.0,arrowlength=1.4,arrowinset=0.0]{->}(6.2663083,-0.95797485)(6.2663083,-1.5801971)(7.9551973,-1.5801971)
  \psline[linecolor=black, linewidth=0.04, arrowsize=0.05291666666666667cm 2.0,arrowlength=1.4,arrowinset=0.0]{->}(10.044086,-1.5135304)(10.655197,-1.5135304)
  \psline[linecolor=black, linewidth=0.04, arrowsize=0.05291666666666667cm 2.0,arrowlength=1.4,arrowinset=0.0]{->}(2.4419355,1.0861112)(4.6354837,1.0861112)
  \psline[linecolor=black, linewidth=0.04, arrowsize=0.05291666666666667cm 2.0,arrowlength=1.4,arrowinset=0.0]{->}(2.4741936,-0.10743722)(4.6354837,-0.13969529)
  \rput[bl](0.57,0.97256273){Cliente SSH}
  \rput[bl](5.0287094,-0.46243727){Apache HTTPD}
  \rput[bl](3.4548387,1.3175627){22/tcp}
  \rput[bl](3.2748386,0.080465995){443/tcp}
  \rput[bl](5.64871,0.94256276){OpenSSH}
  \rput[bl](8.18871,-1.6274372){\textsc{WebDAV}}
  \rput[bl](8.55871,0.9875627){pam}
  \rput[bl](10.38371,0.34256268){OpenLDAP}
  \rput[bl](10.674839,-1.4493726){Archivos}
  \rput[bl](10.654839,-1.9074372){Carpetas}
  \rput[bl](0.2,0.022562772){Cliente nativo}
  \rput[bl](0.0,-0.5516308){Navegador web}
  \rput[bl](6.8282256,1.88646960){Debian \textsc{GNU/Linux} 7}
  \rput[bl](5.3037096,0.47535850){\textit{\scriptsize Administraci\'{o}n}}
  \rput[bl](4.9937096,-0.8246415){\textit{\scriptsize Acceso de usuarios}}
  \rput[bl](10.40871,-0.12464152){\textit{\scriptsize Autenticaci\'{o}n}}
  \rput[bl](10.53871,-0.42464152){\textit{\scriptsize centralizada}}
  \end{pspicture}
% }

 }

\end{frame}
%------------------------------------------------

\begin{frame}
\frametitle{Introdution}
\framesubtitle{Vehicules Solaires}

\end{frame}

%------------------------------------------------

\begin{frame}
\frametitle{Pruebas}
\framesubtitle{Historique}
"Si tous les avions devenaient solaires, ce serait formidable." - Yann-Arthus Bertrand

\begin{figure}
%\includegraphics[width=0.7\textwidth]{images/solar-impulse.png}
\end{figure}

\end{frame}


%------------------------------------------------

\begin{frame}
\frametitle{Resultados}
\framesubtitle{Historique}
\begin{block}{Juillet 2010}
L'aventure Solar Impulse commence avec un 1er avion prototype immatricule HB-SIA. 
\end{block}

\begin{block}{Juillet 2011}
Mise en chantier du 2eme avion immatricule HB-SIB. Solar Impulse 2 devra pouvoir voler plus de cent-vingt heures d'affilee, cinq jours et cinq nuits, le temps dont il a besoin pour traverser le Pacifique ou l'Atlantique.
\end{block}

\begin{block}{Juin 2014}
1er essaie de vol avec succes pour le HB-SIB.
\end{block}


\begin{block}{Avril-Juillet 2015}
Tour du monde prevu en 2015 au depart d'Ahbu Dahbi pour une periode de 5 mois avec des vols prevu jusqu'a 5 jours sans escale. 
\end{block}

\end{frame}


%-------------------------------------------------------------

\begin{frame}
\frametitle{Solar Impulse}
\framesubtitle{Comparaison HB-SIA/HB-SIB}

\begin{columns}[c] % The "c" option specifies centered vertical alignment while the "t" option is used for top vertical alignment

\column{.45\textwidth} % Left column and width
\textbf{SIA}
\begin{enumerate}
\item Envergure: 63.4m
\item Taille du Cockpit: 1.6m3
\item Poids: 1.3 tonnes
\end{enumerate}

\column{.5\textwidth} % Right column and width
\textbf{SIB}
\begin{enumerate}
\item Envergure: 72m
\item Taille du Cockpit: 3.8m3
\item Poids: 2.3 tonnes
\end{enumerate}

\end{columns}

L'avion HB-SIB:
\begin{itemize}
\item Charge utile sera augmentee
\item Circuits electriques rendus etanches pour pouvoir voler sous la pluie
\item Systemes redondants ameliorent la fiabilite
\item Bayer Material Sciences fait profiter le projet de ses nanotechnologies 
\item Decision utilise de fibres de carbones d'une legerete jamais vues jusqu'a aujourd'hui.
\end{itemize}
\end{frame}


%------------------------------------------------


\begin{frame}
\frametitle{Solar Impulse}
\framesubtitle{Les caracteristiques et le  « fonctionnement » du Solar Impulse SIB}

Le projet Solar Impulse totalise aujourd'hui plus de 12 annees d'etudes de faisabilite, de conception, design et construction. Le Solar Impulse :\\

\begin{itemize}
\item est compose de 80 ingenieurs et techniciens de Solar Impulse
\item 80 partenaires technologies
\item plus de 100 conseillers et fournisseurs
\item investissement d'environ 100 millions d'euros


\end{itemize}

\end{frame}

%------------------------------------------------
\begin{frame}
\frametitle{Solar Impulse}
\framesubtitle{Principe du Vol}




\end{frame}

%------------------------------------------------

\begin{frame}
\frametitle{Solar Impulse}
\framesubtitle{Propositions d'amellioration}




\end{frame}
%------------------------------------------------

\begin{frame}
\frametitle{Conclusion}





\end{frame}
%------------------------------------------------

\begin{frame}
\frametitle{Sources}





\end{frame}
%------------------------------------------------
\begin{frame}
\frametitle{Table}
\begin{table}
\begin{tabular}{l l l}
\toprule
\textbf{Treatments} & \textbf{Response 1} & \textbf{Response 2}\\
\midrule
Treatment 1 & 0.0003262 & 0.562 \\
Treatment 2 & 0.0015681 & 0.910 \\
Treatment 3 & 0.0009271 & 0.296 \\
\bottomrule
\end{tabular}
\caption{Table caption}
\end{table}
\end{frame}

%------------------------------------------------

\begin{frame}
\frametitle{Theorem}
\begin{theorem}[Mass--energy equivalence]
$E = mc^2$
\end{theorem}
\end{frame}

%------------------------------------------------

\begin{frame}[fragile] % Need to use the fragile option when verbatim is used in the slide
\frametitle{Verbatim}
\begin{example}[Theorem Slide Code]
\begin{verbatim}
\begin{frame}
\frametitle{Theorem}
\begin{theorem}[Mass--energy equivalence]
$E = mc^2$
\end{theorem}
\end{frame}\end{verbatim}
\end{example}
\end{frame}

%------------------------------------------------

\begin{frame}
\frametitle{Figure}
Uncomment the code on this slide to include your own image from the same directory as the template .TeX file.
%\begin{figure}
%\includegraphics[width=0.8\linewidth]{test}
%\end{figure}
\end{frame}

%------------------------------------------------

\begin{frame}[fragile] % Need to use the fragile option when verbatim is used in the slide
\frametitle{Citation}
An example of the \verb|\cite| command to cite within the presentation:\\~

This statement requires citation \cite{p1}.
\end{frame}

%------------------------------------------------

\begin{frame}
\frametitle{References}
\footnotesize{
\begin{thebibliography}{99} % Beamer does not support BibTeX so references must be inserted manually as below
\bibitem[Smith, 2012]{p1} John Smith (2012)
\newblock Title of the publication
\newblock \emph{Journal Name} 12(3), 45 -- 678.
\end{thebibliography}
}
\end{frame}

%------------------------------------------------

\begin{frame}
\Huge{\centerline{Gracias}}
\end{frame}

%----------------------------------------------------------------------------------------

\end{document}
