\restylefloat{table}
  \chapter{Marco Te\'{o}rico}
  \label{chap:cap1}
    \section {Sistemas de almacenamiento}
    
Definici\'{o}n

Los sistemas de almacenamiento son importantes puesto que guardan los datos del usuario.

%Caracter\'{i}sticas de los sistemas de almacenamiento
%
%\begin{itemize}
%  \item Persistencia de la informaci\'{o}n
%  \item Capacidad Multi-escritura
%\end{itemize}

\subsection {Clasificaci\'{o}n por tipo de medio}

%\textbf{\\ Medios F\'{i}sicos \\}
%\begin{itemize}
%  \item Tarjetas Perforadas
%
%  Caracteristicas
%  Componentes
%  Metodo de operacion
%    Lectura
%    Escritura
%  Similitud con otros medios de almacenamiento
%
%\end{itemize}

\textbf{\\ Medios basados en circuitos \\}

Descripci\'{o}n de los medios, caracter\'{i}sticas, componentes, m\'{e}todo de operaci\'{o}n, similitudes o aplicaciones y capacidad actual.

\begin{itemize}
  \item RAM
  
  Es un medio de almacenamiento vol\'{a}til que se utiliza para guardar datos temporales o como memoria intermedia de un equipo de c\'{o}mputo. Se compone de arreglos de circuitos y tiene una gran velocidad de lectura y de escritura comparada con cualquier otro medio de almacenamiento.
  
  \item NVRAM
  
  Es una variante de la memoria RAM donde se almacena la informaci\'{o}n de manera no vol\'{a}til, generalmente su capacidad es peque\~{n}a y este tipo de memoria se utiliza para fines muy especializados, por ejemplo para guardar las configuraciones de algunos sistemas embebidos.
  
  \item ROM
  
  Es una memoria de s\'{o}lo lectura que puede ser grabado una sola vez, generalmente se utilizaba para almacenar firmware de dispositivos o alg\'{u}n programa embebido en hardware que requiere una salida fija para una entrada determinada.
  
  \item EEPROM
  
  Una variante de la memoria ROM donde el contenido puede ser borrado por varios m\'{e}todos, ya sea el\'{e}ctricamente o por medio de la exposici\'{o}n a rayos ultravioleta. En condiciones normales de operaci\'{o}n su comportamiento es similar al de la memoria ROM y se debe entrar en un modo especial para grabar nuevos datos en la misma.
  
  \item FLASH
  
  Una variante m\'{a}s de los medios de almacenamiento basados en circuitos es este tipo de memoria se ha hecho popular en los \'{u}ltimos a\~{n}os gracias a que no depende de partes m\'{o}viles y es de tama\~{n}o peque\~{n}o por lo que es un medio de almacenamiento port\'{a}til y eficiente.
  
  \item SSD
  
  Un medio de almacenamiento relativamente reciente utiliza arreglos de circuitos para guardar la informaci\'{o}n, generalmente se utiliza memoria tipo NAND para mantener los datos y a diferencia de los discos duros no tienen partes m\'{o}viles por lo que son menos propensos a fallos y disipan menos calor.
  
\end{itemize}

\textbf{\\ Medios Magn\'{e}ticos \\}

Este tipo de medios se caracterizan porque el acceso a los datos se realiza mediante un cabezal que lee o escribe el campo magn\'{e}tico impreso en el material que lo guarda. Este tipo de medios es susceptible a fallar si se le expone a campos magn\'{e}ticos, golpes o temperaturas extremas.

\begin{itemize}
  \item Cinta
  
  Las unidades de cinta son medios que almacenan de manera secuencial los datos, por lo que no son de acceso aleatorio. Dentro de sus componentes internos destacan dos carretes que sirven para almacenar la cinta mientras se lee, para acceder datos en una posici\'{o}n anterior es necesario rebobinar la cinta.
  
  Generalmente tienen capacidades que oscilan entre los GB y TB  \cite{b306cc575a86d1e84e6ba100dcfb4417} y son utilizadas para archivar informaci\'{o}n.
  
  \item Discos Flexibles
  
  Los discos flexibles fueron muy populares en las \'{u}ltimas tres d\'{e}cadas del siglo pasado, sus capacidades eran peque\~{n}as y los datos se pueden acceder de manera aleatoria gracias a que cuenta con cabezales para acceder a cualquier sector del disco de manera no secuencial.
  
  \item Discos Duros
  
  Parecidos a los discos flexibles, estos medios tienen mayor capacidad y actualmente son el medio primordial de almacenar informaci\'{o}n en los equipos de c\'{o}mputo. A diferencia de los discos flexibles, tienen los componentes mec\'{a}nicos dentro del armaz\'{o}n del disco y pueden almacenar grandes cantidades de informaci\'{o}n gracias a que se apilan varios discos en una estructura cilindrica. Para buscar un dato se debe hacer referencia al plato donde est\'{a}, al sector del disco y a la cabeza de lectura que corresponda.
  
\end{itemize}

\textbf{\\ Medios \'{O}pticos \\}

Descripci\'{o}n de los medios, caracter\'{i}sticas, componentes, m\'{e}todo de operaci\'{o}n, similitudes o aplicaciones y capacidad actual.

\begin{itemize}
  \item Discos de s\'{o}lo lectura (WORM)
  
  Los discos pre-masterizados (CD-ROM, DVD-ROM, BD-ROM) se graban en las f\'{a}bricas donde se tiene un disco maestro que sirve para transferir los datos al medio final por un proceso de vaciado t\'{e}rmico.
  
  Los discos grabables (CD-R, DVD+R, DVD-R y BD-R) pueden ser grabados mediante un laser al fundir una capa de policarbonato en la superficie inferior del disco para grabar los bits en 1. El formato de los bits generalmente va de acuerdo al est\'{a}ndar ISO-9660.
  
  \item Discos regrabables
  
  Los discos regrabables (CD-RW, DVD+RW, DVD-RW, BD-RE) tienen una ventaja adicional comparados con los discos grabables de una sola vez, es posible borrar la informaci\'{o}n contenida para almacenar nuevos datos en el medio, esto se logra inicializar parcial o totalmente los sectores del disco para que este pueda admitir nuevos datos.
  
\end{itemize}

\textbf{\\ Medios Magneto-\'{o}pticos \\}

\begin{itemize}
  \item Discos MO (Magneto-optic)
  
  Los discos magneto-\'{o}pticos tienen las bondades de la rapidez de los discos magn\'{e}ticos y la versatilidad de los discos \'{o}pticos, 
  
\end{itemize}

\textbf{\\ Medios Hologr\'{a}ficos \\}

\begin{itemize}
  \item HVD - Disco Vers\'{a}til Hologr\'{a}fico
  
  Es un nuevo medio de almacenamiento a\'{u}n en desarrollo que ofrece un m\'{e}todo m\'{a}s vers\'{a}til de escribir los datos utilizando un mecanismo hologr\'{a}fico donde dependiendo la manera en la que se leen los datos es la informaci\'{o}n que se obtiene.
  
\end{itemize}
      \subsection {Comparativa de medios de almacenamiento actuales}

\begin{table}[H]
\caption{Comparativa de medios de almacenamiento}{}
\label{tab:comparativa}
\noindent\makebox[\textwidth]{%
% manually center the table in page
\hspace*{-1.1cm}
\begin{tabular}{|c|c|c|c|c|c|}
\hline
%Medio & Capacidad & Persistente & Acceso aleatorio & Lectura y escritura & Vulnerabilidades \\
\multirow{2}{*}{Medio} & \multirow{2}{*}{Capacidad} & \multirow{2}{*}{Persistente} & Acceso & Lectura y & \multirow{2}{*}{Vulnerabilidades} \\
 & & & aleatorio & escritura & \\
\hline
RAM & MB & No & Si & Si & \multirow{6}{*}{Electricidad est\'{a}tica} \\
NVRAM & MB & Si & Si & Si &  \\
ROM & MB & Si & Si & No &  \\
EEPROM & MB & Si & Si & No &  \\
FLASH & GB & Si & Si & Si &  \\
SSD & GB & Si & Si & Si &  \\
\hline
Cinta & TB & Si & No & Si & \multirow{3}{*}{Campos magn\'{e}ticos} \\
Disco Flexible & MB & Si & Si & Si &  \\
Disco Duro & TB & Si & Si & Si &  \\
\hline
Disco \'{O}ptico & MB/GB & Si & Si & No & \multirow{2}{*}{Rayaduras} \\
Disco Regrabable & MB/GB & SI & SI & Si &  \\
\hline
\multirow{2}{*}{Disco MO} & \multirow{2}{*}{MB/GB} & \multirow{2}{*}{Si} & \multirow{2}{*}{Si} & \multirow{2}{*}{Si} & Rayaduras \\
 & & & & & Campos magn\'{e}ticos \\
\hline
Disco Hologr\'{a}fico & GB & Si & Si & Si & Rayaduras \\
\hline
\end{tabular}
} % ending of \makebox
\end{table}

      \subsection {Escenarios de falla}

Da\~{n}o f\'{i}sico del medio

Fallo de componentes internos

% Componentes mec\'{a}nicos

% Circuitos

      \subsection {M\'{e}todos de protecci\'{o}n}

Protecci\'{o}n antiest\'{a}tica

Seguro contra escritura

Protecci\'{o}n contra da\~{n}o f\'{i}sico

Protecci\'{o}n contra golpes

Protecci\'{o}n contra campos magn\'{e}ticos

        \subsubsection {T\'{e}cnicas de respaldo}

Definici\'{o}n de respaldo

Copia a otros discos

Copia a medios de solo lectura

Copia de archivos a recursos de red compartidos (NFS, CIFS, SSHFS)

Montaje de dispositivos de bloque por red (NBD, SAN)

        \subsubsection {Arreglos RAID}

Definici\'{o}n de arreglo de discos (f\'{i}sico y l\'{o}gico)

Tipo de arreglos RAID

0, 1, 5, 6, 01, 10

Ventajas respecto al almacenamiento en un s\'{o}lo disco

    \section {Appliances}

Definici\'{o}n de appliance

    \section {Seguridad Inform\'{a}tica}
    
Definici\'{o}n de seguridad inform\'{a}tica

      \subsection {Principios de seguridad inform\'{a}tica}

Confidencialidad

Integridad

Disponibilidad

      \subsection {Vulnerabilidades}

Definici\'{o}n de vulnerabilidad

      \subsection {Hardening}

Definici\'{o}n de Hardening

Importancia del hardening para mitigar las vulnerabilidades

    \section {GNU/Linux}
      \subsection {Historia}
      \subsection {Uso de GNU/Linux en la industria}
      \subsection {Distribuciones de GNU/Linux}
      \subsection {Debian GNU/Linux}
    \section {Protocolo HTTP}
      \subsection {HTTPS - HTTP over SSL}
      \subsection {WebDAV}
    \section {Protocolo LDAP}
      \subsection {Directorio de usuarios}
    \section {Protocolo SSH}
      \subsection {SCP - Secure Copy}
      \subsection {SFTP - Secure FTP}
      \subsection {SSHFS - Secure Shell Filesystem}
