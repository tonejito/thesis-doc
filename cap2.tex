{
  \linespread{1}
  \cleardoublepage  
  \chapter{Definici\'{o}n del problema y soluci\'{o}n propuesta}
  \label{chap:cap2}
}

    \section {Problem\'{a}tica actual}

Los usuarios de la Divisi\'{o}n de Ingenier\'{i}as Civil y Geom\'{a}tica utilizan memorias \textsc{USB}, discos \'{o}pticos y carpetas compartidas en red para acceder a archivos de uso interno como documentos, presentaciones e im\'{a}genes que ocupan para realizar sus actividades.

Adicionalmente el personal de la Unidad de C\'{o}mputo almacena los archivos de instalaci\'{o}n de programas y controladores en dispositivos externos como memorias \textsc{USB}, discos duros externos \'{o} en discos \'{o}pticos y los utiliza al dar mantenimiento a los equipos de c\'{o}mputo de la divisi\'{o}n.

El problema al mantener los archivos en carpetas compartidas o utilizar un medio de almacenamiento externo es que los datos dejan de estar disponibles si el equipo que comparte la carpeta se encuentra apagado o si el medio de almacenamiento no est\'{a} conectado al equipo donde se procesa la informaci\'{o}n.

Por ejemplo, si un profesor necesita compartir algunos archivos con los alumnos parar utilizarlos en clase copia a una memoria \textsc{USB} los archivos y la pasa a cada uno de los estudiantes para que copien los archivos a su equipo lo cual, adem\'{a}s de que implica tiempo, es un proceso propenso a errores donde se puede da\~{n}ar la memoria flash que circula entre los equipos o se puede infectar con \textit{malware} \footnote{\textit{Malware}: software malicioso} en un equipo y al pasar de equipo a equipo propagar la infecci\'{o}n.

Otro caso muy com\'{u}n es el env\'{i}o de archivos modificados por correo electr\'{o}nico, generando as\'{i} una lista de versiones diferentes del mismo archivo que causan confusi\'{o}n porque se desconoce cual es la versi\'{o}n m\'{a}s actualizada.

    \section {Soluci\'{o}n propuesta}

Para solventar este problema se propone instalar un servidor de almacenamiento que provea espacio para que los usuarios de la \textsc{DICyG} puedan guardar su informaci\'{o}n y accederla desde los equipos conectados a la red interna a trav\'{e}s de una conexi\'{o}n cifrada.

La implementaci\'{o}n del servidor de almacenamiento ayudar\'{a} a liberar espacio en los equipos de c\'{o}mputo de la divisi\'{o}n que se utilizan para compartir archivos y proporcionar\'{a} un repositorio central que guarde la informaci\'{o}n y la haga accesible a los usuarios sin depender de un equipo externo.

% Usuarios de la Divisi\'{o}n de Ingenier\'{i}a Civil y Geom\'{a}tica (Depto de Geomatica) y Unidad de Computo
% 50 usuarios
% Almacenamiento centralizado para guardar imagenes o documentos y poderlos visualizar o editar en cualquier equipo conectado a la red de la division (archivos ligeros)
% Almacenamiento de archivos de intalacion para uso de la unidad de computo (archivos pesados)
% % Alcance
% Implementaci\'{o}n, hardening, interfaz de cambio de contrase\~{n}a, pruebas de seguridad y de compatibilidad
% % Justificacion
% proveer el servicio de almacenamiento centralizado a los usuarios de la dicyg
% reglas de acceso para las redes, 
% la transmision de datos esta cifrada, los datos no se cifran en el servidor
% disponibilidad de los datos en los horarios establecidos
% liberar espacio y evitar el uso de medios externos para transmitir la informacion
% poderlos visualizar o editar en cualquier equipo conectado a la red de la division

    \section {Tecnolog\'{i}as a utilizar}

Para la implementaci\'{o}n de este proyecto se utilizar\'{a}n las siguientes tecnolog\'{i}as:

\begin{itemize}
  \item Debian GNU/Linux como sistema operativo del \textit{appliance}.
  \item Apache HTTPD para proveer el servicio de HTTPS.
  \item OpenLDAP para implementar el directorio de usuarios.
  \item OpenSSH para el acceso remoto por linea de comandos.
\end{itemize}

    \section {Arquitectura del prototipo}

El prototipo ser\'{a} implementado en un servidor proporcionado por la \textsc{DICyG} que almacenar\'{a} los archivos que los profesores utilizan en sus clases, as\'{i} como los programas de instalaci\'{o}n y software de controladores de dispositivo que la Unidad de C\'{o}mputo utiliza para dar mantenimiento a los equipos de la divisi\'{o}n.

      \subsection {Diagrama funcional}

\diagramblock
{Diagrama funcional de la soluci\'{o}n propuesta.}
{DiagramaFuncional}
{
 \psscalebox{1.0 1.0} % Change this value to rescale the drawing.
 {
  \begin{pspicture}(0,-2.2364695)(12.5903225,2.2364695)
  \rput[bl](0.57,0.97256273){Cliente SSH}
  \rput[bl](3.5548387,1.3175627){22/tcp}
  \rput[bl](3.3748386,0.080465995){443/tcp}
  \rput[bl](5.54871,0.94256276){OpenSSH}
  \psframe[linecolor=black, linewidth=0.04, dimen=outer](7.7580647,0.10707894)(4.819355,-0.99614686)
  \rput[bl](5.0287094,-0.46243727){Apache
  HTTPD}
  \rput[bl](8.18871,-1.6274372){\textsc{WebDAV}}
  \rput[bl](8.55871,0.9875627){pam}
  \psframe[linecolor=black, linewidth=0.04, dimen=outer](9.469356,1.4151434)(8.308065,0.7699821)
  \rput[bl](10.38371,0.34256268){OpenLDAP}
  \rput[bl](10.674839,-1.4493726){Archivos}
  \rput[bl](10.654839,-1.9074372){Carpetas}
  \rput[bl](0.2,0.022562772){Cliente Nativo}
  \rput[bl](0.0,-0.5516308){Navegador Web}
  \psframe[linecolor=black, linewidth=0.02, dimen=outer](12.5903225,1.7796595)(4.616129,-2.2364695)
  \psframe[linecolor=black, linewidth=0.04, dimen=outer](12.358065,1.5441756)(10.219355,-0.9590501)
  \psframe[linecolor=black, linewidth=0.04, dimen=outer](7.7580647,1.5441756)(4.819355,0.34094986)
  \psframe[linecolor=black, linewidth=0.04, dimen=outer](10.058064,-1.0558243)(7.919355,-1.9590502)
  \psline[linecolor=black, linewidth=0.04, arrowsize=0.05291666666666667cm 2.0,arrowlength=1.4,arrowinset=0.0]{->}(7.732975,1.0420251)(8.321864,1.0420251)
  \psline[linecolor=black, linewidth=0.04, arrowsize=0.05291666666666667cm 2.0,arrowlength=1.4,arrowinset=0.0]{->}(9.455197,1.0420251)(10.232975,1.0420251)
  \psline[linecolor=black, linewidth=0.04, arrowsize=0.05291666666666667cm 2.0,arrowlength=1.4,arrowinset=0.0]{->}(7.732975,-0.49130818)(10.232975,-0.49130818)
  \psline[linecolor=black, linewidth=0.04, arrowsize=0.05291666666666667cm 2.0,arrowlength=1.4,arrowinset=0.0]{->}(6.2663083,-0.95797485)(6.2663083,-1.5801971)(7.9551973,-1.5801971)
  \psline[linecolor=black, linewidth=0.04, arrowsize=0.05291666666666667cm 2.0,arrowlength=1.4,arrowinset=0.0]{->}(10.044086,-1.5135304)(10.655197,-1.5135304)
  \psline[linecolor=black, linewidth=0.04, arrowsize=0.05291666666666667cm 2.0,arrowlength=1.4,arrowinset=0.0]{->}(2.4419355,1.0861112)(4.6354837,1.0861112)
  \psline[linecolor=black, linewidth=0.04, arrowsize=0.05291666666666667cm 2.0,arrowlength=1.4,arrowinset=0.0]{->}(2.4741936,-0.10743722)(4.6354837,-0.13969529)
  \rput[bl](6.8282256,1.88646960){Debian \textsc{GNU/Linux} 7}
  \rput[bl](5.2037096,0.47535850){\textit{\scriptsize Administraci\'{o}n}}
  \rput[bl](4.9537096,-0.8246415){\textit{\scriptsize Acceso de usuarios}}
  \rput[bl](10.30871,-0.12464152){\textit{\scriptsize Autenticaci\'{o}n}}
  \rput[bl](10.43871,-0.42464152){\textit{\scriptsize centralizada}}
  \end{pspicture}
 }
}

      \subsection {Autenticaci\'{o}n centralizada}

El t\'{e}rmino \textit{autenticaci\'{o}n centralizada} se refiere a contar con un repositorio central de usuarios que permita que las aplicaciones puedan verificar las credenciales de acceso \textit{autenticando} as\'{i} al usuario.

        \subsubsection {Autenticaci\'{o}n por medio de directorio}

Para tener un esquema de autenticaci\'{o}n centralizada se tendr\'{a} un directorio de usuarios que implemente el protocolo LDAP mediante el software \textsc{OpenLDAP}.

        \subsubsection {Estructura del directorio}

La estructura propuesta comprende varios contenedores que sirven para separar los tipos de objeto que forman parte del directorio, cada contenedor est\'{a} representado por una \textit{unidad organizacional} cuyo atributo \'{u}nico es el nombre \textbf{ou}. Los contenedores principales en la ra\'{i}z del directorio son los siguientes:

\begin{itemize}
  \item Contenedor de usuarios
  \item Contenedor de grupos
  \item Contenedor de materias
  \item Contenedor de cuentas de servicio
\end{itemize}

\textbf{\\ Contenedor de usuarios \\}

Clasifica los usuarios del sistema por tipo, cada usuario tiene un identificador \'{u}nico asignado llamado \textbf{uid} se subdivide en tres ramas:

\begin{itemize}

  \item \textbf{Personal de la \textit{Unidad de C\'{o}mputo} de la divisi\'{o}n}

  Estas cuentas de usuario son objetos de tipo \textit{posixAccount} y representan cuentas est\'{a}ndar de \textsc{UNIX}.

  \item \textbf{Profesores}

  Se almacenan en el directorio como objetos de tipo \textit{posixAccount} que representan cuentas est\'{a}ndar de \textsc{UNIX}.

  \item \textbf{Alumnos}

  Su representaci\'{o}n en el directorio es un objeto de tipo \textit{simpleSecurityObject} que se utiliza para asignar un usuario y contrase\~{n}a \'{u}nicamente.

\end{itemize}

\textbf{\\ Contenedor de grupos \\}

Existen tres clases de grupos de usuarios contemplados en el sistema, esto ayuda a permitir o negar el acceso a los recursos del servidor, un usuario puede pertenecer a uno o m\'{a}s grupos que se listan a continuaci\'{o}n.

\begin{itemize}

  \item \textbf{Usuarios de la \textsl{Unidad de C\'{o}mputo} de la divisi\'{o}n}

  Existe un grupo \'{u}nico que contiene a todos los usuarios pertenecientes a la \textit{Unidad de C\'{o}mputo} y se almacena como un objeto de tipo \textit{posixGroup} que representa un grupo est\'{a}ndar de usuarios \textsc{UNIX}.

  \item \textbf{Profesores}

  Existe un grupo individual para cada usuario y se almacena como \textit{posixGroup}.

  \item \textbf{Alumnos}

  Existen varios grupos que contienen al profesor, la materia y el grupo en el que se imparte, permiten dar acceso a la carpeta donde se almacenan los archivos de un grupo en particular se almacena internamente como un objeto de tipo \textit{groupOfNames}.

\end{itemize}

\textbf{\\ Contenedor de materias \\}

Permite almacenar el cat\'{a}logo de las materias que se imparten en la divisi\'{o}n. Se requiere para la asignaci\'{o}n de grupos a los profesores y alumnos.

\textbf{\\ Contenedor de cuentas de servicio \\}

Contiene cuentas de usuario asociadas a servicios de sistema, se utiliza para almacenar las cuentas que realizan b\'{u}squedas en el directorio.

\textbf{\\ Diagrama del \'{a}rbol de directorio \\}
% Works with `pdflatex', not compatible with `latex'
%\pdfannot{/Subtype/Text/Contents(TEXT)}

El siguiente diagrama muestra la estructura del \'{a}rbol del directorio \textsc{LDAP}.

\diagramblock
{Diagrama del \'{a}rbol de directorio}
{LDAP-Tree}
{
 \psscalebox{1.0 1.0} % Change this value to rescale the drawing.
 {
  \begin{pspicture}(0,-2.6826556)(11.01282,2.6826556)
  \rput[bl](4.2,2.3726556){dc=xnas,dc=local}
  \rput[bl](2.4692307,1.5392582){cn=admin}
  \rput[bl](0.0,0.36118132){ou=users}
  \rput[bl](2.9615386,0.33118132){ou=groups}
  \rput[bl](6.6025643,0.32118133){ou=materias}
  \rput[bl](9.002564,0.31092492){ou=services}
  \rput[bl](6.951282,0.053296704){\small cn=1024}
  \rput[bl](9.312820,-0.28728938){\small cn=...}
  \rput[bl](3.3,0.027783882){\small ou=unix}
  \rput[bl](0.3,-0.94675830){\small ou=profesores}
  \rput[bl](0.3,-2.68265560){\small ou=grupos}
  \rput[bl](0.3,0.048296705){\small ou=staff}
  \rput[bl](3.3,-0.83996340){\small ou=webdav}
  \rput[bl](0.3,-1.77789380){\small ou=alumnos}
  \rput[bl](6.951282,-0.23600733){\small cn=...}
  \rput[bl](9.31282,-0.037985347){\small cn=apache2}
  \rput[bl](0.6014652,-1.1713004){\footnotesize cn=AETA820507}
  \rput[bl](0.6014652,-2.0610440){\footnotesize cn=302105966}
  \rput[bl](3.596337,-0.26600733){\footnotesize cn=AETA820507}
  \rput[bl](3.596337,-1.17130040){\footnotesize cn=1024-AETA820507}
  \rput[bl](0.6014652,-0.2660073){\footnotesize uid=andres}
  \rput[bl](0.6014652,-0.5544506){\footnotesize uid=...}
  \rput[bl](0.6014652,-1.4456594){\footnotesize cn=...}
  \rput[bl](0.6014652,-2.3097620){\footnotesize cn=...}
  \rput[bl](3.596337,-1.44565940){\footnotesize cn=...}
  \rput[bl](3.596337,-0.51945055){\footnotesize cn=...}
  \psline[linecolor=black, linewidth=0.04, arrowsize=0.0529166666666666cm 2.0,arrowlength=1.4,arrowinset=0.0]{->}(5.702817,2.361388)(5.702817,1.0092754)(0.46338028,1.0092754)(0.46338028,0.5867402)
  \psline[linecolor=black, linewidth=0.04, arrowsize=0.0529166666666666cm 2.0,arrowlength=1.4,arrowinset=0.0]{->}(5.702817,1.0092754)(9.505633,1.0092754)(9.505633,0.5867402)
  \psline[linecolor=black, linewidth=0.04, arrowsize=0.0529166666666666cm 2.0,arrowlength=1.4,arrowinset=0.0]{->}(7.1394367,1.0092754)(7.1394367,0.5867402)
  \psline[linecolor=black, linewidth=0.04, arrowsize=0.0529166666666666cm 2.0,arrowlength=1.4,arrowinset=0.0]{->}(3.4211268,1.0092754)(3.4211268,0.5867402)
  \psline[linecolor=black, linewidth=0.04, arrowsize=0.0529166666666666cm 2.0,arrowlength=1.4,arrowinset=0.0]{->}(5.702817,1.6008247)(4.18169,1.6008247)
  \psline[linecolor=black, linewidth=0.04, tbarsize=0.07055555555555555cm 5.0]{-|}(0.11276596,0.33861312)(0.11276596,0.083293974)
  \psline[linecolor=black, linewidth=0.04, tbarsize=0.07055555555555555cm 5.0]{-|}(0.11276596,0.083293974)(0.11276596,-0.85287625)
  \psline[linecolor=black, linewidth=0.04, tbarsize=0.07055555555555555cm 5.0]{-|}(0.11276596,-0.85287625)(0.11276596,-1.70394)
  \psline[linecolor=black, linewidth=0.04, tbarsize=0.07055555555555555cm 5.0]{-|}(0.11276596,-1.70394)(0.11276596,-2.555004)
  \psline[linecolor=black, linewidth=0.04, tbarsize=0.07055555555555555cm 5.0]{-|}(3.0914893,0.33861312)(3.0914893,0.083293974)
  \psline[linecolor=black, linewidth=0.04, tbarsize=0.07055555555555555cm 5.0]{-|}(3.0914893,0.083293974)(3.0914893,-0.7677699)
  \psline[linecolor=black, linewidth=0.04, tbarsize=0.07055555555555555cm 5.0]{-|}(6.751064,0.33861312)(6.751064,0.083293974)
  \psline[linecolor=black, linewidth=0.04, tbarsize=0.07055555555555555cm 5.0]{-|}(6.751064,0.083293974)(6.751064,-0.17202517)
  \psline[linecolor=black, linewidth=0.04, tbarsize=0.07055555555555555cm 5.0]{-|}(9.134043,0.25350675)(9.134043,0.083293974)
  \psline[linecolor=black, linewidth=0.04, tbarsize=0.07055555555555555cm 2.0]{-|}(0.4531915,-0.0018124074)(0.4531915,-0.25713155)
  \psline[linecolor=black, linewidth=0.04, tbarsize=0.07055555555555555cm 2.0]{-|}(0.4531915,-0.9379826)(0.4531915,-1.1933018)
  \psline[linecolor=black, linewidth=0.04, tbarsize=0.07055555555555555cm 2.0]{-|}(0.4531915,-1.1933018)(0.4531915,-1.4486209)
  \psline[linecolor=black, linewidth=0.04, tbarsize=0.07055555555555555cm 2.0]{-|}(0.4531915,-1.7890464)(0.4531915,-2.0443656)
  \psline[linecolor=black, linewidth=0.04, tbarsize=0.07055555555555555cm 2.0]{-|}(0.4531915,-2.0443656)(0.4531915,-2.2996848)
  \psline[linecolor=black, linewidth=0.04, tbarsize=0.07055555555555555cm 2.0]{-|}(0.4531915,-0.25713155)(0.4531915,-0.5124507)
  \psline[linecolor=black, linewidth=0.04, tbarsize=0.07055555555555555cm 2.0]{-|}(3.4319148,-0.0018124074)(3.4319148,-0.25713155)
  \psline[linecolor=black, linewidth=0.04, tbarsize=0.07055555555555555cm 2.0]{-|}(3.4319148,-0.25713155)(3.4319148,-0.5124507)
  \psline[linecolor=black, linewidth=0.04, tbarsize=0.07055555555555555cm 2.0]{-|}(3.4319148,-0.85287625)(3.4319148,-1.1081954)
  \psline[linecolor=black, linewidth=0.04, tbarsize=0.07055555555555555cm 2.0]{-|}(3.4319148,-1.1081954)(3.4319148,-1.4486209)
  \psline[linecolor=black, linewidth=0.04, tbarsize=0.07055555555555555cm 5.0]{-|}(9.134043,0.083293974)(9.134043,-0.25713155)
  \end{pspicture}
 }
}

      \subsection {Mecanismos de acceso a los archivos}

Se propone establecer el acceso a los archivos utilizando mecanismos como \textsc{HTTP} est\'{a}ndar y \textsc{WebDAV}.

        \subsubsection {Acceso por HTTP}

          \textbf{\\ Acceso por HTTP est\'{a}ndar \\}

Para acceder a los archivos mediante las llamadas est\'{a}ndar del protocolo HTTP s\'{o}lo se necesita que el usuario tenga un navegador web, que acceda a la \textsc{URL} iniciando sesi\'{o}n, donde podr\'{a} navegar en los directorios a los que tenga acceso y podr\'{a} descargar los archivos si tiene permisos de acceder al directorio.

          \textbf{\\ Acceso por WebDAV \\}

Para el caso del acceso via \textsc{WebDAV} es necesario un cliente, los sistemas operativos de escritorio como \textsc{GNU/Linux}, \textsc{Solaris}, \textsc{*BSD\footnote{Cualquier versi\'{o}n de BSD como \textsc{OpenBSD, FreeBSD y NetBSD}}}, \textsc{Mac OS X} y Windows tienen un cliente nativo en sus interfaces gr\'{a}ficas del navegador de archivos.

Aunque en los sistemas operativos es posible instalar clientes de \textsc{WebDAV} para acceder a los archivos, se propone tomar las interfaces nativas del sistema operativo para facilitar el acceso a los archivos.

Los sistemas operativos m\'{o}viles como \textsc{Android} y Apple \textsc{iOS} es posible instalar clientes para acceder a los archivos via \textsc{WebDAV} o si no se desea acceder por este medio se puede utilizar la interfaz web est\'{a}ndar.

%        \subsubsection {Acceso por SSH}
%
%El sistema proporcionar\'{a} acceso via \textit{Secure Shell} para el grupo de usuarios de la \textit{Unidad de C\'{o}mputo} de la Divisi\'{o}n para que hagan uso del sistema de almacenamiento en los servidores y realicen respaldos automatizados si as\'{i} lo requieren.
%
%El acceso por \textsc{SSH} permite que los usuarios puedan copiar archivos a trav\'{e}s de los siguientes mecanismos descritos en el Cap\'{i}tulo 1\footnote{Ver p\'{a}gina \pageref{Protocolo-SSH} secci\'{o}n \ref{Protocolo-SSH}}:
%
%\begin{itemize}
%  \item{\textsc{SCP} - Secure Copy}.
%  \item{\textsc{SFTP} - Secure File Transfer Protocol}.
%  \item{\textsc{SSHFS} - Secure Shell Filesystem}.
%\end{itemize}

      \subsection {Interfaces de usuario}

Adem\'{a}s del acceso por \textsc{WebDAV} y por medio de un navegador web el \textit{appliance} tendr\'{a} una interfaz de administraci\'{o}n para ver y modificar los atributos de los usuarios y una interfaz para que los usuarios puedan cambiar su contrase\~{n}a sin necesidad de acudir con el administrador.

        \subsubsection {Interfaz de administraci\'{o}n}

La interfaz de administraci\'{o}n que se propone permite ver, agregar, modificar y borrar registros del directorio. Se utilizar\'{a} la interfaz web \textit{LDAP Account Manager} y adem\'{a}s se puede instalar la herramienta \textit{Apache Directory Studio} en la workstation del administrador para realizar  estas tareas.

        \subsubsection {Interfaz de cambio de contrase\~{n}a}

Esta interfaz permite tanto al personal de la \textit{Unidad de C\'{o}mputo} como a los profesores cambiar la contrase\~{n}a de acceso que tienen asignada y al realizar esta acci\'{o}n env\'{i}a un correo para notificar que se llev\'{o} acabo dicho cambio.

    \subsection {Especificaci\'{o}n del \textit{appliance}}

En la siguiente secci\'{o}n se muestran las configuraciones del hardware y software que tendr\'{a} el \textit{appliance}.

      \section {Hardware}

{
 \begin{table}[H]
 \caption{Recursos de hardware utilizados para el \textit{appliance}}{}
 \label{tab:recursos-hardware}
 \noindent\makebox[\textwidth]
 {%
  \begin{tabular}[c]{c|c|c}
  %\hline
  \textbf{Elemento} & \textbf{M\'{i}nimo} & \textbf{Recomendado} \\
  \hline \hline
%  \textit{Servidor} & \textbf{Pruebas} & \textbf{Producci\'{o}n} \\
%  \hline
  \textit{CPU} & 1x 1 GHz & 2x 2 GHz \\
  \textit{RAM} & 1 GB & 4 GB \\
  \textit{Discos Duros} & 1x 80 GB & 2x 500 GB \\
  \textit{Arreglo \textsc{RAID}} & Sin \textsc{RAID} & \textsc{RAID} 1 \\
  %\hline
  \end{tabular}
 } % ending of \makebox
 \end{table}
}

      \section {Software}

{
 \begin{table}[H]
 \caption{Versiones de software utilizados para el \textit{appliance}}{}
 \label{tab:versiones-software}
 \noindent\makebox[\textwidth]
 {%
  \begin{tabular}[c]{c|c|c}
  %\hline
  \textbf{Software} & \textbf{Versi\'{o}n} \\
  \hline
  \textit{Sistema Operativo} & Debian GNU/Linux 7 \textit{Wheezy} \\
  \textit{OpenSSH} & v6.0 \\
  \textit{Apache httpd} & v2.2 \\
  \textit{OpenLDAP} & v2.4 \\
  \textit{PHP} & v5.4 \\
  \textit{LDAP Account Manager} & v4.8 \\
  \textit{LDAP Toolbox: Self Service Password} & v0.8 \\
  %\hline
  \end{tabular}
 } % ending of \makebox
 \end{table}
}

%      \section {L\'{i}mites}
%
%El \textit{appliance} est\'{a} configurado para alojar los archivos del staff de la \textit{unidad de c\'{o}mputo} y los archivos que cada profesor designe para la impartici\'{o}n de cada curso.
%
%{
% \begin{table}[H]
% \caption{Limites establecidos en el \textit{appliance}}{}
% \label{tab:limites-appliance}
% \noindent\makebox[\textwidth]
% {%
%  \begin{tabular}[c]{c|c|c}
%  %\hline
%  \textbf{Servidor} & \textbf{Pruebas} & \textbf{Producci\'{o}n} \\
%  \hline \hline
%  % Se ve feo, mejorar el estilo visual
%  \multirow{2}{*}{Cuotas} & Staff: []GB & Staff: []GB \\
%                          & Profesor: []GB & Profesor: []GB \\
%  \textit{Temperatura} & [] & [] \\
%  \textit{Voltaje} & []AC & []AC \\
%  %\hline
%  \end{tabular}
% } % ending of \makebox
% \end{table}
%}

