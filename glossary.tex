%       Glossary

%\newacronym[longplural=<desc>]{<label>}{<abbrv>}{<full>}

%	http://tex.stackexchange.com/a/69569
%\newglossaryentry{<label>}
%{
%  name={<appears on glossary>},
%  text={<appears on text>},
%  description={<desc>},
%  plural={<plural>}
%}

\newglossaryentry{KB}
{
  name=KB,
  description={\textit{KiloByte} - 1024 bytes},
}

\newglossaryentry{MB}
{
  name=MB,
  description={\textit{MegaByte} - 1024\textsuperscript{2} bytes = 1048576 bytes},
}

\newglossaryentry{GB}
{
  name=GB,
  description={\textit{GigaByte} - 1024\textsuperscript{3} bytes = 1073741824 bytes},
}

\newglossaryentry{TB}
{
  name=TB,
  description={\textit{TeraByte} - 1024\textsuperscript{4} bytes = 1099511627776 bytes},
}

\newglossaryentry{EFF}
{
  name=EFF,
  description={\emph{Electronic Frontier Foundation} por sus siglas en ingl\'{e}s},
}
\newglossaryentry{ECC}
{
  name=ECC,
  description={\emph{Elliptic Curve Cryptography} por sus siglas en ingl\'{e}s},
}
\newglossaryentry{CA}
{
  name={CA},
  description={Autoridad Certificadora, \textit{Certification Authority} por sus siglas en ingl\'{e}s},
}
\newglossaryentry{FHS}
{
  name={FHS},
  description={\textit{Filesystem Hierarchy Standard} por sus siglas en ingl\'{e}s},
}
\newglossaryentry{LSB}
{
  name={LSB},
  description={\textit{Linux Standard Base} por sus siglas en ingl\'{e}s},
}
\newglossaryentry{RTOS}
{
  name={RTOS},
  description={\textit{Real-Time Operating System} por sus siglas en ingl\'{e}s},
}
\newglossaryentry{POSIX}
{
  name={POSIX},
  description={\textit{Portable Operating System Interface} por sus siglas en ingl\'{e}s},
}
%\newglossaryentry{MIME}
%{
%  name={MIME},
%  description={\textit{} por sus siglas en ingl\'{e}s},
%}
\newglossaryentry{RFC}
{
  name={RFC},
  description={\textit{Request for Comments} por sus siglas en ingl\'{e}s},
}
\newglossaryentry{HTTP}
{
  name={HTTP},
  description={\textit{Hypertext Transfer Protocol} por sus siglas en ingl\'{e}s},
}
\newglossaryentry{HTTPS}
{
  name={HTTPS},
  description={\textit{HTTP over SSL} por sus siglas en ingl\'{e}s},
}
\newglossaryentry{SSL}
{
  name={SSL},
  description={\textit{Secure Sockets Layer} por sus siglas en ingl\'{e}s, definido en el \textsc{RFC} 6101 \cite{_rfc_????-4}.},
}
\newglossaryentry{TLS}
{
  name={TLS},
  description={\textit{Transport Layer Security} por sus siglas en ingl\'{e}s},
}
\newglossaryentry{SSH}
{
  name={SSH},
  description={\textit{Secure Shell} por sus siglas en ingl\'{e}s},
}
\newglossaryentry{LDAP}
{
  name={LDAP},
  description={\textit{Lightweight Directory Access Protocol} por sus siglas en ingl\'{e}s},
}
\newglossaryentry{DNS}
{
  name={DNS},
  description={\textit{Domain Name System} por sus siglas en ingl\'{e}s},
}
\newglossaryentry{USB}
{
  name={USB},
  description={\textit{Universal Serial Bus} por sus siglas en ingl\'{e}s},
}
\newglossaryentry{WebDAV}
{
  name={WebDAV},
  description={\textit{Web Distributed Authoring and Versioning} por sus siglas en ingl\'{e}s},
}
\newglossaryentry{RAID}
{
  name={RAID},
  description={\textit{Redundant Array of Independent Disks} por sus siglas en ingl\'{e}s},
}
\newglossaryentry{PHP}
{
  name={PHP},
  description={\textit{PHP Hypertext Preprocessor} por sus siglas en ingl\'{e}s},
}
%\newglossaryentry{NSCD}
%{
%  name={NSCD},
%  description={\textit{Name Server Caching Daemon} por sus siglas en ingl\'{e}s},
%}
%\newglossaryentry{NSLCD}
%{
%  name={NSLCD},
%  description={\textit{} por sus siglas en ingl\'{e}s},
%}
\newglossaryentry{CIS}
{
  name={CIS},
  description={\textit{Center for Internet Security} por sus siglas en ingl\'{e}s},
}
\newglossaryentry{OWASP}
{
  name={OWASP},
  description={\textit{Open Web Application Security Project} por sus siglas en ingl\'{e}s},
}
\newglossaryentry{MTA}
{
  name={MTA},
  description={\textit{Mail Transport Agent} por sus siglas en ingl\'{e}s},
}
\newglossaryentry{PCI-DSS}
{
  name={PCI-DSS},
  description={\textit{Payment Card Industry Data Security Standard} por sus siglas en ingl\'{e}s},
}
\newglossaryentry{ACL}
{
  name={ACL},
  description={\textit{Access Control List} por sus siglas en ingl\'{e}s},
}
\newglossaryentry{URL}
{
  name={URL},
  description={\textit{Uniform Resource Locator} por sus siglas en ingl\'{e}s},
}
\newglossaryentry{UAC}
{
  name={UAC},
  description={\textit{User Account Control} por sus siglas en ingl\'{e}s},
}
\newglossaryentry{PAM}
{
  name={PAM},
  description={\textit{Pluggable Authentication Modules} por sus siglas en ingl\'{e}s},
}
%
\newglossaryentry{UNIX}
{
  name={UNIX},
  description={Familia de sistemas operativos},
}
\newglossaryentry{firewall}
{
  name={Firewall},
  text={firewall},
  description={Dispositivof\'{i}sico o de software que bloquea las conexiones de red entrantes y/o salientes},
}
\newglossaryentry{Proxy}
{
  name={Proxy},
  description={Servidor que funge como intermediario en la conexi\'{o}n e intercambio de datos entre dos equipos},
}
%\newglossaryentry{GECOS}
%{
%  name={GECOS},
%  description={},
%}
%\newglossaryentry{bind}
%{
%  name={bind},
%  description={},
%}
\newglossaryentry{rollback}
{
  name={Rollback},
  text={rollback},
  description={Acci\'{o}n que deshace los cambios realizados},
}
\newglossaryentry{Ruby}
{
  name={Ruby},
  description={Lenguaje de programaci\'{o}n de prop\'{o}sito general. Soporta m\'{u}ltiples paradigmas como funcional, imperativo y orientado a objetos},
}
\newglossaryentry{Gema}
{
  name={Gema},
  description={Paquete de software de \Gls{Ruby}, en este formato se distribuyen las bibliotecas de este lenguaje de programaci\'{o}n},
}
\newglossaryentry{VirtualHost}
{
  name={VirtualHost},
  description={Configuraci\'{o}n de Apache \textsc{HTTPD} que logra diferenciar y aislar diferentes sitios web en un solo servidor f\'{i}sico},
}
\newglossaryentry{script}
{
  name={Script},
  text={script},
  description={Programa escrito en un archivo de texto, que es le\'{i}do y ejecutado por un int\'{e}rprete},
  plural={scripts}
}
\newglossaryentry{frontend}
{
  name={Frontend},
  text={frontend},
  description={Interfaz de usuario},
}
\newglossaryentry{PowerShell}
{
  name={PowerShell},
  description={Lenguaje de programaci\'{o}n e int\'{e}rprete basados en la plataforma .NET, principalmente utilizado para automatizaci\'{o}n de tareas en Windows},
}
\newglossaryentry{bootstrap}
{
  name={Bootstrap},
  text={bootstrap},
  description={\Gls{script} m\'{i}nimo de inicializaci\'{o}n},
}
\newglossaryentry{tarball}
{
  name={Tarball},
  text={tarball},
  description={Formato del archivo contenedor utilizado para almacenar paquetes de software},
}
%\newglossaryentry{SYN}
%{
%  name={SYN},
%  description={},
%}
\newglossaryentry{nmap}
{
  name={nmap},
  description={\Gls{Software} utilizado para revisar los puertos y servicios de un \gls{host}},
}
\newglossaryentry{ownCloud}
{
  name={ownCloud},
  description={\Gls{Software} utilizado para gestionar archivos en un servidor de almacenamiento},
}
\newglossaryentry{aliases}
{
  name={\emph{aliases}},
  text={aliases},
  description={Archivo de configuraci\'{o}n que contiene los alias de correo del sistema, utilizado por el \gls{MTA}},
}
\newglossaryentry{cron}
{
  name={Cron},
  text={cron},
  description={Servicio del sistema que se utiliza para definir y ejecutar tareas programadas},
}
\newglossaryentry{at}
{
  name={AT},
  text={at},
  description={Servicio del sistema que agenda la ejecuci\'{o}n de un comando en una fecha espec\'{i}fica},
}
%\newglossaryentry{hardening}
%{
%  name={hardening},
%  description={Proceso que refuerza la seguridad de un equipo.},
%}
\newglossaryentry{root}
{
  name={\emph{root}},
  text={root},
  description={Usuario administrador en sistemas compatibles con \textsc{UNIX}},
}
\newglossaryentry{loopback}
{
  name={Loopback},
  text={loopback},
  description={Interfaz de red virtual que comunica \'{u}nicamente al \gls{host}, utilizada por procesos internos del sistema operativo},
}
\newglossaryentry{host}
{
  name={Host},
  text={host},
  description={Equipo conectado a una red},
}
\newglossaryentry{localhost}
{
  name={\emph{localhost}},
  text={localhost},
  description={Nombre de red que se refiere al \gls{host} actual, est\'{a} asociado al bloque de direcciones \textsc{IPv4} \texttt{127.0.0.0/8} y a la direcci\'{o}n reservada \texttt{::1} de \textsc{IPv6}},
}
\newglossaryentry{hardware}
{
  name={Hardware},
  text={hardware},
  description={Partes f\'{i}sicas de un equipo de c\'{o}mputo},
}
\newglossaryentry{software}
{
  name={Software},
  text={software},
  description={Sistema operativo y programas instalados en un equipo de c\'{o}mputo},
}
\newglossaryentry{firmware}
{
  name={Firmware},
  text={firmware},
  description={L\'{o}gica programada de un circuito},
}
\newglossaryentry{Malware}
{
  name={Malware},
  text={malware},
  description={\Gls{Software} malicioso},
}
\newglossaryentry{shell}
{
  name={Shell},
  text={shell},
  description={Int\'{e}rprete de comandos del sistema operativo},
}
\newglossaryentry{GNU}
{
  name={GNU},
  description={Acr\'{o}nimo recursivo que significa \textit{GNU is Not \textsc{Unix}}},
}
\newglossaryentry{GPL}
{
  name={GPL},
  description={\gls{GNU} \textit{General Public License}, por sus siglas en ingl\'{e}s},
}
\newglossaryentry{kernel}
{
  name=Kernel,
  text=kernel,
  description={N\'{u}cleo del sistema operativo, ejecuta las tareas en el CPU y administra los recursos del sistema},
}
% IP CIDR banner

