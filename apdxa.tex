{
  \linespread{1}
  \cleardoublepage  
  \appendix
  \chapter{Manuales}
  \label{apdx:a}
}

    \section {Instalaci\'{o}n del certificado ra\'{i}z}
    \label{sec:man-cert}


      \subsection {\textsc{Mozilla Firefox}}
      \label{subsec:man-cert-firefox}

{
\linespread{0.1}
\begin{enumerate}

  \item Abrir el sitio en el navegador web \textsl{Firefox}. Aparecer\'{a} una ventana donde se pregunta la funci\'{o}n del certificado ra\'{iz}, seleccionar la primer opci\'{o}n y dar clic en el bot\'{o}n \textbf{OK}.

    \texttt{http://xnas.tonejito.org/ca.crt}

  \picturebox{figures/Firefox-Cert-001}{565}{396}{0.8}

  \item Despu\'{e}s de instalar el certificado, el sitio web se mostrar\'{a} de manera correcta en el navegador web al accederlo por \textsc{HTTPS}.

    \texttt{https://xnas.tonejito.org/}

  \picturebox{figures/Firefox-Cert-002}{565}{288}{0.8}

\end{enumerate}
}

      \subsection {\textsc{Mac OS X}}
      \label{subsec:man-cert-macosx}

{
\linespread{0.1}
\begin{enumerate}

  \item Abrir el sitio en el navegador web \textsl{Safari} y dar clic en el bot\'{o}n \textsl{Show Certificate}.

    \texttt{https://xnas.tonejito.org/}

%  \picturebox{figures/MacOSX-001}{471}{72}{0.75}
  \picturebox{figures/MacOSX-cert-001}{537}{255}{0.85}

\newpage
  \item Activar la casilla \textsl{Always Trust}. Al terminar dar clic en \textbf{Continue}.

%  \picturebox{figures/MacOSX-002}{536}{370}{0.7}
  \picturebox{figures/MacOSX-cert-002}{536}{258}{0.75}

  \item Escribir la contrase\~{n}a para aplicar los cambios.

  \picturebox{figures/MacOSX-003}{443}{239}{0.5}

  \item Verificar que el sitio cargue a trav\'{e}s de \textsc{HTTPS} y cerrar el navegador.

%  \picturebox{figures/MacOSX-004}{471}{72}{0.75}
  \picturebox{figures/MacOSX-cert-003}{537}{255}{0.85}

\end{enumerate}
}

      \subsection {Windows}
      \label{subsec:man-cert-windows}
{
\linespread{0.1}
\begin{enumerate}
  \item Abrir el navegador de Internet.

%  \picturebox{figures/Windows-001}{359}{39}{0.5}

  \item Escribir la siguiente \textit{URL} en la barra de direcciones y dar \texttt{<Enter>}

    \texttt{http://xnas.tonejito.org/install.cmd}

  \picturebox{figures/Windows-002}{500}{118}{0.8}

  \item El navegador preguntar\'{a} qu\'{e} hacer con el archivo, seleccionar la opci\'{o}n \textbf{guardar}

  \picturebox{figures/Windows-003-ie9}{600}{85}{0.9}

  \item El script de instalaci\'{o}n fue guardado en la carpeta \textsl{Descargas}, dar clic en el bot\'{o}n \textbf{Abrir carpeta}

  \picturebox{figures/Windows-004-ie9}{600}{50}{0.9}

%  \item Una vez guardado el archivo, activar la casilla y dar clic en el bot\'{o}n \textbf{cerrar}
%
%  \picturebox{figures/Windows-005}{419}{318}{0.75}

%\newpage
  \item Dar clic derecho en el archivo \textbf{install.cmd} y seleccionar la opci\'{o}n \textsl{Ejecutar como administrador}

  \picturebox{figures/Windows-006-ie9}{491}{221}{0.7}

  \item Dar clic en el bot\'{o}n \textbf{SI} en el cuadro de di\'{a}logo de \textit{UAC}

  \picturebox{figures/Windows-007}{468}{245}{0.65}

%  \item Esperar a que termine de ejecutar el script y dar clic en el bot\'{o}n \textbf{cerrar}
%
%  \picturebox{figures/Windows-008}{679}{344}{0.5}

  \item Al terminar la instalaci\'{o}n se crear\'{a} un icono llamado \textbf{xNAS} en el escritorio

  \picturebox{figures/Windows-009}{169}{91}{0.25}

\end{enumerate}
}

\newpage

    \section {Acceso de \textsl{s\'{o}lo lectura} para alumnos}
    \label{sec:man-ro}

El sitio tiene la siguiente estructura de directorios para la secci\'{o}n de \textit{s\'{o}lo lectura}. El alumno puede navegar en todos los directorios p\'{u}blicos donde la \textsc{URL} tiene el siguiente formato:

{
 \linespread{1}
 \begin{table}[H]
 \caption{Formato de la \textsc{URL} de la secci\'{o}n de \textit{s\'{o}lo lectura}}{}
 \label{tab:csv-format}
 \noindent\makebox[\textwidth]
 {%
  \begin{tabular}[c]{c|l}
  \textbf{Elemento}                             & \multicolumn{1}{c}{\textbf{Componente de \texttt{URL}}} \\
  \hline \hline
  \textit{\textsc{URL} base}                    & \texttt{https://xnas.tonejito.org/\textrm{\textbf{alumno}}/} \\
  \textit{Nombre del profesor}                  & $\hookrightarrow$ \texttt{andres-leonardo-hernandez-bermudez} \\
  \textit{ID de la materia}                     & \hspace{3mm}$\hookrightarrow$ \texttt{1024} \\
  \textit{ID del grupo}                         & \hspace{6mm}$\hookrightarrow$ \texttt{8} \\
  %\hline
  \multirow{2}{*}{\textbf{Archivos y Carpetas}} & \hspace{9mm}$\Longrightarrow$ \textup{archivos} \\
                                                & \hspace{9mm}$\Longrightarrow$ \textmd{carpetas} \\
  \end{tabular}
 } % ending of \makebox
 \end{table}
}

      \subsection {Navegador Web}
      \label{subsec:man-ro-browser}

{
\linespread{1}
\begin{enumerate}
\setcounter{enumi}{-1} % [start=0]

  \item Instalar el certificado ra\'{i}z en el equipo (ver secci\'{o}n \ref{subsec:man-cert-windows} en la p\'{a}gina \pageref{subsec:man-cert-windows}) y en el navegador web \textsl{Mozilla Firefox}, ver secci\'{o}n \ref{subsec:man-cert-firefox} en la p\'{a}gina \pageref{subsec:man-cert-firefox}

  \item Abrir el sitio en el navegador web

    \texttt{https://xnas.tonejito.org/alumno/}

%  \picturebox{figures/Windows-001}{359}{39}{0.4}

  \item Escribir el n\'{u}mero de cuenta y la contrase\~{n}a en el cuadro de di\'{a}logo. Una vez realizado esto, se podr\'{a} navegar en los directorios.

  \picturebox{figures/Windows-Alumno-Firefox}{610}{282}{0.7}
%  \picturebox{figures/Windows-Alumno-Chrome}{452}{323}{0.7}
%  \picturebox{figures/Windows-Alumno-Opera}{426}{335}{0.7}
%  \picturebox{figures/Windows-Alumno-InternetExplorer}{481}{314}{0.7}

  \item Abrir la carpeta que corresponde al nombre del profesor, materia y grupo. Ver secci\'{o}n \ref{sec:man-ro} en la p\'{a}gina \pageref{sec:man-ro} para conocer la nomenclatura de las carpetas donde los alumnos pueden acceder a los archivos.

\end{enumerate}
}

\newpage

    \section {Acceso de \textsl{lectura-escritura} para profesores}
    \label{sec:man-rw}


El sitio tiene la siguiente estructura de directorios para la secci\'{o}n de \textit{lectura y escritura}. El profesor puede subir archivos al directorio representado por el \emph{ID de materia} y puede generar carpetas adicionales dentro del directorio con el \emph{ID de grupo}:

{
 \linespread{1}
 \begin{table}[H]
 \caption{Formato de la \textsc{URL} de la secci\'{o}n de \textit{lectura y escritura}}{}
 \label{tab:csv-format}
 \noindent\makebox[\textwidth]
 {%
  \begin{tabular}[c]{c|l}
  \textbf{Elemento}                             & \multicolumn{1}{c}{\textbf{Componente de \texttt{URL}}} \\
  \hline \hline
  \textit{\textsc{URL} base}                    & \texttt{https://xnas.tonejito.org/\textrm{\textbf{profesor}}/} \\
  \textit{Nombre del profesor}                  & $\hookrightarrow$ \texttt{andres-leonardo-hernandez-bermudez} \\
  \textit{ID de la materia}                     & \hspace{3mm}$\hookrightarrow$ \texttt{1024} \\
  \textit{ID del grupo}                         & \hspace{6mm}$\hookrightarrow$ \texttt{8} \\
  %\hline
  \multirow{2}{*}{\textbf{Archivos y Carpetas}} & \hspace{9mm}$\Longrightarrow$ \textup{archivos} \\
                                                & \hspace{9mm}$\Longrightarrow$ \textmd{carpetas} \\
  \end{tabular}
 } % ending of \makebox
 \end{table}
}

      \subsection {\textsc{GNU/Linux}}
      \label{subsec:man-rw-gnulinux}

{
\linespread{0.1}
\begin{enumerate}

  \item En el escritorio de \textsl{Gnome} ir al men\'{u} \textsl{Places} y seleccionar la opci\'{o}n \textsl{Connect to Server}.

  \picturebox{figures/Linux-001}{372}{251}{0.75}

\newpage
  \item Escribir los datos de conexi\'{o}n por \textsl{Secure WebDAV}, as\'{i} como las credenciales de usuario y dar clic en el bot\'{o}n \textsl{Connect}.

  \picturebox{figures/Linux-002}{449}{379}{0.6}

  \item Abrir la carpeta que corresponde al nombre del profesor. Ver secci\'{o}n \ref{sec:man-rw} en la p\'{a}gina \pageref{sec:man-rw} para conocer la nomenclatura de las carpetas donde se pueden subir archivos.

\end{enumerate}
}

      \subsection {\textsc{Mac OS X}}
      \label{subsec:man-rw-macosx}

{
\linespread{0.1}
\begin{enumerate}
\setcounter{enumi}{-1} % [start=0]

  \item Instalar el certificado ra\'{i}z en el equipo (ver secci\'{o}n \ref{subsec:man-cert-macosx} en la p\'{a}gina \pageref{subsec:man-cert-macosx}). 

  \item En \textsl{Finder} ir al men\'{u} \textsl{Go} y seleccionar la opci\'{o}n \textsl{Connect to server}.

  \picturebox{figures/MacOSX-005}{458}{243}{0.75}

  \item Escribir la siguiente \textsc{URL} en el campo \textsl{Server Address} y dar clic en el bot\'{o}n \textsl{Connect}.

    \texttt{https://xnas.tonejito.org/profesor/}

  \picturebox{figures/MacOSX-006}{486}{231}{0.75}

  \item Esperar a que se inicialice la conexi\'{n}n.

  \picturebox{figures/MacOSX-007}{399}{90}{0.6}

  \item Para conectar con el servidor se pedir\'{a}n las credenciales de acceso, introducirlas y dar clic en el bot\'{o}n \textsl{Connect}.

  \picturebox{figures/MacOSX-008}{430}{296}{0.6}

  \item Abrir la carpeta que corresponde al nombre del profesor. Ver secci\'{o}n \ref{sec:man-rw} en la p\'{a}gina \pageref{sec:man-rw} para conocer la nomenclatura de las carpetas donde se pueden subir archivos.

  \picturebox{figures/MacOSX-009}{828}{241}{1.0}

\end{enumerate}
}

      \label{man-rw-windows}
      \subsection {Windows}

{
\linespread{0.1}
\begin{enumerate}
\setcounter{enumi}{-1} % [start=0]

  \item Instalar el certificado ra\'{i}z en el equipo (ver secci\'{o}n \ref{subsec:man-cert-windows} en la p\'{a}gina \pageref{subsec:man-cert-windows}).

  \item Dar doble clic en el icono \textbf{xNAS} ubicado en el escritorio

  \picturebox{figures/Windows-009}{169}{91}{0.25}

  \item Escribir la siguiente \textit{URL} en el cuadro de di\'{a}logo y dar clic en el bot\'{o}n \textbf{OK}

    \texttt{http://xnas.tonejito.org/profesor/}

  \picturebox{figures/Windows-010}{302}{202}{0.5}

  \item Escribir las credenciales de acceso y dar clic en el bot\'{o}n \textbf{OK}

  \picturebox{figures/Windows-011}{328}{259}{0.5}

  \item Se muestra el siguiente mensaje cuando la unidad pudo conectarse de manera correcta

  \picturebox{figures/Windows-012}{482}{161}{0.75}

  \item Iniciar el explorador de windows, seleccionar la unidad de red y dar doble clic

  \picturebox{figures/Windows-013}{800}{238}{1.0}

  \item Abrir la carpeta que corresponde al nombre del profesor. Ver secci\'{o}n \ref{sec:man-rw} en la p\'{a}gina \pageref{sec:man-rw} para conocer la nomenclatura de las carpetas donde se pueden subir archivos.

  \picturebox{figures/Windows-014}{800}{223}{1.0}

%  \item Escribir en las subcarpetas de materia y grupo
%
%  \picturebox{figures/Windows-015}{800}{253}{0.75}

\end{enumerate}
}

