{
  \linespread{1}
  \cleardoublepage  
  \appendix
  \chapter{Manuales}
  \label{apdx:a}
}

%    \section {\textsc{GNU/Linux}}

    \section {\textsc{Mac OS X}}

{
\linespread{0.1}
\begin{enumerate}

  \item Abrir el sitio en el navegador web \textsl{Safari}.

    \texttt{https://xnas.tonejito.org/}

  \picturebox{figures/MacOSX-001}{800}{72}{0.75}

  \item Seleccionar la secci\'{o}n \textit{Trust} y aplicar la configuraci\'{o}n \textsl{Always Trust} en la primer opci\'{o}n. Al terminar dar clic en \textbf{Continue}.

  \picturebox{figures/MacOSX-002}{536}{497}{0.75}

  \item Escribir la contrase\~{n}a para aplicar los cambios.

  \picturebox{figures/MacOSX-003}{443}{239}{0.75}

  \item Verificar que el sitio cargue a trav\'{e}s de \textsc{HTTPS} y cerrar el navegador.

  \picturebox{figures/MacOSX-004}{800}{72}{0.75}

  \item En \textsl{Finder} ir al men\'{u} \textsl{Go} y seleccionar la opci\'{o}n \textsl{Connect to server}.

  \picturebox{figures/MacOSX-005}{458}{391}{0.75}

  \item Escribir la siguiente \textsc{URL} en el campo \textsl{Server Address} y dar clic en el bot\'{o}n \textsl{Connect}.

    \texttt{https://xnas.tonejito.org/profesor/}

  \picturebox{figures/MacOSX-006}{486}{231}{0.75}

  \item Esperar a que se inicialice la conexi\'{n}n.

  \picturebox{figures/MacOSX-007}{399}{90}{0.75}

  \item Para conectar con el servidor se pedir\'{a}n las credenciales de acceso, introducirlas y dar clic en el bot\'{o}n \textsl{Connect}.

  \picturebox{figures/MacOSX-008}{430}{296}{0.75}

  \item El profesor puede escribir en su cuenta detro de los directorios asignados a las materias y grupos.

  \picturebox{figures/MacOSX-009}{828}{339}{0.75}

\end{enumerate}
}

    \section {Windows}

%This is \appendixname~\ref{apdx:a}.
%
%You can have additional appendices too
%(\emph{e.g.}, \texttt{apdxb.tex}, \texttt{apdxc.tex}, \emph{etc.}).
%If you don't need any appendices, delete the appendix
%related lines from \texttt{thesis.tex} and the file names
%from \texttt{Makefile}.

{
\linespread{0.1}
\begin{enumerate}

  \item Abrir el navegador de Internet.

  \picturebox{figures/Windows-001}{248}{61}{0.75}

  \item Escribir la siguiente \textit{URL} en la barra de direcciones y dar \texttt{<Enter>}

    \texttt{http://xnas.tonejito.org/install.cmd}

  \picturebox{figures/Windows-002}{500}{118}{0.75}

  \item El navegador preguntar\'{a} qu\'{e} hacer con el archivo, seleccionar la opci\'{o}n \textbf{guardar}

  \picturebox{figures/Windows-003-ie9}{600}{85}{0.75}

  \item El script de instalaci\'{o}n fue guardado en la carpeta \textsl{Descargas}, dar clic en el bot\'{o}n \textbf{Abrir carpeta}

  \picturebox{figures/Windows-004-ie9}{600}{50}{0.75}

%  \item Una vez guardado el archivo, activar la casilla y dar clic en el bot\'{o}n \textbf{cerrar}
%
%  \picturebox{figures/Windows-005}{419}{318}{0.75}

  \item Dar clic derecho en el archivo \textbf{install.cmd} y seleccionar la opci\'{o}n \textsl{Ejecutar como administrador}

  \picturebox{figures/Windows-006-ie9}{491}{221}{0.75}

  \item Dar clic en el bot\'{o}n \textbf{SI} en el cuadro de di\'{a}logo de \textit{UAC}

  \picturebox{figures/Windows-007}{468}{245}{0.75}

%  \item Esperar a que termine de ejecutar el script y dar clic en el bot\'{o}n \textbf{cerrar}
%
%  \picturebox{figures/Windows-008}{679}{344}{0.75}

  \item Dar doble clic en el icono \textbf{xNAS} ubicado en el escritorio

  \picturebox{figures/Windows-009}{169}{91}{0.75}

  \item Escribir la siguiente \textit{URL} en el cuadro de di\'{a}logo y dar clic en el bot\'{o}n \textbf{OK}

    \texttt{http://xnas.tonejito.org/profesor/}

  \picturebox{figures/Windows-010}{302}{202}{0.75}

  \item Escribir las credenciales de acceso y dar clic en el bot\'{o}n \textbf{OK}

  \picturebox{figures/Windows-011}{328}{259}{0.75}

  \item Se muestra el siguiente mensaje cuando la unidad pudo conectarse de manera correcta

  \picturebox{figures/Windows-012}{482}{161}{0.75}

  \item Abrir el explorador de windows, seleccionar la unidad de red y dar doble clic

  \picturebox{figures/Windows-013}{800}{238}{0.75}

  \item Seleccionar la carpeta de usuario.

  \picturebox{figures/Windows-014}{800}{223}{0.75}

  \item Escribir en las subcarpetas de materia y grupo

  \picturebox{figures/Windows-015}{800}{253}{0.75}

\end{enumerate}
}

