{
  \linespread{1}
  \cleardoublepage  
  \chapter{Pruebas}
  \label{chap:cap4}
}

Con el objeto de dar soporte a las plataformas m\'{a}s comunes, se realizaron pruebas sobre los principales sistemas operativos de escritorio y m\'{o}viles. Gracias a que el protocolo \textsc{\gls{WebDAV}} es est\'{a}ndar, s\'{o}lo es cuesti\'{o}n de instalar un cliente en el sistema operativo o verificar si esta funcionalidad ya est\'{a} incorporada para acceder a servidores de archivos de este tipo.

    \section {Plan de pruebas}

A continuaci\'{o}n se muestra una lista de las pruebas que se realizaron en el servidor y la descripci\'{o}n de cada una:

\begin{itemize}

  \item \textbf{Pruebas de compatibilidad multiplataforma}

Se verific\'{o} que los usuarios pudieran establecer la conexi\'{o}n al servidor siguiendo los manuales mostrados en el \textup{Ap\'{e}ndice \ref{apdx:a}}.

  \item \textbf{Pruebas de roles de usuario}

Se revis\'{o} que los perfiles de usuario tuvieran el acceso correspondiente a los archivos y los privilegios necesarios para realizar sus actividades.

  \item \textbf{Pruebas de seguridad web}

En esta secci\'{o}n se valid\'{o} que el servidor no contara con servicios adicionales escuchando en puertos de red, adem\'{a}s se realiz\'{o} una prueba automatizada de las cuentas de usuario y sus privilegios de escritura, por \'{u}ltimo se verific\'{o} el nivel de protecci\'{o}n que ofrecen los par\'{a}metros de cifrado configurados en Apache \textsc{HTTPD}.

\end{itemize}

\newpage
    \section {Compatibilidad multiplataforma}

En el \textup{Ap\'{e}ndice \ref{apdx:a}} se muestran los pasos necesarios para realizar la conexi\'{o}n al servidor \textsc{\gls{WebDAV}} tanto en sistemas operativos de escritorio como en plataformas de m\'{o}viles.

    \section {Pruebas de roles de usuario}

El sistema de almacenamiento presenta tres perfiles de usuario: staff, profesor y alumno. Los dos primeros tienen privilegios de lectura-escritura y el \'{u}ltimo tiene acceso de s\'{o}lo lectura.

El personal de la \emph{Unidad de C\'{o}mputo} verific\'{o} que cada tipo de usario cumpliera con los requerimientos de acceso y adem\'{a}s que la conexi\'{o}n al servidor se pudiera realizar de manera exitosa siguiendo los manuales presentados en el \textup{Ap\'{e}ndice \ref{apdx:a}}.

{
 \linespread{1}
 \begin{table}[H]
 \caption{Perfiles de usuario y tipo de acceso}{}
 \label{tab:user-profiles}
 \noindent\makebox[\textwidth]
 {%
  \begin{tabular}[c]{c|l}
  \textbf{Perfil de usuario}      & \multicolumn{1}{c}{\textbf{Descripci\'{o}n}} \\
  \hline \hline
  \multirow{2}{*}{\textit{Staff}} & Acceso de lectura y escritura a su carpeta de usuario \\
                                  & Acceso de lectura a la secci\'{o}n de profesores \\
  \hline
  \textit{Profesor}               & Acceso de lectura y escritura a su carpeta de usuario \\
  \hline
  \textit{Alumno}                 & Acceso de s\'{o}lo lectura a las carpetas de profesor \\
  %\hline
  \end{tabular}
 } % ending of \makebox
 \end{table}
}

    \section {Pruebas de seguridad}

      \subsection {Detecci\'{o}n de puertos abiertos}

Se ejecut\'{o} una b\'{u}squeda de puertos abiertos en el servidor utilizando el programa \emph{nmap}. El siguiente comando realiza la verificaci\'{o}n de todo el rango de puertos \textsc{TCP} enviando paquetes de tipo \textsc{SYN} y adem\'{a}s intenta obtener la versi\'{o}n del programa que escucha en el puerto encontrado.

%, los argumentos indican que no se debe de hacer \emph{ping} al \textit{host} (\texttt{-Pn}), se realiza una verificaci\'{o}n de los puertos \textsc{TCP} enviando un paquete de tipo \textsc{SYN} (\texttt{-sS}), se intenta obtener la versi\'{o}n del software que escucha en el puerto (\texttt{-sV} y \texttt{--version-all}), se realiza una b\'{u}squeda con el perfil r\'{a}pido (\texttt{-T5}) y se verifica todo el rango de puertos (\texttt{-p 0-65535}), por \'{u}ltimo se guarda el reporte (\texttt{-oA}).

{
\scriptsize
\linespread{1}
\begin{verbatim}
# nmap -Pn -sS -sV --version-all -T 5 -p 0-65535 -oA xnas xnas.tonejito.org

Starting Nmap 6.00 ( http://nmap.org ) at 2015-09-28 00:00 EDT
Nmap scan report for xnas.tonejito.org (132.248.139.147)
Host is up (0.064s latency).
Not shown: 65533 closed ports
PORT    STATE SERVICE  VERSION
22/tcp  open  ssh      OpenSSH 6.0p1 (protocol 2.0)
80/tcp  open  http     Apache httpd
443/tcp open  ssl/http Apache httpd

Service detection performed. Please report any incorrect results at http://nmap.org/submit/
Nmap done: 1 IP address (1 host up) scanned in 256.32 seconds
\end{verbatim}
}

% Se encontraron tres servicios: \textsc{\gls{SSH}}, \textsc{\gls{HTTP}} y \textsc{\gls{HTTPS}}. 

      \subsection {Autenticaci\'{o}n}

Las pruebas de autenticaci\'{o}n se realizaron de manera automatizada, se utiliz\'{o} el siguiente algoritmo para iterar entre los usuarios y carpetas utilizando el cliente \textsc{\gls{WebDAV}} \textsl{cadaver}:

{
\scriptsize
\linespread{1}
\begin{verbatim}
  para cada usuario u
    para cada carpeta c
      intenta acceder como el usuario u a la carpeta c
      el servidor dio acceso
        agrega el usuario y la carpeta a la lista de accesos exitosos
      caso contrario
        agrega el usuario y la carpeta a la lista de accesos fallidos
      fin de condicional
    fin de ciclo de carpetas
  fin de ciclo de usuarios
\end{verbatim}
}

      \subsection {Par\'{a}metros de cifrado para \textsc{HTTPS}}

Al realizar una verificaci\'{o}n de seguridad del cifrado \textsc{\gls{SSL}} utilizado en el \textsl{appliance}, se obtuvo una calificaci\'{o}n regular \textbf{C} y se encontraron los siguientes problemas con la configuraci\'{o}n predeterminada:

\begin{itemize}
  \item Vulnerabilidad al ataque \texttt{POODLE}\footnote{Padding Oracle On Downgraded Legacy Encryption} \cite{_ssl_????-1} \cite{_ssl-poodle.pdf_????} \cite{_this_????} \cite{barnes_poodle_????} \cite{_security_????-1}

  \item El servidor soporta el algoritmo cifrado \texttt{RC4}, mismo que es clasificado como d\'{e}bil \cite{_security_????}

  \item Est\'{a} habilitado el soporte de \textsc{SSLv3}

  \item La configuraci\'{o}n de \textsc{\gls{SSL}} aplicada no permite el uso de \textsl{Perfect Forward Secrecy} \cite{_forward_2016}
\end{itemize}

A comtinuaci\'{o}n se muestra una captura de pantalla con los resultados obtenidos tras la primer prueba:

  \picturebox{figures/SSL-Labs_Test-Default}{772}{476}{0.8}{ssl-test-default}{Prueba de par\'{a}metros de cifrado \textsc{HTTPS} est\'{a}ndar}{}

Para resolver los problemas encontrados se tomaron en cuenta los requerimientos para \textsc{\gls{SSL}} del est\'{a}ndar \textsc{\gls{PCI-DSS}}, deshabilitando por completo el soporte de \textsc{RC4} y \textsc{SSLv3}.

Gracias a la configuraci\'{o}n aplicada se obtuvo una calificaci\'{o}n buena \textbf{A}. En la siguiente imagen se muestra el resultado de las pruebas de seguridad SSL donde la \'{u}nica advertencia mostrada es referente a la autoridad que realiz\'{o} la firma del certificado. \cite{_ssl_????}

  \picturebox{figures/SSL-Labs_Test-Enhanced}{772}{346}{0.8}{ssl-test-enhanced}{Prueba de par\'{a}metros de cifrado \textsc{HTTPS} reforzados}{}

