{
  \linespread{1}
  \cleardoublepage  
  \chapter{Pruebas}
  \label{chap:cap4}
}

Con el objeto de dar soporte a las plataformas m\'{a}s comunes, se realizaron pruebas sobre los principales sistemas operativos de escritorio y m\'{o}viles. Gracias a que el protocolo \textsc{WebDAV} es est\'{a}ndar, solo es cuesti\'{o}n de instalar un cliente en el sistema operativo o verificar si incorpora la funcionalidad para acceder a servidores de archivos \textsc{WebDAV}.

    \section {Compatibilidad multiplataforma}

Para cada sistema operativo se lista si la conexi\'{o}n con \textsc{WebDAV} se realiza a trav\'{e}s de los mecanismos nativos o si se debe instalar un cliente adicional para habilitar la conexi\'{o}n. Tambi\'{e}n se listan los pasos necesarios para realizar la conexi\'{o}n al servidor.

      \subsection {\textsc{GNU/Linux}}

En \textsc{GNU/Linux} la funcionalidad de conexi\'{o}n a \textsc{WebDAV} ya viene incluida en el entorno de escritorio \textit{Gnome}, para realizar la conexi\'{o}n al servidor se deben realizar los siguientes pasos:

\begin{itemize}
  \item Abrir \textsl{Nautilus}.
  \item Seleccionar el men\'{u} \textsl{Go} y la opci\'{o}n \textsl{Connect to server...}
  \item Escribir el nombre del servidor \texttt{xnas.local}, seleccionar el tipo de conexi\'{o}n \textsl{Secure WebDAV (https)} y escribir el nombre de usuario y la contrase\~{n}a.
  \item Seleccionar la opci\'{o}n \textsl{Remember for this session} para recordar la contrase\~{n}a mientras el usuario tiene la sesi\'{o}n gr\'{a}fica abierta.
  \item Dar clic en el bot\'{o}n \textsl{OK} para establecer la conexi\'{o}n al servidor de archivos.
\end{itemize}

      \subsection {Mac OS X}

El soporte nativo para \textsc{WebDAV} est\'{a} incluido en \textsc{Mac OS X} desde las primeras versiones, los pasos a seguir para establecer una conexi\'{o}n son:

\begin{itemize}
  \item Abrir una nueva ventana de \textsl{Finder} \cmdkey + n
  \item Ir al men\'{u} \textsl{Go} y a la opci\'{o}n \textsl{Connect to server...}, o bien presionar \cmdkey + k
  \item Escribir la URL del servidor en el cuadro de texto \texttt{https://xnas.local/profesor} y dar clic en \textsl{Connect}
  \item El servidor pedir\'{a} las credenciales de acceso y al aceptarlas la unidad ser\'{a} visible en la barra lateral de \textsl{Finder}
\end{itemize}

      \subsection {Apple iOS}
      \subsection {Android}
      \subsection {Windows}
    \section {Pruebas de carga al sistema}
      \subsection {N\'{u}mero m\'{a}ximo de usuarios}
    \section {Pruebas de seguridad}
      \subsection {Autenticaci\'{o}n}
      \subsection {\textit{Directory Traversal}}
